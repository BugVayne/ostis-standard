\begin{SCn}

\scnsectionheader{Предметная область и онтология Технологии OSTIS}
\scnrelfromlist{подраздел}{Предметная область и онтология семантических моделей компьютерных систем, построенных по Технологии OSTIS (ostis-систем);Предметная область и онтология действий и методик разработки ostis-систем;Предметная область и онтология средств разработки ostis-систем;Предметная область и онтология Экосистемы OSTIS – коллектива ostis-систем, поддерживающих свою семантическую совместимость}

\scnstartsubstruct

\scnheader{Предметная область Технологии OSTIS}
\scnsdmainclasssingle{***}
\scnsdclass{***}
\scnsdrelation{***}

\scnheader{Технология OSTIS}
\scnexplanation{Предлагаемая нами технология разработки семантических компьютерных систем названа \textbf{\textit{Технологией OSTIS}} (Open Semantic Technology for Intelligent Systems).

В основе этой технологии лежит \textbf{\textit{SC-код}} -- разработанный нами стандарт смыслового представления информации в памяти компьютерных систем.

В целом, \textit{Технология OSTIS} -- это
\begin{scnitemize}
    \item \textbf{cтандарт} \textit{семантических компьютерных систем}, обеспечивающий семантическую совместимость систем, соответствующих этому стандарту;
    \item \textbf{методы построения} таких компьютерных систем и их совершенствования в процессе их эксплуатации;
    \item \textbf{cредства построения} и совершенствования этих систем
     \begin{scnitemizeii}
        \item языковые средства;
        \item библиотеки типовых технических решений;
        \item инструментальные средства
        \begin{scnitemizeiii}
            \item cредства синтеза и модификации;
            \item средства анализа, верификации, диагностики, тестирования;
            \item средства устранения обнаруженных ошибок и недостатков.
        \end{scnitemizeiii}
    \end{scnitemizeii}
\end{scnitemize}


Существенно подчеркнуть, что \textit{Технология OSTIS} -- это не просто \textbf{стандарт семантических компьютерных систем}, а стандарт, который постоянно и интенсивно совершенствуется в ходе постоянного расширения и совершенствования формализации используемых видов знаний и моделей решения задач путем достижения консенсуса (согласования точек зрения) с участием всех заинтересованных физических и юридических лиц.

Принципиальным является то, что \textit{Технология OSTIS} позволяет создавать системы, которые вовсе не обязательно должны решать \textit{интеллектуальные задачи}, но такая реализация \textit{компьютерных систем} обеспечивает:
\begin{scnitemize}
    \item их совместимость;
    \item высокую степень их гибкости, что позволяет неограниченным образом расширять функциональные возможности компьютерных систем вплоть до возможности решать \textit{интеллектуальные задачи}.
\end{scnitemize}
}

\scnheader{Технология OSTIS}
\scnprinciples{\begin{scnitemize}
    \item ориентация на смысловое однозначное представление знаний в виде семантических сетей, имеющих базовую теоретико-множественную интерпретацию, что обеспечивает решение проблемы многообразия форм представления одного и того же смысла, и проблемы неоднозначности семантической интерпретации информационных конструкций;
    \item использование ассоциативной графодинамической модели памяти;
    \item применение агентно-ориентированной модели обработки знаний;
    \item реализация \textit{Технологии OSTIS} в виде интеллектуальной \textit{\textbf{Метасистемы IMS.ostis}}, которая сама построена по \textit{Технологии OSTIS} и осуществляет поддержку проектирования компьютерных систем, разрабатываемых по \textit{Технологии OSTIS};
    \item обеспечение в проектируемых системах высокого уровня гибкости, стратифицированности, рефлексивности, гибридности, совместимости и, как следствие, обучаемости.
\end{scnitemize}}

\scnheader{Технология OSTIS}
\scnadvantages{\begin{scnitemize}
    \item \textit{Технология OSTIS} имеет открытый характер как для ее пользователей (разработчиков прикладных интеллектуальных систем), так и для тех, кто желает участвовать в ее совершенствовании;
    \item \textit{Технология OSTIS} ориентирована на постоянное повышение темпов ее эволюции;
    \item \textit{Технология OSTIS} является основой для решения проблем семантической совместимости самых различных научных и технических знаний, так как она ориентирована на формализацию междисциплинарных связей самого различного вида.
\end{scnitemize}}

\scnheader{Технология OSTIS}
\scnidtf{OSTIS}
\scnidtf{Open Semantic Technology for Intelligent Systems}
\scnidtf{Открытая семантическая технология проектирования интеллектуальных компьютерных систем}
\scnidtf{Технология онтологического проектирования семантически совместимых гибридных интеллектуальных компьютерных систем}
\scnidtf{Комплексная массовая семантическая технология модульного (крупноблочного, сборочного) проектирования интеллектуальных систем}
\scnidtf{семантическая технология сборочного проектирования интеллектуальных систем на основе интегрированной библиотеки ip-компонентов интеллектуальных систем}
\scnaddlevel{1}
\begin{adjustwidth}{0.4em}{0em}
\scnnote{т.е. таких компонентов интеллектуальных систем, которые могут быть многократно использованы в самых различных интеллектуальных системах и которые являются компонентами интеллектуальной собственности (ip-компонентами -- intellectual property components)}
\end{adjustwidth}
\scnaddlevel{-1}
\scnidtf{Семантическая теория и технология разработки совместимых гибридных интеллектуальных систем}
\scnidtf{Интеграция современных результатов в области искусственного интеллекта в рамках общей формальной теории интеллектуальных систем и комплексной технологии их разработки}
\scnidtf{Теория гибридных семантически совместимых интеллектуальных систем и технология их онтологического проектирования}
\scnidtf{Семантическая теория и технология разработки экосистемы семантически совместимых гибридных интеллектуальных систем}
\scnidtf{Семантическая теория и технология разработки гибридных совместимых интеллектуальных систем}
\scnidtf{Открытая (интенсивно эволюционируемая) технология модульного (крупноблочного) проектирования (и реализации) гибридных семантически (!) совместимых интеллектуальных систем}


\scnendstruct

\end{SCn}