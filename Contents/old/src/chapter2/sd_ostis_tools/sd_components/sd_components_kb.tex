\begin{SCn}

\scnsectionheader{\currentname}
\scnsuperset{Предметная область многократно используемых компонентов баз знаний ostis-систем}
\scnaddlevel{1}
\scnsdmainclasssingle{...}
\scnsdclass{}
\scnsdrelation{}
\scnaddlevel{-1}

\scnstartsubstruct

\scnheader{Библиотека многократно используемых компонентов sc-моделей баз знаний}
\scnidtf{многократно используемый компонент sc-моделей баз знаний}
\scnexplanation{Каждый \textbf{\textit{многократно используемый компонент sc-моделей баз знаний}} представляет собой \textit{структуру}, либо явно представленную в текущем состоянии \textit{sc-памяти}, либо не полностью сформированную \textit{структуру}, которая при необходимости может быть полностью сформирована путем объединения своих частей, указанных при помощи какого-либо \textit{отношения декомпозиции}, например \textit{разбиение*}, или отношения \textit{включение*}.

Интеграция \textbf{\textit{многократно используемого компонента sc-моделей баз знаний}} в дочернюю систему сводится к склеиванию ключевых узлов по идентификаторам и устранению возможных дублирований и противоречий, которые могли возникнуть в случае, если разработчик дочерней системы вручную вносил какие-либо изменения в ее базу знаний.

К основным типам компонентов баз знаний, хранящихся в библиотеке компонентов баз знаний, относятся:
\begin{scnitemize}
    \item онтологии различных предметных областей, которые могут быть самыми различными по содержанию, однако должны быть семантически совместимыми;
    \item базовые фрагменты теорий, соответствующие различным уровням знания пользователя, начиная от базового школьного до профессионального;
    \item различные \textit{семантические окрестности} различных объектов;
    \item спецификации формальных \textit{sc-языков}, соответствующих различным \textit{предметным областям}.
\end{scnitemize}

Для обеспечения семантической совместимости компонентов баз знаний, которые являются унифицированными семантическими моделями, необходимо
\begin{scnitemize}
    \item согласовать семантику всех используемых ключевых узлов;
    \item согласовать \textit{системные идентификаторы*} ключевых узлов, используемых в разных компонентах. После этого интеграция всех компонентов, входящих в состав библиотеки, и в любых комбинациях осуществляется автоматически, без вмешательства разработчика.
\end{scnitemize}
Для включения компонента в библиотеку необходимо его специфицировать по следующим критериям:
\begin{scnitemize}
    \item предметная область, описание которой содержится в компоненте;
    \item класс (тип) компонента базы знаний;
    \item состав компонента;
    \item количественные характеристики ключевых узлов компонента;
    \item информация о разработчиках компонента;
    \item дата создания компонента;
    \item информация о верификации компонента;
    \item версия компонента;
    \item условия распространения компонента базы знаний;
    \item сопровождающая информация.
\end{scnitemize}
}

\scnendstruct

\end{SCn}