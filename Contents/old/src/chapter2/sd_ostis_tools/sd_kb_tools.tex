\begin{SCn}

\scnsectionheader{Предметная область и онтология средств поддержки проектирования баз знаний ostis-систем}
\scnrelfromlist{подраздел}{Семантическая модель средств понимания информации, приобретаемой ostis-системами;Семантическая модель средств обнаружения и анализа противоречий в базах знаний ostis-систем;
Семантическая модель средств обнаружения и спецификации
информационных дыр в базах знаний ostis-систем; Семантическая модель средств автоматизированного управления взаимодействием менеджеров, авторов и рецензентов проектируемых баз знаний ostis-систем}

\scnstartsubstruct

\scnheader{Предметная область средств поддержки проектирования баз знаний ostis-систем}
\scnsdmainclasssingle{***}
\scnsdclass{***}
\scnsdrelation{***}

\scnheader{sc-модель понимания}
\scnexplanation{Очевидно, что формализация \textbf{смыслового представления информации} в памяти компьютерной системы существенно упрощает уточнение того, как происходит процесс понимания новой информации, поступающей на вход компьютерной системы, либо генерируемой в процессе обработки информации. Этот процесс можно разбить на три этапа:

\begin{scnitemize}
    \item \textbf{трансляция} информации с некоторого внешнего языка на внутренней смысловой язык (\textbf{\textit{SC-код}}). Этот этап отсутствует, если новая информация не вводится извне, а непосредственно генерируется в памяти компьютерной системы;
    \item \textbf{погружение} новой информации, представленной в виде \textit{sc-текста} в текущее состояние информационного ресурса, хранимого в памяти компьютерной системы и представленного также в виде \textit{sc-текста}; 
    \item \textbf{выравнивание} (согласование) понятий, используемых в новой вводимой извне или сгенерированной информационной конструкции, с понятиями, используемыми в текущем состоянии хранимого в памяти компьютерной системы информационного ресурса. 
\end{scnitemize}

Рассмотрим каждый из перечисленных этапов подробнее.

\textbf{Трансляция} информации с некоторого внешнего языка в \textit{SC-код} упрощается благодаря тому, что:

\begin{scnitemize}
    \item средствами \textit{SC-кода} можно описать \textbf{синтаксис} внешнего языка, т.к. универсальность \textit{SC-кода} позволяет с его помощью и с любой степенью детализации описывать любые объекты, в том числе, и такие сложные системы внешней среды компьютерных систем, как внешние языки;
    \item процесс \textbf{синтаксического анализа} исходного текста внешнего языка можно выполнить путем манипуляции текстами \textit{SC-кода} и в результате получить описание структуры исходного текста, имеющее достаточную полноту (детализацию) для последующей генерации семантически эквивалентного ему текста \textit{SC-кода};
    \item cредствами \textit{SC-кода} можно описать \textbf{семантику} внешнего языка, трактуя ее как описание свойств морфизмов между \textit{sc-текстами}, описывающими синтаксическую структуру исходных внешних текстов, и \textit{sc-текстами}, которые семантически эквивалентны этим исходным текстам;
    \item процесс \textbf{генерации sc-текста, семантически эквивалентного исходному} внешнему тексту, также можно выполнить путем манипуляции \textit{sc-текстами}.
\end{scnitemize}

Таким образом, эффективность применения \textit{SC-кода} для трансляции текста с некоторого внешнего языка в \textit{SC-код} обусловлено тем, что с помощью \textit{SC-кода} можно описать и синтаксис и семантику внешнего языка. Можно осуществлять синтаксический анализ внешнего текста и последующую генерацию \textit{sc-текста}, семантически эквивалентного исходному внешнему тексту, оставаясь в рамках \textit{SC-кода}.

\textbf{Погружение} (интеграция) нового сгенерированного \textit{sc-текста} в состав заданного \textit{sc-текста} (например, в состав базы знаний, представленной в \textit{SC-коде}) сводится к \textbf{склеиванию} (отождествлению) некоторых \textit{sc-элементов} нового \textit{sc-текста} с синонимичными им \textit{sc-элементами}, входящими в состав заданного\textit{ sc-текста}. Таким образом, задача погружения нового \textit{sc-текста} в состав заданного \textit{sc-текста} сводится к задаче построения множества пар синонимичных \textit{sc-элементов}, один из которых входит в состав нового погружаемого \textit{sc-текста}, а второй -- в состав заданного \textit{sc-текста}.

Установление пар синонимичных \textit{sc-элементов} осуществляется:
\begin{scnitemize}
    \item путем поиска пар \textit{sc-элементов}, у которых совпадают \underline{согласованные} внешние имена (подчеркнем при этом, что все используемые понятия \underline{обязаны} иметь соответствующие им согласованные внешние имена);
    \item путем логических рассуждений, использующих логические формулы следующих видов:
     \begin{scnitemizeii}
        \item формулы о несуществовании;
        \item формулы о существовании и единственности;
        \item формулы о существовании конечного и указываемого числа значений соответствующих переменных.
    \end{scnitemizeii}
\end{scnitemize}

Для упрощения установления пар синонимичных sc-элементов некоторые высказывания о несуществовании, о существовании и единственности, о существовании заданного конечного числа структур заданного вида можно переформулировать в более "конструктивном"\ ключе с явным введением отношения \textit{\textbf{синонимии sc-элементов}}. Так, например, вместо утверждения о том, что ``Для каждой пары точек существует единственная проходящая через них прямая'' можно использовать следующую формулировку: ``Если прямые \textit{\textbf{pi}} и \textit{\textbf{pj}} проходят через точки \textit{\textbf{ti}} и \textit{\textbf{tj}}, то либо \textit{\textbf{pi}} = \textit{\textbf{pj}}, либо \textit{\textbf{ti}} = \textit{\textbf{tj}}, либо \textit{\textbf{ti}} $\notin$ \textit{\textbf{pi}}, либо \textit{\textbf{ti}} $\notin$ \textit{\textbf{pj}}, либо \textit{\textbf{tj}} $\notin$ \textit{\textbf{pi}}, либо либо \textit{\textbf{tj}} $\notin$ \textit{\textbf{pj}}''. 

Достаточно подробное описание примера погружения \textit{sc-текста} в базу знаний, представленную также в в \textit{SC-коде}, приведено в разделе IX статьи~\cite{Golenkov2018} - Пример 4.

\textbf{Выравнивание понятий}, используемых в новом интегрируемом (вводимом, погружаемом) sc-тексте, с понятиями, используемыми в заданном интегрирующем \textit{sc-тексте}, осуществляется следующим образом:
\begin{scnitemize}
    \item Заданный интегрирующий \textit{sc-текст} (обычно это база знаний, представленная в \textit{SC-коде}) должен явно содержать:
    \begin{scnitemizeii}
        \item информацию о текущем статусе (состоянии, характере) использования каждого известного базе знаний понятия, используемого либо непосредственно в самой базе знаний, либо внешними субъектами, информация от которых может поступать на вход указанной базы знаний;
        \item информацию о текущем статусе (состоянии, характере) использования каждого внешнего знака (чаще всего термина, имени), соответствующего каждому используемому понятию, а также некоторым общеизвестным сущностям, которые не являются понятиями;
    \end{scnitemizeii}
    \item Интегрируемый (вводимый, погружаемый) текст должен:
    \begin{scnitemizeii}
        \item максимально возможным образом использовать \textbf{согласованные понятия} и соответствующие им \textbf{согласованные внешние знаки} (термины, имена);
        \item включать в себя \textbf{определения} всех понятий, которые являются новыми, неизвестными в интегрирующем тексте (при этом в определении должны использоваться только те понятия, которые известны интегрирующему тексту);
    \end{scnitemizeii}
    \item Для решения задачи \textbf{выравнивания} используемых понятий для текущего состояния базы знаний и для нового вводимого (интегрируемого) в эту базу знаний текста все используемые в базе знаний понятия делятся на:
    \begin{scnitemizeii}
        \item согласованные (признанные) на текущий момент и не меняющие своего статуса;
        \item устаревшие = понятия, бывшие в употреблении и редко используемые;
        \item устаревающие = понятия, для которых в течение заданного отрезка времени происходит замена их статуса из статуса согласованного понятия в статус отклоненного понятия;
        \item возвращаемые = понятия, статус которых меняется из статуса отклоненного понятия в статус согласованного понятия;
        \item предложенные новые понятия = новые понятия, проходящие согласование = понятия, статус которых меняется из статуса предложенных в статус либо одобренных, либо отклоненных = согласуемые понятия; 
        \item одобренные понятия = понятия, которые успешно прошли согласование;
        \item отклоненные понятия = понятия, результаты согласования которых отрицательны; 
        \item вводимые новые понятия = понятие, статус которых меняется и статуса одобренного понятия в статус согласованного понятия = понятия, вводимые в употребление.
    \end{scnitemizeii}
\end{scnitemize}

Таким образом, процесс выравнивания понятий, целью которого является сведение всех понятий, используемых в интегрируемом \textit{sc-тексте}, к согласованным понятиям \textit{базы знаний}, осуществляется \textbf{в условиях постоянного изменения статуса используемых понятий} и постоянного увеличения числа таких понятий. 

При этом следует отличать:

\begin{scnitemize}
    \item cемейство всех понятий, известных \textit{базе знаний} в текущий момент;
    \item текущее состояние статуса всех этих понятий;
    \item множество всех переходных процессов, направленных на изменение статуса понятий и осуществляемых в текущий момент.
\end{scnitemize}

Заметим также, что перманентный процесс согласования всех используемых понятий является необходимым условием обеспечения совместимости (интегрируемости) текстов \textit{SC-кода}. Но для обеспечения совместимости текстов \textit{SC-кода} необходим перманентный процесс согласования не только самих используемых понятий, но и соответствующих им \textit{внешних знаков} (имен, терминов). Более того, \textit{внешние знаки} (имена) и их согласование могут потребоваться не только  для понятий, но и для сущностей других видов (например, для людей, населенных пунктов, географических объектов, исторических событий и т.д.).

Подчеркнем при этом, что принципы организации согласования \textit{внешних знаков} (имен) аналогичны рассмотренным выше принципам организации согласования понятий в условиях их постоянного изменения. Так, например, каждой связке отношения \textit{\textbf{быть внешним знаком*}}, связывающей sc-знак некоторой сущности с \textit{sc-узлом}, обозначающим файл внешнего знака указанной сущности, как и каждому понятию, можно поставить в соответствие ее текущий статус (согласованный, устаревший, устаревающий, возвращаемый, предложенный, одобренный, отклоненный, вводимый). 

Завершая рассмотрение модели понимания как модели семантического ввода некоторого текста, не обязательно принадлежащего \textit{SC-коду}, в заданный текст \textit{SC-кода}, сделаем несколько замечаний. 

Понимание может быть искаженным (в том числе противоречивым) и поверхностным (неполным), обусловленным некачественным погружением новой информации в текущее состояние информационного ресурса, хранимого в памяти компьютерной системы (ошибка в отождествлении знаков и, как следствие, неверно установленная синонимия, либо неполнота отождествления, не все новые знаки, синонимичные имеющимся в базе знаний, склеены с со своими синонимами). 

\textbf{Проблема понимания}, взаимопонимания между людьми, между компьютерными системами, между компьютерными системами и их пользователями является \underline{эпицентром} современного этапа эволюции компьютерных систем и ждет своего решения. Чем глубже мы проникаем в формализацию процесса понимания (особенно, понимания текстов естественного языка), тем все больше и больше приходится удивляться тому, что люди все же как-то понимают друг друга, хотя далеко не всегда. Чаще это не понимание, а иллюзия понимания. Здесь уместно напомнить известную фразу: ``Счастье -- это когда тебя понимают.''\\}

\scnendstruct

\end{SCn}