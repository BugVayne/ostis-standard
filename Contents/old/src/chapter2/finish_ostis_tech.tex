\begin{SCn}

\scnsectionheader{\nameref{chap_ostis}}

\scnstartsubstruct

\scnheader{Технология OSTIS}
\scnspheresapplication{\begin{scnitemize}
    \item Разработка на базе \textit{Технологии OSTIS} частной технологии проектирования интеллектуальных справочных систем, интеллектуальных семантических учебников, обучающих систем и интеллектуальных help-систем в различных областях;
    \item Целостный комплекс совместимых семантических электронных учебников по всему набору школьных дисциплин;
    \item Интеллектуальные персональные ассистенты (секретари, референты), осуществляющие персонифицированное информационное обслуживание, интеграцию доступных сервисов, мониторинг и контроль деятельности пользователей;
    \item Интеллектуальные системы управления различными предприятиями, организациями, проектами на основе онтологий и формального описания выполняемых действий, событий, ситуаций;
    \item Интеллектуальные системы автоматизации проектирования различных классов искусственных систем на основе онтологических моделей;
    \item Порталы научных знаний и семантические средства поддержки развития различных научно-технических направлений;
    \item Распределенное глобальное смысловое пространство знаний, представляющее собой результат интеграции баз знаний всех систем, построенных по \textit{Технологии OSTIS} и связанных между собой глобальной сетью;
    \item Интеллектуальные системы экскурсионного обслуживания;
    \item Интеллектуальные системы комплексного индивидуального медицинского мониторинга и обслуживания;
    \item Интеллектуальные робототехнические системы;
    \item Умная среда жизнедеятельности (умный дом, умная дорога, умный город).
\end{scnitemize}
}

\scnheader{Предметная область и онтология Технологии OSTIS}
\scntext{заключение}{Основными направлениями решения проблемы \textit{\textbf{семантической совместимости}} компьютерных систем являются:

\begin{scnitemize}
    \item \textit{семантическая информационная технология}, в основе которой лежит смысловое представление информации в памяти компьютерных систем и иерархическая система формальных онтологий;
    \item \textit{cамоорганизующаяся экосистема}, поддерживающая эволюцию и совместимость компьютерных систем, построенных по семантической информационной технологии, в ходе эксплуатации этих систем.
\end{scnitemize}

Текущий этап развития традиционных и интеллектуальных информационных технологий знаменует переход от современных информационных технологий к \textit{\textbf{семантическим информационным технологиям}} и к соответствующей самоорганизующейся экосистеме, состоящей из \textit{\textbf{семантических компьютерных систем}}. Эпицентром текущего этапа развития информационных технологий является обеспечение и самообеспечение семантической совместимости компьютерных систем и согласованности их функционирования.

Очевидно, что темпы развития семантических информационных технологий, а также рынка прикладных семантических компьютерных систем в первую очередь зависит от числа специалистов, принимающих участие в развитии этих технологий и в расширении многообразия их приложений. Наиболее результативной формой достижения этих целей являются \textbf{открытые проекты} и, прежде всего, открытый проект развития \textit{Метасистемы IMS.ostis}, предоставляющий возможность каждому желающему внести свой вклад в развитие открытой технологии проектирования семантически совместимых компьютерных систем.}

\scnendstruct

\end{SCn}
