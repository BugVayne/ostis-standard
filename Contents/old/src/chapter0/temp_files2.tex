\begin{SCn}

\scnheader{файл ostis-системы}
\scnnote{Традиционно в области информационных технологий существует разделение файлов на текстовые файлы, состоящие из печатных символов и пригодные для прочтения как человеком, так и машиной, и бинарные файлы, пригодные только для машинной обработки. Для просмотра или редактирования бинарных файлов при необходимости используются специальные средства.

Однако, данное разделение основано на представлении файлов в традиционной линейной памяти, в связи с чем не представляет интереса при классификации файлов ostis-систем. Тем не менее, в качестве файлов ostis-системы могут выступать файлы, представленные в каком-либо из широко используемых на сегодняшний день форматов, например pdf-файлы или html-файлы.
}
\scnsuperset{естественно-языковой файл ostis-системы}
\scnsuperset{графический файл ostis-системы}
\scnsuperset{текстово-графический файл ostis-системы}
\scnsuperset{видеофайл ostis-системы}
\scnsuperset{аудиофайл ostis-системы}

\scnheader{естественно-языковой файл ostis-системы}
\scnidtf{текстовый файл ostis-системы}
\scnexplanation{Файлы данного типа включают только текст, состоящий из символов некоторого алфавита, при необходимости закодированных в какой-либо кодировке. Файлы данного типа не могут включать графические иллюстрации, однако могут содержать ссылки на файлы других типов в виде идентификаторов указанных файлов или путей к данным файлам на файловой системе, если речь идет про естественно-языковой файл, являющийся исходным текстом базы знаний. При необходимости текст в файлах данного типа может быть структурирован и форматирован с использованием отступов, различного рода выделений, гарнитуры и размера шрифта и т.д.}
\scnsuperset{sc.n-файл}
\scnsuperset{линейный текстовый файл ostis-системы}
\scnaddlevel{1}
    \scnexplanation{Файлы данного типа содержать текст, размещение (отступы, число символов в строке, выравнивание и положение при просмотре) которого не фиксируется (не имеет значения), но который может быть форматирован с точки зрения шрифта, т.е. допускается выделение курсивом, полужирным шрифтом, увеличение размера шрифта и т.д.}
        \scnsuperset{sc.s-файл}
        \scnsuperset{простой естественно-языковой файл}
        \scnaddlevel{1}
            \scnexplanation{Файлы указанного типа используют только \textit{Алфавит символов, входящих в состав строковых идентификаторов} и могут входить непосредственно в состав исходного текста базы знаний. Для редактирования такого рода файлов достаточно обычного текстового редактора. В такого рода файлов сильно ограничены возможности форматирования и разметки текста, отсутствует возможность изменять гарнитуру, шрифт, размер и другие параметры текста. При визуализации такие файлы отображаются "как есть"\ с использованием формата текста по умолчанию. Чаще всего файлы данного типа представляют собой строковые идентификаторы sc-элементов и небольшие пояснения или комментарии к каким-либо sc-элементам.}
            \scnsuperset{sc.s-файл исходного текста базы знаний}
            \scnsuperset{размеченный естественно-языковой файл}
            \scnaddlevel{1}
                \scnexplanation{Файлы данного типа размечаются при помощи какого-либо стандартного языка разметки. Возможности структурирования и форматирования текста, содержащегося в таких файлах сильно ограничены, однако такая разметка далее может быть учтена при визуализации файла, зачет чего достигается возможность форматирования и структурирования текста. Для редактирования такого рода файлов достаточно обычного текстового редактора, однако требуется владение используемым языком разметки.}
                \scnsuperset{xml-файл}
                \scnsuperset{html-файл}
                \scnaddlevel{1}
                    \scnnote{При визуализации файлы данного типа поддерживают возможность форматирования текста, изменение различных параметров шрифта, использование маркированных списков и каскадных таблиц стилей (CSS), благодаря чему в настоящее время данный тип естественно-языковых файлов является основным при разработке баз знаний ostis-систем. Кроме того, html-файлы легко визуализируются обычным интернет-браузером.}
                \scnaddlevel{-1}
            \scnaddlevel{-1}
        \scnaddlevel{-1}
    \scnaddlevel{-1}

\scnheader{графический файл ostis-системы}
\scnidtf{файл-изображение ostis-системы}
\scnexplanation{графические файл ostis-системы в общем случае могут содержать не только любую информацию, представленную в графической форме, но и фрагменты текста, например, имена каких-либо изображенных сущностей}
\scnsubdividing{векторный графический файл ostis-системы\\
\scnaddlevel{1}
    \scnsuperset{sc.g-файл}
\scnaddlevel{-1}
;растровый графический файл ostis-системы}
\scnsubdividing{динамический графический файл ostis-системы\\
\scnaddlevel{1}
    \scnsuperset{динамический sc.g-файл}
\scnaddlevel{-1}
;статический графический файл ostis-системы}

\scnheader{текстово-графический файл ostis-системы}
\scnexplanation{Файлы данного типа в общем случае могут содержать произвольную информацию, записанную в комбинированной форме, включающей фрагменты текста, рисунки, формулы и т.д. Для визуализации и редактирования файлов данного типа требуются специализированные средства (как правило, для каждого формата файла требуются собственные средства визуализации и редактирования).}
\scnsuperset{pdf-файл}
\scnsuperset{doc-файл}
\scnsuperset{docx-файл}
\scnsuperset{публикация раздела базы знаний ostis-системы}
\scnaddlevel{1}
    \scnhaselement{Публикация Документации Технологии OSTIS-2020}
\scnaddlevel{-1}

\scnauthorcomment{Куда отнести sc.n-текст, считаем ли мы, что в нем могут быть изображения или это достигается просто за счет средств просмотра содержимого файлов, а сам sc.n-текст содержит просто их имена.}

\scnauthorcomment{Получается что, например html-файл на уровне исходника -- размеченный естественно-языковой файл, в котором нет возможности его форматировать и структурировать, а при отображении в браузере, например, в составе sc.n-текста, такой файл становится полноценным ея-файлом с форматированием. В современных средствах файл -- это то, что лежит на файловой системе, а то, что в браузере отображается это как бы уже не сам файл, это его отображение, так что тут все более однозначно. Как быть у нас.}

\scnauthorcomment{Вводить ли понятие текстово-графического файла и где грань между таким файлом и графическим файлом, ведь графический файл тоже может содержать фрагменты текста.}

\scnauthorcomment{Можно ли считать sc.g-файл векторным графическим файлом и как отделить его отображение от способа хранения. Сейчас такие файлы хранятся в виде XML например, то есть это вообще говоря текстовые файлы. В будущем вообще есть идея хранить их как SCs с дополнительным указанием взаимного расположения элементов.}

\end{SCn}