\begin{SCn}

\scnsectionheader{Предметная область и онтология решателей задач компьютерных систем}

\scnstartsubstruct

\scnheader{Предметная область решателей задач современных интеллектуальных компьютерных систем}
\scnsdmainclasssingle{решатель задач компьютерных систем}
\scnsdclass{решатель задач компьютерных систем;гибридный решатель задач;объединенный решатель задач;многоагентная система}

\scnheader{решатель задач компьютерных систем}
\scnexplanation{Одним из ключевых компонентов интеллектуальной системы, обеспечивающим возможность решать широкий круг задач, является решатель задач. Особенностью решателей задач интеллектуальных систем по сравнению с другими современными программными системами является необходимость решать задачи в условиях, когда сведения, необходимые для решения задачи, не локализованы явно в базе знаний интеллектуальной системы и должны быть найдены в процессе решения задачи на основании каких-либо критериев}

\scnsuperset{объединенный решатель задач}
\scnaddlevel{1}
\scnrelfromlist{требования}{
	\scnfileitem{обеспечение основной функциональности системы (решение явно сформулированных задач по требованию)};
	\scnfileitem{обеспечение корректности и оптимизация работы системы (перманентно на протяжении жизненного цикла системы)};
	\scnfileitem{обеспечение автоматизации развития интеллектуальной системы}
}
\scnaddlevel{-1}
\scnsuperset{гибридный решатель задач}

\scntext{проблемы разработки}{Несмотря на то что в настоящее время существует большое число моделей решения задач, многие из которых реализованы и успешно используются на практике в различных системах, остается актуальной проблема низкой согласованности принципов, лежащих в основе реализации таких моделей, и отсутствия единой унифицированной основы для реализации и интеграции различных моделей, что приводит к тому, что:
\begin{scnitemize}
\item затруднена возможность одновременного использования различных моделей решения задач в рамках одной системы при решении одной и той же комплексной задачи; практически невозможно комбинировать различные модели с целью решения задачи, для которой априори отсутствует алгоритм ее
решения;
\item практически невозможно использовать технические решения, реализованные в одной системе, в других системах, т. е. возможности использования компонентного подхода при построении решателей задач сильно ограничены. Как следствие, велико количество дублирований аналогичных решений в разных системах;
\item фактически отсутствуют комплексные методики и средства построения решателей задач, которые бы обеспечивали возможность проектирования, реализации и отладки решателей различного вида.
\end{scnitemize}

Следствиями указанных проблем являются:
\begin{scnitemize}
\item высокая трудоемкость разработки каждого решателя, увеличение сроков их разработки, а значит, и увеличение затрат на разработку и поддержку соответствующих интеллектуальных систем;
\item высокая трудоемкость внесения изменений в уже разработанные решатели, т. е. отсутствует или сильно затруднена возможность дополнения уже разработанного решателя новыми компонентами и внесения изменений в уже существующие компоненты в процессе эксплуатации системы. Таким образом, высока трудоемкость поддержки разработанных решателей;
\item высокий уровень профессиональных требований к разработчикам решателей задач, что обусловлено, в частности:
\begin{scnitemizeii}
\item высокой сложностью существующих формализмов в области решения задач, рассчитанных на их интерпретацию компьютерной системой, а не человеком;
\item отсутствием возможности рассматривать разрабатываемые решатели на разных уровнях детализации, выделения на каждом уровне достаточно независимых компонентов, что затрудняет процесс проектирования, тестирования и отладки таких решателей, а также снижает эффективность попыток объединения разработчиков решателей в коллективы по причине увеличения накладных расходов на согласование их деятельности;
\item низким уровнем информационной поддержки разработчиков и автоматизации их деятельности.
\end{scnitemizeii}
\end{scnitemize}
}


\scnheader{гибридный решатель задач}
\scnexplanation{Расширение областей применения интеллектуальных систем требует от таких систем возможности решения комплексных задач, решение каждой из которых предполагает совместное использование целого ряда различных моделей представления знаний и различных моделей решения задач. Кроме того, решение комплексных задач предполагает использование общих информационных ресурсов (в предельном случае -- всей базы знаний интеллектуальной системы) различными компонентами решателя, ориентированными на решение различных подзадач. Поскольку решатель комплексных задач осуществляет интеграцию различных моделей решения задач, будем называть его \textbf{\textit{гибридным решателем задач}}.}
\scnrelfromset{требования}{
	\scnfileitem{обеспечение решения задач из оговоренного класса за оговоренное время, при этом результат решения задачи должен удовлетворять некоторым известным требованиям. Более детально рассмотрим некоторые положения, уточняющие сформулированное требование:
	\begin{scnitemize}
		\item для явно сформулированных задач система всегда должна давать какой-либо ответ за оговоренное время, при этом ответ может быть отрицательным (система не смогла решить поставленную задачу), возможно, с объяснением причин, по которым решение в текущий момент оказалось невозможным. Одним из факторов безуспешности решения является выход за рамки установленного промежутка времени;
		\item если явно сформулированная задача решена, то все информационные процессы, направленные на ее решение, должны быть уничтожены. Особенно актуальным данное требование становится в ситуации, когда для решения одной и той же задачи параллельно используются сразу несколько подходов и заранее неизвестно, какой из них приведет к результату раньше других;
		\item после решения задачи вся временная информация, сгенерированная в процессе решения этой задачи и имеющая ценность только в контексте решения указанной задачи, должна быть удалена из памяти.
	\end{scnitemize}};
	\scnfileitem{обеспечение возможности согласованного использования различных моделей решения задач при решении одной и той же комплексной задачи в случае необходимости};
	\scnfileitem{решатель должен быть легко модифицируемым, т. е. трудоемкость внесения изменений в уже разработанный решатель должна быть минимальна. Путями повышения модифицируемости решателя являются обеспечение локальности вносимых изменений, в том числе -- за счет стратификации решателя на независимые уровни и обеспечение максимальной независимости компонентов решателя друг от друга, а также наличие готовых компонентов, которые могут быть встроены в решатель при необходимости. При этом внесение изменений должно осуществляться непосредственно в процессе эксплуатации системы};
	\scnfileitem{для того чтобы интеллектуальная система имела возможность анализировать и оптимизировать имеющийся решатель задач, интегрировать в его состав новые компоненты (в том числе самостоятельно), оценивать важность тех или иных компонентов и применимость их для решения той или	иной задачи, спецификация решателя должна быть описана языком, понятным системе, например, при помощи тех же средств, что и обрабатываемые знания. Возможность интеллектуальной системы анализировать (верифицировать, корректировать, оптимизировать) собственные компоненты будем называть рефлексивностью}
}

\scnheader{многоагентная система}
\scntext{достоинства}{
\begin{scnitemize}
\item автономность (независимость) агентов в рамках такой системы, что позволяет локализовать изменения, вносимые в решатель при его эволюции, и снизить соответствующие трудозатраты;
\item децентрализация обработки, т.е. отсутствие единого контролирующего центра, что также позволяет локализовать вносимые в решатель изменения;
\item возможность параллельной работы разных информационных процессов, соответствующих как одному агенту, так и разным агентам, как следствие, -- возможность распределенного решения задач;
\item активность агентов и многоагентной системы в целом, дающая возможность при общении с такой системой не указывать явно способ решения поставленной задачи, а формулировать задачу в декларативном ключе.
\end{scnitemize}
}
\scntext{структура}{В общем случае для построения некоторой конкретной многоагентной системы необходимо уточнить следующие ее компоненты:
\begin{scnitemize}
\item модель собственно агента, входящего в состав такой системы, включая классификацию таких агентов и набор понятий, характеризующих каждый агент в рамках системы. В настоящее время наиболее популярной является модель BDI (belief-desire-intention), в рамках которой предполагается описывать на соответствующих языках "убеждения"{}, "желания"{} и "намерения"{} каждого агента системы;
\item модель среды, в рамках которой находятся агенты, на события в которой они реагируют и в рамках которой могут осуществлять некоторые преобразования. Обзор разновидностей сред для многоагентных систем приводится в работе \scncite{Weyns2007};
\item модель коммуникации агентов, в рамках которой уточняется язык взаимодействия агентов (структура и классификация сообщений) и способ передачи сообщений между агентами. В настоящее время существует ряд стандартов, описывающих языки взаимодействия агентов, например, KQML \scncite{Finin1994} и ACL \scncite{ACL};
\item модель координации агентов, регламентирующая принципы их деятельности, в том числе, механизмы решения возможных конфликтов. В настоящее время основное число работ в области многоагентных систем направлено именно на разработку механизмов координации агентов, в числе которых выделение агентов более высокого уровня (метаагентов) \scncite{Hartung2008}, различные социально-психологические модели \scncite{Vasconcelos2009,Rumbell2012}, поведение на основе онтологий \scncite{Gorodetsky2015} и другие.
\end{scnitemize}}
\scntext{проблемы разработки}{
\begin{scnitemize}
\item жесткая ориентация большинства средств построения многоагентных систем на модель BDI (Belief-Desire-Intention) приводит к существенным накладным расходам, связанным с необходимостью выражения конкретной практической задачи в системе понятий BDI. В то же время, ориентация на модель BDI неявно провоцирует искусственное разделение языков, описывающих собственно компоненты BDI и знания агента о внешней среде, что приводит к отсутствию унификации представления и, соответственно, дополнительным накладным расходам;
\item большинство современных средств построения многоагентных систем ориентированы на представление знаний агента при помощи узкоспециализированных языков, зачастую не предназначенных для представления знаний в широком смысле. Речь при этом идет как о знаниях агента о себе самом (например, в соответствии с моделью BDI), так и знаниях о внешней среде. В некоторых подходах вначале строится онтология, для создания которой, однако, часто используются средства с низкой выразительной способностью, не предназначенные для построения онтологий \scncite{Evertsz2004,JADE2017}. В конечном итоге такой подход приводит к сильной ограниченности возможностей разработанных многоагентных систем и их несовместимости;
\item абсолютное большинство современных средств построения многоагентных систем предполагает, что взаимодействие агентов осуществляется путем обмена сообщениями непосредственно от агента к агенту. Такой подход обладает существенным недостатком, связанным с тем, что в этом случае каждый агент системы должен иметь достаточно полную информацию о других агентах в системе, что приводит к дополнительным затратам ресурсов, кроме того добавление или удаление одного или нескольких агентов приводит к необходимости оповещения об этом других агентов. Данная проблема решается путем организации общения агентов по принципу <<доски объявлений>> \scncite{Jagannathan1989}, предполагающему, что сообщения помещаются в некоторую общую для всех агентов область, при этом каждый агент в общем случае может не знать, какому из агентов адресовано сообщение и от какого из агентов получено то или иное сообщение. Однако, данный подход не исключает проблему, связанную с необходимостью разработки специализированного языка взаимодействия агентов, который в общем случае не связан с языком, на котором описываются знания агента о решаемых задачах и окружающей среде;
\item многие средства построения многоагентных систем построены таким образом, что логический уровень взаимодействия агентов жестко привязан к физическому уровню реализации многоагентной системы. Например, при передаче сообщений от агента к агенту разработчику многоагентной системы необходимо помимо семантически значимой информации указывать ip-адрес компьютера, на котором расположен агент-получатель, кодировку, с помощью которой закодирован текст сообщения и другую техническую информацию, обусловленную исключительно особенностями текущей реализации средств;
\item в большинстве подходов среда, с которой взаимодействуют агенты, уточняется отдельно разработчиком для каждой многоагентной системы, что с одной стороны, расширяет возможности применения соответствующих средств, но с другой стороны приводит к существенным накладным расходам и несовместимости таких многоагентных систем. Кроме того, в ряде случаев разработчик также обязан учитывать особенности технической реализации средств разработки в плане их стыковки с предполагаемой средой, в роли которой может выступать, например, локальная или глобальная сеть.
\end{scnitemize}}

\scnendstruct

\end{SCn}