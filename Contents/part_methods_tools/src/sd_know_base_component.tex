\begin{SCn}
    \scnsectionheader{Предметная область и онтология многократно используемых компонентов баз знаний ostis-систем}
    \begin{scnsubstruct}
        \scnheader{Предметная область многократно используемых компонентов баз знаний ostis-систем}
        \scniselement{предметная область}
        \scnrelto{частная предметная область}{Предметная область многократно используемых компонентов ostis-систем}
        \begin{scnhaselementrolelist}{класс объектов исследования}
            \scnitem{многократно используемый компонент баз знаний ostis-систем}
        \end{scnhaselementrolelist}
        \scnhaselementrole{класс объектов исследования}{отношение, специфицирующее многократно используемый компонент баз знаний ostis-систем}
        
        \scnheader{многократно используемый компонент баз знаний}
        \scnsuperset{предметная область и онтология}
        \scnsubset{раздел базы знаний}
        \scnsuperset{семантическая окрестность}
        \scnrelfrom{смотрите}{\nameref{sd_sem_neigh}}
        \scnsuperset{базовые фрагменты предметных областей и онтологий}
        \scntext{примечание}{Базовый фрагмент предметной области и онтологии включает в себя теоретико-множественную и логическую онтологию, а также фрагменты терминологической онтологии, описывающие основные идентификаторы объектов исследования предметной области.}
        \scntext{примечание}{Данный вид многократно используемых компонентов позволяет использовать только те знания, которые непосредственно необходимы для функционирования интеллектуальных систем, исключив то, что никак не влияет на работу конечной системы (пояснения, примеры, дидактический материал и т.д.).}
        \scnhaselement{Базовый фрагмент теории логических формул, высказываний и логических sc-языков}
        \scnhaselement{Базовый фрагмент теории множеств}
        \scnhaselement{Базовый фрагмент теории связок и отношений}
        \scnsuperset{база знаний прикладной ostis-системы}
        \scntext{примечание}{Целые базы знаний могут быть многократно используемыми компонентами в случае разработки интеллектуальных систем, имеющих схожие функциональные требования.}
        
        \scnheader{многократно используемый компонент базы знаний}
        \scnhaselement{Расширенное ядро базы знаний}
        \scnhaselement{Ядро базы знаний}
        \scntext{пояснение}{\textit{Ядро базы знаний} представляет собой компонент, входящий в состав каждой базы знаний, разрабатываемой по \textit{Технологии OSTIS}, и устанавливаемый в первую очередь.}
        \scntext{примечание}{Список приведенных классов многократно используемых компонентов не является окончательным. В случае, когда разработчик базы знаний интеллектуальной системы считает, что разработанный им компонент сможет стать неотъемлемой частью библиотеки, то компонент будет добавлен в библиотеку, как многократно используемый, в случае, если компонент прошел верификацию и соответствует требованиям разработчиков библиотеки.}
        
        \scnheader{отношение, специфицирующее многократно используемый компонент баз знаний ostis-систем}
        \scnsubset{отношение, специфицирующее многократно используемый компонент ostis-систем}
        \scnhaselement{максимальный класс объектов исследования\scnrolesign}
        \scnhaselement{немаксимальный класс объектов исследования\scnrolesign}
        \scnhaselement{исследуемое отношение\scnrolesign}
        \bigskip
    \end{scnsubstruct}
\end{SCn}
