\begin{SCn}
\scnsectionheader{Предметная область и онтология денотационной семантики Базового языка программирования ostis-систем}
\begin{scnsubstruct}
	
\scnheader{Предметная область денотационной семантики языка SCP}
\scniselement{предметная область}
\scnhaselementrole{максимальный класс объектов исследования}{scp-оператор}
\begin{scnhaselementrolelist}{исследуемое отношение}
    \scnitem{scp-операнд\scnrolesign}
    \scnitem{scp-константа\scnrolesign}
    \scnitem{scp-переменная\scnrolesign}
    \scnitem{scp-операнд с заданным значением\scnrolesign}
    \scnitem{scp-операнд со свободным значением\scnrolesign}
    \scnitem{формируемое множество\scnrolesign}
    \scnitem{удаляемый sc-элемент\scnrolesign}
\end{scnhaselementrolelist}

\scnheader{scp-оператор}
\scnrelto{включение}{действие в sc-памяти}
\scnrelto{семейство подмножеств}{атомарный тип scp-оператора}
\begin{scnsubdividing}
	%TODO: check by human--->
	\scnitem{scp-оператор генерации конструкций}
		\begin{scnindent}
			\begin{scnsubdividing}
				%TODO: check by human--->
				\scnitem{scp-оператор генерации конструкции по произвольному образцу}
				\scnitem{scp-оператор генерации пятиэлементной конструкции}
				\scnitem{scp-оператор генерации трехэлементной конструкции}
				\scnitem{scp-оператор генерации одноэлементной конструкции}
				%<---TODO: check by human
			\end{scnsubdividing}
		\end{scnindent}
	\scnitem{scp-оператор ассоциативного поиска конструкций}
		\begin{scnindent}
			\begin{scnsubdividing}
				%TODO: check by human--->
				\scnitem{scp-оператор поиска конструкции по произвольному образцу}
				\scnitem{scp-оператор поиска пятиэлементной конструкции с формированием множеств}
				\scnitem{scp-оператор поиска трехэлементной конструкции с формированием множеств}
				\scnitem{scp-оператор поиска пятиэлементной конструкции}
				\scnitem{scp-оператор поиска трехэлементной конструкции}
				%<---TODO: check by human
			\end{scnsubdividing}
		\end{scnindent}
	\scnitem{scp-оператор удаления конструкций}
		\begin{scnindent}
			\begin{scnsubdividing}
				%TODO: check by human--->
				\scnitem{scp-оператор удаления множества элементов трехэлементной конструкции}
				\scnitem{scp-оператор удаления одноэлементной конструкции}
				\scnitem{scp-оператор удаления пятиэлементной конструкции}
				\scnitem{scp-оператор удаления трехэлементной конструкции}
				%<---TODO: check by human
			\end{scnsubdividing}
		\end{scnindent}
	\scnitem{scp-оператор проверки условий}
		\begin{scnindent}
			\begin{scnsubdividing}
				%TODO: check by human--->
				\scnitem{scp-оператор сравнения числовых содержимых файлов}
				\scnitem{scp-оператор проверки равенства числовых содержимых файлов}
				\scnitem{scp-оператор проверки совпадения значений операндов}
				\scnitem{scp-оператор проверки наличия содержимого у файла}
				\scnitem{scp-оператор проверки наличия значения у переменной}
				\scnitem{scp-оператор проверки типа sc-элемента}
				%<---TODO: check by human
			\end{scnsubdividing}
		\end{scnindent}
	\scnitem{scp-оператор управления значениями операндов}
		\begin{scnindent}	
			\begin{scnsubdividing}
				%TODO: check by human--->
				\scnitem{scp-оператор удаления значения переменной}
				\scnitem{scp-оператор присваивания значения переменной}
				%<---TODO: check by human
			\end{scnsubdividing}
		\end{scnindent}
	\scnitem{scp-оператор управления scp-процессами}
		\begin{scnindent}
			\begin{scnsubdividing}
				%TODO: check by human--->
				\scnitem{scp-оператор удаления значения переменной}
				\scnitem{scp-оператор завершения выполнения программы}
				\scnitem{конъюнкция предшествующих scp-операторов}
				\scnitem{scp-оператор ожидания завершения выполнения множества scp-программ}
				\scnitem{scp-оператор ожидания завершения выполнения scp-программы}
				\scnitem{scp-оператор асинхронного вызова подпрограммы}
				%<---TODO: check by human
			\end{scnsubdividing}
		\end{scnindent}
	\scnitem{scp-оператор управления событиями}
		\begin{scnindent}
		\begin{scnreltoset}{разбиение}
			%TODO: check by human--->
			\scnitem{scp-оператор ожидания события}
			%<---TODO: check by human
		\end{scnreltoset}
		\end{scnindent}
	\scnitem{scp-оператор обработки содержимых файлов}
		\begin{scnindent}
			\begin{scnsubdividing}
                %TODO: check by human--->
                \scnitem{scp-оператор вычисления арксинуса числового содержимого файла}
                \scnitem{scp-оператор вычисления арккосинуса числового содержимого файла}
                \scnitem{scp-оператор деления числовых содержимых файлов}
                \scnitem{scp-оператор умножения числовых содержимых файлов}
                \scnitem{scp-оператор вычитания числовых содержимых файлов}
                \scnitem{scp-оператор сложения числовых содержимых файлов}
                \scnitem{scp-оператор вычисления тангенса числового содержимого файла}
                \scnitem{scp-оператор вычисления косинуса числового содержимого файла}
                \scnitem{scp-оператор вычисления синуса числового содержимого файла}
                \scnitem{scp-оператор вычисления логарифма числового содержимого файла}
                \scnitem{scp-оператор возведения числового содержимого файла в степень}
                \scnitem{scp-оператор удаления содержимого файла}
                \scnitem{scp-оператор копирования содержимого файла}
                \scnitem{scp-оператор нахождения остатка от деления числовых содержимых файлов}
                \scnitem{scp-оператор нахождения целой части от деления числовых содержимых файлов}
                \scnitem{scp-оператор вычисления арктангенса числового содержимого файла}
                \scnitem{scp-оператор перевода в верхний регистр строкового содержимого файла}
                \scnitem{scp-оператор перевода в верхний регистр строкового содержимого файла}
                \scnitem{scp-оператор замены определенной части строкового содержимого файла на содержимое указанной файла}
                \scnitem{scp-оператор проверки совпадения конца строкового содержимого файла со строковом содержимым другого файла}
                \scnitem{scp-оператор проверки совпадения начальной части строкового содержимого файла со строковом содержимым другого файла}
                \scnitem{scp-оператор получения части строкового содержимого файла по индексам}
                \scnitem{scp-оператор поиска строкового содержимого файла в строковом содержимом другого файла}
                \scnitem{scp-оператор вычисления длины строкового содержимого файла}
                \scnitem{scp-оператор разбиения строки на подстроки}
                \scnitem{scp-оператор лексикографического сравнения строковых содержимых файлов}
                \scnitem{scp-оператор проверки равенства строковых содержимых файлов}
                %<---TODO: check by human
			\end{scnsubdividing}
		\end{scnindent}
	\scnitem{scp-оператор управления блокировками}
		\begin{scnindent}
			\begin{scnsubdividing}
                %TODO: check by human--->
                \scnitem{scp-оператор снятия всех блокировок данного scp-процесса}
                \scnitem{scp-оператор снятия блокировки с sc-элемента}
                \scnitem{scp-оператор установки полной блокировки на sc-элемент}
                \scnitem{scp-оператор установки блокировки на изменение sc-элемента}
                \scnitem{scp-оператор установки блокировки на удаление sc-элемента}
                \scnitem{scp-оператор снятия блокировки со структуры}
                \scnitem{scp-оператор установки полной блокировки на структуру}
                \scnitem{scp-оператор установки блокировки на изменение структуры}
                \scnitem{scp-оператор установки блокировки на удаление структуры}
                %<---TODO: check by human
			\end{scnsubdividing}
		\end{scnindent}
	%<---TODO: check by human
\end{scnsubdividing}

\scnheader{scp-операнд\scnrolesign}
\scnrelto{включение}{аргумент действия\scnrolesign}
\scniselement{неосновное понятие}
\scniselement{ролевое отношение}
\begin{scnsubdividing}
	%TODO: check by human--->
	\scnitem{scp-константа\scnrolesign}
	\scnitem{scp-переменная\scnrolesign}
	%<---TODO: check by human
\end{scnsubdividing}
\begin{scnsubdividing}
	%TODO: check by human--->
	\scnitem{scp-операнд с заданным значением\scnrolesign}
	\scnitem{scp-операнд со свободным значением\scnrolesign}
	%<---TODO: check by human
\end{scnsubdividing}
\begin{scnsubdividing}
	%TODO: check by human--->
	\scnitem{константный sc-элемент\scnrolesign}
	\scnitem{переменный sc-элемент\scnrolesign}
	%<---TODO: check by human
\end{scnsubdividing}
\begin{scnrelfromlist}{включение}
%TODO: check by human--->
	\scnitem{формируемое множество\scnrolesign}
		\begin{scnindent}
			\begin{scnsubdividing}
			%TODO: check by human--->
			\scnitem{формируемое множество 1\scnrolesign}
			\scnitem{формируемое множество 2\scnrolesign}
			\scnitem{формируемое множество 3\scnrolesign}
			\scnitem{формируемое множество 4\scnrolesign}
			\scnitem{формируемое множество 5\scnrolesign}
			%<---TODO: check by human
			\end{scnsubdividing}
		\end{scnindent}
	\scnitem{удаляемый sc-элемент\scnrolesign}
	\scnitem{тип sc-элемента\scnrolesign}
		\begin{scnindent}
			\begin{scnsubdividing}
				%TODO: check by human--->
				\scnitem{sc-узел\scnrolesign}
					\begin{scnindent}
						\begin{scnsubdividing}
							%TODO: check by human--->
							\scnitem{структура\scnrolesign}
							\scnitem{отношение\scnrolesign}
								\begin{scnindent}
									\scnrelfrom{включение}{ролевое отношение\scnrolesign}
								\end{scnindent}
							\scnitem{класс\scnrolesign}
							%<---TODO: check by human
						\end{scnsubdividing}
					\end{scnindent}
				\scnitem{sc-дуга\scnrolesign}
					\begin{scnindent}
						\begin{scnsubdividing}
							%TODO: check by human--->
							\scnitem{sc-дуга общего вида\scnrolesign}
							\scnitem{sc-дуга принадлежности\scnrolesign}
								\begin{scnindent}
									\scnrelfrom{включение}{sc-дуга основного вида\scnrolesign}
									\begin{scnsubdividing}
										%TODO: check by human--->
										\scnitem{позитивная sc-дуга принадлежности\scnrolesign}
										\scnitem{негативная sc-дуга принадлежности\scnrolesign}
										\scnitem{нечеткая sc-дуга принадлежности\scnrolesign}
										%<---TODO: check by human
									\end{scnsubdividing}
									\begin{scnsubdividing}
										%TODO: check by human--->
										\scnitem{временная sc-дуга принадлежности\scnrolesign}
										\scnitem{постоянная sc-дуга принадлежности\scnrolesign}
										%<---TODO: check by human
									\end{scnsubdividing}
								\end{scnindent}
							%<---TODO: check by human
						\end{scnsubdividing}
					\end{scnindent}
				\scnitem{sc-ребро\scnrolesign}
				\scnitem{файл\scnrolesign}
				%<---TODO: check by human
			\end{scnsubdividing}
		\end{scnindent}
	%<---TODO: check by human
\end{scnrelfromlist}
\scntext{пояснение}{Ролевое отношение \textit{scp-операнд\scnrolesign} является неосновным понятием и указывает на принадлежность аргументов \textit{scp-оператору}. Помимо указания какого-либо класса \textit{scp-операндов\scnrolesign} порядок аргументов \textit{scp-оператора} дополнительно уточняется \textit{ролевыми отношениями 1\scnrolesign}, \textit{2\scnrolesign} и т. д.}

\scnheader{scp-константа\scnrolesign}
\scntext{пояснение}{В рамках \textit{scp-программы} \textit{scp-константы\scnrolesign} явно участвуют в \textit{\mbox{scp-операторах}} в качестве элементов (в теоретико-множественном смысле) и напрямую обрабатываются при интерпретации \textit{scp-программы}. Константами в рамках \textit{scp-программы} могут быть \textit{sc-элементы} любого типа, как \textit{\mbox{sc-константы}}, так и \textit{\mbox{sc-переменные}}. Константа в рамках \textit{scp-программы} остается неизменной в течение всего срока интерпретации. Константа \textit{\mbox{scp-программы}} может быть рассмотрена как переменная, значение которой совпадает с самой переменной в каждый момент времени и изменено быть не может. Таким образом, далее будем считать, что \textit{scp-константа\scnrolesign} и ее значение это одно и то же. Каждый \textit{in-параметр\scnrolesign} при интерпретации каждой конкретной копии \textit{scp-программы} становится \textit{scp-константой\scnrolesign} в рамках всех ее операторов, хотя в исходном теле данной программы в каждом из этих операторов он является \textit{scp-переменной\scnrolesign}.}

\scnheader{scp-переменная\scnrolesign}
\scntext{пояснение}{В рамках \textit{scp-программы} \textit{scp-переменные\scnrolesign} не обрабатываются явно при интерпретации, обрабатываются значения переменных. Каждая переменная \textit{scp-программы} может иметь одно значение в каждый момент времени, т. е. представляет собой ситуативный \textit{синглетон}, элементом которого является текущее значение \textit{scp-переменной\scnrolesign}. Значение каждой \textit{scp-переменной\scnrolesign} может меняться в ходе интерпретации \textit{scp-программы}. При этом интерпретатор при обработке \textit{scp-оператора} работает непосредственно со значениями \textit{\mbox{scp-переменных\scnrolesign}}, а не самими \textit{scp-переменными\scnrolesign} (которые также являются узлами той же семантической сети).}

\scnheader{scp-операнд с заданным значением\scnrolesign}
\scntext{пояснение}{Значение операндов, помеченных ролевым отношением \textit{scp-операнд с заданным значением\scnrolesign}, считается заданным в рамках текущего \textit{scp-оператора}. Данное значение учитывается при выполнении \textit{scp-оператора} и остается неизменным после окончания выполнения \textit{scp-оператора}. Каждая \textit{scp-константа\scnrolesign} по умолчанию рассматривается как \textit{scp-операнд с заданным значением\scnrolesign}, в связи с чем явное использование данного ролевого отношения в таком случае является избыточным. В таком случае в качестве значения рассматривается непосредственно сам операнд. В случае, если отношением \textit{\mbox{scp-операнд} с заданным значением\scnrolesign} помечена \textit{scp-переменная\scnrolesign}, то осуществляется попытка поиска значения для данной \textit{scp-переменной\scnrolesign} (ее элемента). Если попытка оказалась безуспешной, то возникает ошибка времени выполнения, которая должна быть обработана соответствующим образом.\\
Любой \textit{scp-операнд с заданным значением\scnrolesign} независимо от конкретного типа \textit{scp-оператора} может быть \textit{scp-переменной\scnrolesign}.}

\scnheader{scp-операнд со свободным значением\scnrolesign}
\scntext{пояснение}{Значение операндов, помеченных ролевым отношением \textit{scp-операнд со свободным значением\scnrolesign}, считается свободным (не заданным заранее) в рамках текущего \textit{scp-оператора}. В начале выполнения \textit{scp-оператора} связь между \textit{scp-переменной\scnrolesign}, помеченной данным ролевым отношением, и ее элементом (значением) всегда удаляется. В результате выполнения данного оператора может быть либо сгенерировано новое значение \textit{scp-переменной\scnrolesign}, либо не сгенерировано, тогда \textit{scp-переменная\scnrolesign} будет считаться не имеющей значения. Ни одна \textit{scp-константа\scnrolesign} не может быть помечена как \textit{scp-операнд со свободным значением\scnrolesign}, поскольку константа не может изменять свое значение в ходе интерпретации \textit{scp-программы}.}
\end{scnsubstruct}
\end{SCn}

%В стандарте нет таблицы такой
%Таблица \ref{table_operands_roles} показывает возможные сочетания различных ролевых отношений, указывающих роль операнда в рамках scp-оператора:

%\begin{table}[H]
%  \caption{Роли операндов в рамках scp-оператора}\label{table_operands_roles}
%\begin{tabularx}{\hsize}{| p{43mm} | X | X |}
%  \hline
%  \textbf{Тип значения}
%  & \multirow{2}{*}{\textbf{\shortstack[l]{scp-операнд с\\ заданным значением\scnrolesign}}} & \multirow{2}{*}{\textbf{\shortstack[l]{scp-операнд со\\ свободным значением\scnrolesign}}} \\
%  \cline{0-0}
%  \textbf{Константность} & & \\
%\hline
%\textbf{scp-константа\scnrolesign} & Разрешено, может быть опущено & Запрещено \\
%\hline
%\textbf{scp-переменная\scnrolesign} & Разрешено, значение останется неизменным & Разрешено, значение переменной будет изменено либо потеряно\\
%\hline
%\end{tabularx}
%\end{table}
%}