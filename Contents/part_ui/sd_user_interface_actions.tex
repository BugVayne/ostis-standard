\begin{SCn}
\scnsectionheader{\currentname}
\begin{scnsubstruct}
\scnrelfrom{соавтор}{Садовский М.Е.}

\scnheader{Предметная область интерфейсных действий пользователей}
\scniselement{предметная область}
\begin{scnhaselementrole}{максимальный класс объектов исследования}
	{интерфейсное действие пользователя}
\end{scnhaselementrole}
\begin{scnhaselementrolelist}{класс объектов исследования}
{
    \scnitem{действие мышью}
    \scnitem{прокрутка мышью}
    \scnitem{прокрутка мышью вверх}
    \scnitem{прокрутка мышью вниз}
    \scnitem{наведение мышью}
    \scnitem{отпускание мышью}
    \scnitem{нажатие мыши}
    \scnitem{одиночное нажатие мыши}
    \scnitem{двойное нажатие мыши}
    \scnitem{жест мышью}
    \scnitem{отведение мышью}
    \scnitem{перетаскивание мышью}
    \scnitem{действие голосом}
    \scnitem{действие клавиатурой}
    \scnitem{нажатие функциональной клавиши}
    \scnitem{нажатие клавиши набора текста}
    \scnitem{действие осязанием}
    \scnitem{действие сенсором}
    \scnitem{нажатие сенсора}
    \scnitem{одиночное нажатие сенсора}
    \scnitem{двойное нажатие сенсора}
    \scnitem{жест по сенсору}
    \scnitem{жест по сенсору одним пальцем}
    \scnitem{жест по сенсору несколькими пальцами}
    \scnitem{отпускание сенсором}
    \scnitem{перетаскивание сенсором}
    \scnitem{действие пером}
    \scnitem{нажатие функциональной клавиши пером}
    \scnitem{рисование пером}
    \scnitem{написание текста пером}
}
\end{scnhaselementrolelist}
\begin{scnhaselementrole}{исследуемое отношение}
	{инициируемое пользовательским интерфейсом действие*}
\end{scnhaselementrole}
\scnrelfrom{частная предметная область}{Предметная область интерфейсных действий пользователей ostis-системы}

\scnheader{интерфейсное действие пользователя}
\scnidtf{user interface action}
\scntext{пояснение}{Действие, выполняемое пользователем над некоторым \textit{компонентом пользовательского интерфейса}. Для связи данного действия с \textit{компонентом пользовательского интерфейса} и необходимым к выполнению \textit{внутренним действием системы} используется отношение \textit{инициируемое пользовательским интерфейсом действие*}.}
\scnsuperset{действие мышью}
\begin{scnindent}
	\scnidtf{mouse-action}
	\scnsuperset{прокрутка мышью}
	\scnidtf{mouse-scroll}
	\scnsuperset{прокрутка мышью вверх}
	\begin{scnindent}
		\scnidtf{mouse-scroll-up}
		\scnsuperset{прокрутка мышью вниз}
		\begin{scnindent}
		\scnidtf{mouse-scroll-down}
		\end{scnindent}
	\end{scnindent}
	\scnsuperset{наведение мышью}
	\begin{scnindent}
		\scnidtf{mouse-hover}
	\end{scnindent}
	\scnsuperset{отпускание мышью}
	\begin{scnindent}
		\scnidtf{mouse-drop}
	\end{scnindent}
	\scnsuperset{нажатие мыши}
	\begin{scnindent}
		\scnidtf{mouse-click}
		\scnsuperset{одиночное нажатие мыши}
		\begin{scnindent}
			\scnidtf{mouse-single-click}
		\end{scnindent}
		\scnsuperset{двойное нажатие мыши}
		\begin{scnindent}
			\scnidtf{mouse-double-click}
		\end{scnindent}
	\end{scnindent}
	\scnsuperset{жест мышью}
	\begin{scnindent}
		\scnidtf{mouse-gesture}
	\end{scnindent}
	\scnsuperset{отведение мышью}
	\begin{scnindent}
		\scnidtf{mouse-unhover}
	\end{scnindent}
	\scnsuperset{перетаскивание мышью}
	\begin{scnindent}
		\scnidtf{mouse-drag}
	\end{scnindent}
\end{scnindent}
\scnsuperset{действие голосом}
\begin{scnindent}
	\scnidtf{speech-action}
\end{scnindent}
\scnsuperset{действие клавиатурой}
\begin{scnindent}
	\scnidtf{keyboard-action}
	\scnsuperset{нажатие функциональной клавиши}
	\begin{scnindent}
		\scnidtf{press-function-key}
	\end{scnindent}
	\scnsuperset{нажатие клавиши набора текста}
	\begin{scnindent}
		\scnidtf{type-text}
	\end{scnindent}
\end{scnindent}
\scnsuperset{действие осязанием}
\begin{scnindent}
	\scnidtf{tangible-action}
\end{scnindent}
\scnsuperset{действие сенсором}
\begin{scnindent}
	\scnidtf{touch-action}
	\scnsuperset{нажатие сенсора}
	\begin{scnindent}
		\scnidtf{touch-click}
		\scnsuperset{одиночное нажатие сенсора}
		\begin{scnindent}
			\scnidtf{touch-single-click}
		\end{scnindent}
		\scnsuperset{двойное нажатие сенсора}
		\begin{scnindent}
			\scnidtf{touch-double-click}
		\end{scnindent}
	\end{scnindent}
	\scnsuperset{жест по сенсору}
	\begin{scnindent}
		\scnidtf{touch-gesture}
		\scnsuperset{жест по сенсору одним пальцем}
		\begin{scnindent}
			\scnidtf{one-fingure-gesture}
		\end{scnindent}
		\scnsuperset{жест по сенсору несколькими пальцами}
		\begin{scnindent}
			\scnidtf{multiple-finger-gesture}
		\end{scnindent}
	\end{scnindent}
	\scnsuperset{отпускание сенсором}
	\begin{scnindent}
		\scnidtf{touch-drop}
	\end{scnindent}
	\scnsuperset{перетаскивание сенсором}
	\begin{scnindent}
		\scnidtf{touch-drag}
	\end{scnindent}
\end{scnindent}
\scnsuperset{действие пером}
\begin{scnindent}
	\scnidtf{pen-base-action}
	\scnsuperset{нажатие функциональной клавиши пером}
	\begin{scnindent}
		\scnidtf{touch-function-key}
	\end{scnindent}
	\scnsuperset{рисование пером}
	\begin{scnindent}
		\scnidtf{draw}
	\end{scnindent}
	\scnsuperset{написание текста пером}
	\begin{scnindent}
		\scnidtf{write-text}
	\end{scnindent}
\end{scnindent}	

\scnheader{прокрутка мышью}
\scntext{пояснение}{\textit{прокрутка мышью} --- интерфейсное действие пользователя, соответствующее прокрутке содержимого некоторого компонента пользовательского интерфейса при помощи мыши.}

\scnheader{наведение мышью}
\scntext{пояснение}{\textit{наведение мышью} --- интерфейсное действие пользователя, соответствующее появлению курсора мыши в рамках компонента пользовательского интерфейса.}

\scnheader{отпускание мышью}
\scntext{пояснение}{\textit{отпускание мышью} --- интерфейсное действие пользователя, соответствующее отпусканию некоторого компонента пользовательского интерфейса в рамках другого компонента пользовательского интерфейса при помощи мыши.}

\scnheader{нажатие мыши}
\scntext{пояснение}{\textit{нажатие мыши} --- интерфейсное действие пользователя, соответствующее выполнению нажатия мыши в рамках некоторого компонента пользовательского интерфейса.}

\scnheader{отведение мышью}
\scntext{пояснение}{\textit{отведение мышью} --- интерфейсное действие пользователя, соответствующее выходу курсора мыши за рамки компонента пользовательского интерфейса.}

\scnheader{перетаскивание мышью}
\scntext{пояснение}{\textit{перетаскивание мышью} --- интерфейсное действие пользователя, соответствующее перетаскиванию компонента пользовательского интерфейса при помощи мыши.}

\scnheader{нажатие сенсора}
\scntext{пояснение}{\textit{нажатие сенсора} --- интерфейсное действие пользователя, соответствующее выполнению нажатия сенсора в рамках некоторого компонента пользовательского интерфейса.}

\scnheader{жест по сенсору}
\scntext{пояснение}{\textit{жест по сенсору} --- интерфейсное действие пользователя, соответствующее выполнению некоторого жеста, выполняемого при помощи движения пальцев на экране сенсора.}

\scnheader{отпускание сенсором}
\scntext{пояснение}{\textit{отпускание сенсором} --- интерфейсное действие пользователя, соответствующее отпусканию некоторого компонента пользовательского интерфейса в рамках другого компонента пользовательского интерфейса при помощи сенсора.}

\scnheader{перетаскивание сенсором}
\scntext{пояснение}{\textit{перетаскивание сенсором} --- интерфейсное действие пользователя, соответствующее перетаскиванию компонента пользовательского интерфейса при помощи сенсора.}

\scnheader{действие пером}
\scntext{пояснение}{\textit{действие пером} --- интерфейсное действие пользователя, осуществляемое при помощи пера на графическом планшете.}

\scnheader{класс интерфейсных действий пользователя}
\scnsubset{класс действий}
\scnrelto{семейство подмножеств}{интерфейсное действие пользователя}
\scntext{пояснение}{\textit{класс интерфейсных действий пользователя} --- множество, элементами которого являются классы \textit{интерфейсных действий пользователя}.}

\scnheader{инициируемое пользовательским интерфейсом действие*}
\scntext{пояснение}{При взаимодействии пользователя с \textit{компонентом пользовательского интерфейса} могут быть произведены различные интерфейсные действия. В зависимости от выполненного интерфейсного действия и компонента, над которым оно было выполнено, происходит инициирование некоторого \textit{внутреннего действия системы}. Для задания такого инициируемого при взаимодействии с пользовательским интерфейсом действия и используется указанное отношение. Первым компонентом связки отношения \textit{инициируемое пользовательским интерфейсом действие*} является связка, элементами которой являются элемент множества компонентов пользовательского интерфейса и элемент множества \textit{класс интерфейсных действий пользователя}. Вторым компонентом является элемент множества \textit{класс внутренних действий системы}.}\scniselement{квазибинарное отношение}
\scniselement{ориентированное отношение}
\scnrelfrom{первый домен}{компонент пользовательского интерфейса $\cup$ класс интерфейсных действий пользователя}
\scnrelfrom{второй домен}{класс внутренних действий системы}
\scnrelfrom{описание примера}{\scnfileimage[40em]{figures/sd_ui/ui_initiated_action.png}}

\bigskip
\end{scnsubstruct}
\scnendcurrentsectioncomment
\end{SCn}
