\begin{SCn}
\scnsectionheader{\currentname}
\begin{scnsubstruct}
\scnrelfrom{соавтор}{Садовский М.Е.}
\begin{scnrelfromlist}{дочерний раздел}
{
    \scnitem{Предметная область и онтологий интерфейсных действий пользователей ostis-систем}
    \scnitem{Предметная область и онтология естественных языков}
}
\end{scnrelfromlist}
\scnheader{Предметная область интерфейсов ostis-систем}
\scniselement{предметная область}
\begin{scnhaselementrolelist}{класс объектов исследования}
    \scnitem{пользовательский интефейс}
\end{scnhaselementrolelist}
\begin{scnhaselementrolelist}{класс объектов исследования}
{
    \scnitem{командный пользовательский интерфейс}
    \scnitem{графический пользовательский интерфейс}
    \scnitem{WIMP-интерфейс}
    \scnitem{SILK-интерфейс}
    \scnitem{естественно-языковой интерфейс}
    \scnitem{речевой интерфейс}
    \scnitem{пользовательский интерфейс ostis-системы}
    \scnitem{компонент пользовательского интерфейса}
    \scnitem{атомарный компонент пользовательского интерфейса}
    \scnitem{неатомарный компонент пользовательского интерфейса}
    \scnitem{визуальная часть пользовательского интерфейса ostis-системы}
    \scnitem{компонент пользовательского интерфейса для представления}
    \scnitem{компонент вывода}
    \scnitem{компонент выполнения}
    \scnitem{параграф}
    \scnitem{декоративный компонент пользовательского интерфейса}
    \scnitem{контейнер}
    \scnitem{меню}
    \scnitem{строка меню}
    \scnitem{панель инструментов}
    \scnitem{панель вкладок}
    \scnitem{окно}
    \scnitem{модальное окно}
    \scnitem{немодальное окно}
    \scnitem{интерактивный компонент пользовательского интерфейса}
    \scnitem{флаговая кнопка}
    \scnitem{радиокнопка}
    \scnitem{переключатель}
    \scnitem{кнопка-счетчик}
    \scnitem{полоса прокрутки}
    \scnitem{кнопка}
}\end{scnhaselementrolelist}
\scnheader{пользовательский интерфейс}
\scnsuperset{командный пользовательский интерфейс}
\scnsuperset{графический пользовательский интерфейс}
\scnsuperset{WIMP-интерфейс}
\scnsuperset{пользовательский интерфейс ostis-системы}
\scnhaselement{Пользовательский интерфейс Метасистемы IMS.ostis}
\scnhaselement{Пользовательский интерфейс ИСС по геометрии}
\scnidtf{Пользовательский интерфейс интеллектуальной справочной системы по геометрии}
\scnhaselement{Пользовательский интерфейс ИСС по дискретной математике}
\scnidtf{Пользовательский интерфейс интеллектуальной справочной системы по дискретной математике}
\scnhaselement{Пользовательский интерфейс ИСС по географии}
\scnidtf{Пользовательский интерфейс интеллектуальной справочной системы по географии}
\scnhaselement{Пользовательский интерфейс ИСС по искусственным нейронным сетям}
\scnidtf{Пользовательский интерфейс интеллектуальной справочной системы по искусственным нейронным сетям}
\scnhaselement{Пользовательский интерфейс ИСС по лингвистике}
\scnidtf{Пользовательский интерфейс интеллектуальной справочной системы по лингвистике}
\scnsuperset{SILK-интерфейс}
\scnidtf{(Speech  речь, Image  образ, Language  язык, Knowledge  знание)}
\scnsuperset{естественно-языковой интерфейс}
\scnsuperset{речевой интерфейс}
\scnheader{пользовательский интерфейс}
\scntext{explanation}{\textit{пользовательский интерфейс} -- один из наиболее важных компонентов компьютерной системы. Представляет собой совокупность аппаратных и программных средств, обеспечивающих обмен информацией между пользователем и компьютерной системой.}\scnheader{командный пользовательский интерфейс}
\scntext{explanation}{\textit{командный пользовательский интерфейс} -- пользовательский интерфейс, при котором обмен информацией между компьютерной системой и пользователем осуществляется путем написания текстовых инструкций или команд.}\scnheader{графический пользовательский интерфейс}
\scntext{explanation}{\textit{графический пользовательский интерфейс} -- пользовательский интерфейс, при котором обмен информацией между компьютерной системой и пользователем осуществляется при помощи графических компонентов компьютерной системы.}\scnheader{WIMP-интерфейс}
\scntext{explanation}{\textit{WIMP-интерфейс} -- пользовательский интерфейс, при котором обмен информацией между компьютерной системой и пользователем осуществляется в форме диалога при помощью окон, меню и других элементов управления.}\scnheader{SILK-интерфейс}
\scntext{explanation}{\textit{SILK-интерфейс} -- пользовательский интерфейс, наиболее приближенный к естественной для человека форме общения. Компьютерная система находит для себя команды, анализируя человеческую речь и находя в ней ключевые фразы. Результат выполнения команд преобразуется в понятную человеку форму, например, в естественно-языковую форму или изображение.}\scnheader{естественно-языковой интерфейс}
\scntext{explanation}{\textit{естественно-языковой интерфейс} -- SILK-интерфейс, обмен информацией между компьютерной системой и пользователем в котором происходит за счёт диалога. Диалог ведётся на одном из естественных языков.}\scnheader{речевой интерфейс}
\scntext{explanation}{\textit{речевой интерфейс} -- SILK-интерфейс, обмен информацией в котором происходит за счёт диалога, в процессе которого компьютерная система и пользователь общаются с помощью речи. Данный вид интерфейса наиболее приближен к естественному общению между людьми.}\scnheader{пользовательский интерфейс ostis-системы}
\scnsubset{ostis-система}
\scntext{explanation}{\textit{пользовательский интерфейс ostis-системы} представляет собой специализированную \textit{ostis-систему}, ориентированную на решение интерфейсных задач, и имеющую в своем составе базу знаний и решатель задач пользовательского интерфейса ostis-системы.
~\\Для решения задачи построения пользовательского интерфейса в базе знаний \textit{пользовательского интерфейса ostis-системы} необходимо наличие sc-модели \textit{компонентов пользовательского интерфейса}, \textit{интерфейсных действий пользователей}, а также классификации \textit{пользовательских интерфейсов} вцелом. При проектировании интерфейса используется компонентный подход,который предполагает представление всего интерфейса приложения в виде отдельных специфицированных компонентов, которые могут разрабатываться и совершенствоваться независимо.}\scnheader{компонент пользовательского интерфейса}
\scntext{explanation}{\textit{компонент пользовательского интерфейса} -- знак фрагмента базы знаний, имеющий определённую форму внешнего представления на экране.}\begin{scnsubdividing}
\scnitem{атомарный компонент пользовательского интерфейса}
\scnitem{неатомарный компонент пользовательского интерфейса}
\end{scnsubdividing}
\scnheader{атомарный компонент пользовательского интерфейса}
\scntext{explanation}{\textit{атомарный компонент пользовательского интерфейса} -- компонент пользовательского интерфейса, не содержащий в своём составе других компонентов пользовательского интерфейса.}\scnheader{неатомарный компонент пользовательского интерфейса}
\scntext{explanation}{\textit{неатомарный компонент пользовательского интерфейса} -- компонент пользовательского интерфейса, состоящий из других компонентов пользовательского интерфейса.}\scnheader{визуальная часть пользовательского интерфейса ostis-системы}
\scnsubset{неатомарный компонент пользовательского интерфейса}
\scntext{explanation}{\textit{визуальная часть пользовательского интерфейса ostis-системы} -- часть базы знаний пользовательского интерфейса ostis-системы, содержащая необходимые для отображения пользовательского интерфейса компоненты.}\scnheader{компонент пользовательского интерфейса}
\scnidtf{user interface component}
\scnsuperset{компонент пользовательского интерфейса для отображения}
\scnidtf{presentation user interface component}
\scnsuperset{компонент вывода}
\scnidtf{output}
\scnsuperset{компонент вывода изображения}
\scnidtf{image-output}
\scnsuperset{компонент вывода графической информации}
\scnidtf{graphical-output}
\scnsuperset{диаграмма}
\scnidtf{chart}
\scnsuperset{карта}
\scnidtf{map}
\scnsuperset{индикатор выполнения}
\scnidtf{progress-bar}
\scnsuperset{компонент вывода видео}
\scnidtf{video-output}
\scnsuperset{компонент вывода звука}
\scnidtf{sound-output}
\scnsuperset{компонент вывода текста}
\scnidtf{text-output}
\scnsuperset{заголовок}
\scnidtf{headline}
\scnsuperset{параграф}
\scnidtf{paragraph}
\scnsuperset{сообщение}
\scnidtf{message}
\scnsuperset{декоративный компонент пользовательского интерфейса}
\scnidtf{decorative user interface component}
\scnsuperset{разделитель}
\scnidtf{separator}
\scnsuperset{пустое пространство}
\scnidtf{blank-space}
\scnsuperset{контейнер}
\scnidtf{container}
\scnsuperset{меню}
\scnidtf{menu}
\scnsuperset{строка меню}
\scnidtf{menu-bar}
\scnsuperset{панель инструментов}
\scnidtf{tool-bar}
\scnsuperset{строка состояния}
\scnidtf{status-bar}
\scnsuperset{таблично-строковый контейнер}
\scnidtf{table-row-container}
\scnsuperset{списковый контейнер}
\scnidtf{list-container}
\scnsuperset{таблично-клеточный контейнер}
\scnidtf{table-cell-container}
\scnsuperset{древовидный контейнер}
\scnidtf{tree-container}
\scnsuperset{панель вкладок}
\scnidtf{tab-pane}
\scnsuperset{панель вращения}
\scnidtf{spin-pane}
\scnsuperset{узловой контейнер}
\scnidtf{tree-node-container}
\scnsuperset{панель прокрутки}
\scnidtf{scroll-pane}
\scnsuperset{окно}
\scnidtf{window}
\scnsuperset{модальное окно}
\scnidtf{modal-window}
\scnsuperset{немодальное окно}
\scnidtf{non-modal-window}
\scnsuperset{интерактивный компонент пользовательского интерфейса}
\scnidtf{interactive user interface component}
\scnsuperset{компонент ввода данных}
\scnidtf{data-input-component}
\scnsuperset{компонент ввода данных с прямой ответной реакцией}
\scnidtf{data-input-component-with-direct-feedback}
\scnsuperset{компонент ввода текста с прямой ответной реакцией}
\scnidtf{text-input-component-with-direct-feedback}
\scnsuperset{многострочное текстовое поле}
\scnidtf{multi-line-text-field}
\scnsuperset{однострочное текстовое поле}
\scnidtf{single-line-text-field}
\scnsuperset{ползунок}
\scnidtf{slider}
\scnsuperset{область рисования}
\scnidtf{drawing-area}
\scnsuperset{компонент выбора}
\scnidtf{selection-component}
\scnsuperset{компонент выбора нескольких значений}
\scnidtf{selection-component-multiple-values}
\scnsuperset{компонент выбора одного значения}
\scnidtf{selection-component-single-values}
\scnsuperset{компонент выбора данных}
\scnidtf{selectable-data-representation}
\scnsuperset{флаговая кнопка}
\scnidtf{check-box}
\scnsuperset{радиокнопка}
\scnidtf{radio-button}
\scnsuperset{переключатель}
\scnidtf{toggle-button}
\scnsuperset{выбираемый элемент}
\scnidtf{selectable-item}
\scnsuperset{компонент ввода данных без прямой ответной реакции}
\scnidtf{data-input-component-without-direct-feedback}
\scnsuperset{кнопка-счётчик}
\scnidtf{spin-button}
\scnsuperset{компонент речевого ввода}
\scnidtf{speech-input}
\scnsuperset{компонент ввода движений}
\scnidtf{motion-input}
\scnsuperset{компонент для представления и взаимодействия с пользователем}
\scnidtf{presentation-manipulation-component}
\scnsuperset{активирующий компонент}
\scnidtf{activating-component}
\scnsuperset{компонент непрерывной манипуляции}
\scnidtf{continuous-manipulation-component}
\scnsuperset{полоса прокрутки}
\scnidtf{scrollbar}
\scnsuperset{компонент редактирования размера}
\scnidtf{resizer}
\scnsuperset{компонент запроса действий}
\scnidtf{operation-trigger-component}
\scnsuperset{компонент выбора команд}
\scnidtf{command-selection-component}
\scnsuperset{кнопка}
\scnidtf{button}
\scnsuperset{пункт меню}
\scnidtf{menu-item}
\scnsuperset{компонент ввода команд}
\scnidtf{command-input-component}
\scnheader{компонент пользовательского интерфейса для представления}
\scntext{explanation}{\textit{компонент пользовательского интерфейса для представления} -- компонент пользовательского интерфейса, не подразумевающий взаимодействия с пользователем.}\scnheader{компонент вывода}
\scntext{explanation}{\textit{компонент вывода} -- компонент пользовательского интерфейса, предназначенный для представления информации.}\scnheader{индикатор выполнения}
\scntext{explanation}{\textit{индикатор выполнения} -- компонент пользовательского интерфейса, предназначенный для отображения процента выполнения какой-либо задачи.}\scnheader{параграф}
\scntext{explanation}{\textit{параграф} -- компонент пользовательского интерфейса, предназначенный для отображения блоков текста. Он отделяется от других блоков пустой строкой или первой строкой с отступом.}\scnheader{декоративный компонент пользовательского интерфейса}
\scntext{explanation}{\textit{декоративный компонент пользовательского интерфейса} -- компонент пользовательского интерфейса, предназначенный для стилизации интерфейса.}\scnheader{контейнер}
\scntext{explanation}{\textit{контейнер} -- компонент пользовательского интерфейса, задача которого состоит в размещении набора компонентов, включённых в его состав.}\scnheader{меню}
\scntext{explanation}{\textit{меню} -- компонент пользовательского интерфейса, содержащий несколько вариантов для выбора пользователем.}\scnheader{строка меню}
\scntext{explanation}{\textit{строка меню} -- горизонтальная полоса , содержащая ярлыки меню. Строка меню предоставляет пользователю место в окне, где можно найти большинство основных функций программы.}\scnheader{панель инструментов}
\scntext{explanation}{\textit{панель инструментов} -- компонент пользовательского интерфейса, на котором размещаются элементы ввода или вывода данных.}\scnheader{панель вкладок}
\scntext{explanation}{\textit{панель вкладок} -- контейнер, который может содержать несколько вкладок (секций) внутри, которые могут быть отображены, нажав на вкладке с названием в верхней части панели. Одновременно отображается только одна вкладка.}\scnheader{окно}
\scntext{explanation}{\textit{окно} -- обособленная область экрана, содержащая различные элементы пользовательского интерфейса. Окна могут располагаться поверх друг друга.}\scnheader{модальное окно}
\scntext{explanation}{\textit{модальное окно} -- окно, которое блокирует работу пользователя с системой до тех пор, пока пользователь окно не закроет.}\scnheader{немодальное окно}
\scntext{explanation}{\textit{немодальное окно} -- окно, которое позволяет выполнять переключение между данным окном и другим окном без необходимости закрытия окна.}\scnheader{интерактивный компонент пользовательского интерфейса}
\scntext{explanation}{\textit{интерактивный компонент пользовательского интерфейса} -- компонент пользовательского интерфейса, с помощью которого осуществляется взаимодействие с пользователем.}\scnheader{флаговая кнопка}
\scntext{explanation}{\textit{флаговая кнопка} -- компонент пользовательского интерфейса, позволяющий пользователю управлять параметром с двумя состояниями  включено и отключено.}\scnheader{радиокнопка}
\scntext{explanation}{\textit{радиокнопка} -- компонент пользовательского интерфейса, который позволяет пользователю выбрать одну опцию из предопределенного набора.}\scnheader{переключатель}
\scntext{explanation}{\textit{переключатель} -- компонент пользовательского интерфейса, который позволяет пользователю переключаться между двумя состояниями.}\scnheader{кнопка-счетчик}
\scntext{explanation}{\textit{кнопка-счетчик} -- компонент пользовательского интерфейса, как правило, ориентированный вертикально, с помощью которого пользователь может изменить значение в прилегающем текстовом поле, в результате чего значение в текстовом поле увеличивается или уменьшается.}\scnheader{полоса прокрутки}
\scntext{explanation}{\textit{полоса прокрутки} -- компонент пользовательского интерфейса, который используется для отображения компонентов пользовательского интерфейса, больших по размеру, чем используемый для их отображения контейнер.}\scnheader{кнопка}
\scntext{explanation}{\textit{кнопка} -- компонент пользовательского интерфейса, при нажатии на который происходит программно связанное с этим нажатием действие либо событие.}\scnheader{Стартовая страница пользовательского интерфейса Метасистемы IMS.ostis}
\scniselement{Визуальная часть пользовательского интерфейса Метасистемы IMS.ostis}
\scnsubset{визуальная часть пользовательского интерфейса ostis-системы}
\scnrelto{часть}{Пользовательский интерфейс Метасистемы IMS.ostis}
\scniselement{окно}
\scnrelfrom{иллюстрация}{\scnfileimage[20em]{figures/sd_ui/startPage.png}}
\begin{scnrelfromset}{декомпозиция}
    \scnitem{Панель навигации
~\\\scniselement{неатомарный компонент пользовательского интерфейса}
    \begin{scnrelfromset}{декомпозиция}
        \scnitem{Главное меню
~\\\scniselement{меню}
        \begin{scnrelfromset}{декомпозиция}
            \scnitem{Пункт меню для навигации по ключевым понятиям
~\\\scniselement{пункт меню}
            }
            \scnitem{Пункт меню для выполнения команд просмотра базы знаний
~\\\scniselement{пункт меню}
            }
            \scnitem{Компонент перехода в экспертный режим
~\\\scniselement{переключатель}
            }
        \end{scnrelfromset}
        }
        \scnitem{Компонент выбора языка
~\\\scniselement{компонент выбора одного значения}
        }
        \scnitem{Компонент авторизации
~\\\scniselement{кнопка}
        }
    \end{scnrelfromset}
    }

    \scnitem{Блок истории запросов пользователя
~\\\scniselement{неатомарный компонент пользовательского интерфейса}
    }

    \scnitem{Основной блок
~\\\scniselement{неатомарный компонент пользовательского интерфейса}
    \begin{scnrelfromset}{декомпозиция}
        \scnitem{Главное окно
~\\\scniselement{окно}
        }
        \scnitem{Панель инструментов
~\\\scniselement{неатомарный компонент пользовательского интерфейса}
        \begin{scnrelfromset}{декомпозиция}
            \scnitem{Кнопка отправки содержимого главного окна на печать
~\\\scniselement{кнопка}
            }
            \scnitem{Кнопка управления видимостью блока истории запросов пользователя
~\\\scniselement{кнопка}
            }
            \scnitem{Кнопка отображения ссылки на текущий запрос пользователя
~\\\scniselement{кнопка}
            }
            \scnitem{Поле поиска
~\\\scniselement{однострочное текстовое поле}
            }
        \end{scnrelfromset}
        }
    \end{scnrelfromset}
    }

    \scnitem{Панель отображения информации об авторских правах
~\\\scniselement{неатомарный компонент пользовательского интерфейса}
    }
\end{scnrelfromset}
\bigskip\end{scnsubstruct}
\scnendcurrentsectioncomment
\end{SCn}
