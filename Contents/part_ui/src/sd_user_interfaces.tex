\begin{SCn}
\scnsectionheader{Предметная область и онтология интерфейсов ostis-систем}
\begin{scnsubstruct}
\scnrelfrom{соавтор}{Садовский М.Е.}
\begin{scnrelfromlist}{дочерний раздел}
    \scnitem{Предметная область и онтология интерфейсных действий пользователей ostis-систем}
    \scnitem{Предметная область и онтология естественных языков}
\end{scnrelfromlist}

\scnheader{Предметная область интерфейсов ostis-систем}
\scniselement{предметная область}
\begin{scnhaselementrole}{максимальный класс объектов исследования}
    {пользовательский интефейс}
\end{scnhaselementrole}
\begin{scnhaselementrolelist}{класс объектов исследования}
    \scnitem{командный пользовательский интерфейс}
    \scnitem{графический пользовательский интерфейс}
    \scnitem{WIMP-интерфейс}
    \scnitem{SILK-интерфейс}
    \scnitem{естественно-языковой интерфейс}
    \scnitem{речевой интерфейс}
    \scnitem{пользовательский интерфейс ostis-системы}
    \scnitem{компонент пользовательского интерфейса}
    \scnitem{атомарный компонент пользовательского интерфейса}
    \scnitem{неатомарный компонент пользовательского интерфейса}
    \scnitem{визуальная часть пользовательского интерфейса ostis-системы}
    \scnitem{компонент пользовательского интерфейса для представления}
    \scnitem{компонент вывода}
    \scnitem{компонент выполнения}
    \scnitem{параграф}
    \scnitem{декоративный компонент пользовательского интерфейса}
    \scnitem{контейнер}
    \scnitem{меню}
    \scnitem{строка меню}
    \scnitem{панель инструментов}
    \scnitem{панель вкладок}
    \scnitem{окно}
    \scnitem{модальное окно}
    \scnitem{немодальное окно}
    \scnitem{интерактивный компонент пользовательского интерфейса}
    \scnitem{флаговая кнопка}
    \scnitem{радиокнопка}
    \scnitem{переключатель}
    \scnitem{кнопка-счетчик}
    \scnitem{полоса прокрутки}
    \scnitem{кнопка}
\end{scnhaselementrolelist}

\scnheader{пользовательский интерфейс}
\scnsuperset{командный пользовательский интерфейс}
\scnsuperset{графический пользовательский интерфейс}	
\begin{scnindent}
	\scnsuperset{WIMP-интерфейс}
	\begin{scnindent}
		\scnsuperset{пользовательский интерфейс ostis-системы}
		\begin{scnindent}
			\scnhaselement{Пользовательский интерфейс Метасистемы IMS.ostis}
			\scnhaselement{Пользовательский интерфейс ИСС по геометрии}
			\begin{scnindent}
				\scnidtf{Пользовательский интерфейс интеллектуальной справочной системы по геометрии}
			\end{scnindent}
			\scnhaselement{Пользовательский интерфейс ИСС по дискретной математике}
			\begin{scnindent}
				\scnidtf{Пользовательский интерфейс интеллектуальной справочной системы по дискретной математике}
			\end{scnindent}
			\scnhaselement{Пользовательский интерфейс ИСС по географии}
			\begin{scnindent}
				\scnidtf{Пользовательский интерфейс интеллектуальной справочной системы по 	географии}
			\end{scnindent}
			\scnhaselement{Пользовательский интерфейс ИСС по искусственным нейронным сетям}
			\begin{scnindent}
				\scnidtf{Пользовательский интерфейс интеллектуальной справочной системы по искусственным нейронным сетям}
			\end{scnindent}
			\scnhaselement{Пользовательский интерфейс ИСС по лингвистике}
			\begin{scnindent}
				\scnidtf{Пользовательский интерфейс интеллектуальной справочной системы по лингвистике}
			\end{scnindent}
		\end{scnindent}
	\end{scnindent}
	\scnsuperset{SILK-интерфейс}
	\begin{scnindent}
		\scnidtf{(Speech --- речь, Image --- образ, Language --- язык, Knowledge --- знание)}
		\scnsuperset{естественно-языковой интерфейс}
		\begin{scnindent}
			\scnsuperset{речевой интерфейс}
		\end{scnindent}
	\end{scnindent}
\end{scnindent}

\scnheader{пользовательский интерфейс}
\scntext{пояснение}{\textit{пользовательский интерфейс} --- один из наиболее важных компонентов компьютерной системы. Представляет собой совокупность аппаратных и программных средств, обеспечивающих обмен информацией между пользователем и компьютерной системой.}

\scnheader{командный пользовательский интерфейс}
\scntext{пояснение}{\textit{командный пользовательский интерфейс} --- пользовательский интерфейс, при котором обмен информацией между компьютерной системой и пользователем осуществляется путем написания текстовых инструкций или команд.}

\scnheader{графический пользовательский интерфейс}
\scntext{пояснение}{\textit{графический пользовательский интерфейс} --- пользовательский интерфейс, при котором обмен информацией между компьютерной системой и пользователем осуществляется при помощи графических компонентов компьютерной системы.}

\scnheader{WIMP-интерфейс}
\scntext{пояснение}{\textit{WIMP-интерфейс} --- пользовательский интерфейс, при котором обмен информацией между компьютерной системой и пользователем осуществляется в форме диалога при помощью окон, меню и других элементов управления.}

\scnheader{SILK-интерфейс}
\scntext{пояснение}{\textit{SILK-интерфейс} --- пользовательский интерфейс, наиболее приближенный к естественной для человека форме общения. Компьютерная система находит для себя команды, анализируя человеческую речь и находя в ней ключевые фразы. Результат выполнения команд преобразуется в понятную человеку форму, например, в естественно-языковую форму или изображение.}

\scnheader{естественно-языковой интерфейс}
\scntext{пояснение}{\textit{естественно-языковой интерфейс} --- SILK-интерфейс, обмен информацией между компьютерной системой и пользователем в котором происходит за счёт диалога. Диалог ведётся на одном из естественных языков.}

\scnheader{речевой интерфейс}
\scntext{пояснение}{\textit{речевой интерфейс} --- SILK-интерфейс, обмен информацией в котором происходит за счёт диалога, в процессе которого компьютерная система и пользователь общаются с помощью речи. Данный вид интерфейса наиболее приближен к естественному общению между людьми.}

\scnheader{пользовательский интерфейс ostis-системы}
\scnsubset{ostis-система}
\scntext{пояснение}{\textit{пользовательский интерфейс ostis-системы} представляет собой специализированную \textit{ostis-систему}, ориентированную на решение интерфейсных задач и имеющую в своем составе базу знаний и решатель задач пользовательского интерфейса ostis-системы.\\
	Для решения задачи построения пользовательского интерфейса в базе знаний \textit{пользовательского интерфейса ostis-системы} необходимо наличие sc-модели \textit{компонентов пользовательского интерфейса}, \textit{интерфейсных действий пользователей}, а также классификации \textit{пользовательских интерфейсов} в целом. При проектировании интерфейса используется компонентный подход, который предполагает представление всего интерфейса приложения в виде отдельных специфицированных компонентов, которые могут разрабатываться и совершенствоваться независимо.}

\scnheader{компонент пользовательского интерфейса}
\scntext{пояснение}{\textit{компонент пользовательского интерфейса} --- знак фрагмента базы знаний, имеющий определённую форму внешнего представления на экране.}
\begin{scnsubdividing}
	\scnitem{атомарный компонент пользовательского интерфейса}
	\scnitem{неатомарный компонент пользовательского интерфейса}
\end{scnsubdividing}

\scnheader{атомарный компонент пользовательского интерфейса}
\scntext{пояснение}{\textit{атомарный компонент пользовательского интерфейса} --- компонент пользовательского интерфейса, не содержащий в своём составе других компонентов пользовательского интерфейса.}

\scnheader{неатомарный компонент пользовательского интерфейса}
\scntext{пояснение}{\textit{неатомарный компонент пользовательского интерфейса} --- компонент пользовательского интерфейса, состоящий из других компонентов пользовательского интерфейса.}

\scnheader{визуальная часть пользовательского интерфейса ostis-системы}
\scnsubset{неатомарный компонент пользовательского интерфейса}
\scntext{пояснение}{\textit{визуальная часть пользовательского интерфейса ostis-системы} --- часть базы знаний пользовательского интерфейса ostis-системы, содержащая необходимые для отображения пользовательского интерфейса компоненты.}

\scnheader{компонент пользовательского интерфейса}
\scnidtf{user interface component}
\scnsuperset{компонент пользовательского интерфейса для отображения}
\begin{scnindent}
	\scnidtf{presentation user interface component}
	\scnsuperset{компонент вывода}
	\begin{scnindent}
		\scnidtf{output}
		\scnsuperset{компонент вывода изображения}
		\begin{scnindent}
			\scnidtf{image-output}
		\end{scnindent}
		\scnsuperset{компонент вывода графической информации}
		\begin{scnindent}
			\scnidtf{graphical-output}
			\scnsuperset{диаграмма}
			\begin{scnindent}
				\scnidtf{chart}
			\end{scnindent}
			\scnsuperset{карта}
			\begin{scnindent}
				\scnidtf{map}
			\end{scnindent}
			\scnsuperset{индикатор выполнения}
			\begin{scnindent}
				\scnidtf{progress-bar}
			\end{scnindent}
		\end{scnindent}
		\scnsuperset{компонент вывода видео}
		\begin{scnindent}
			\scnidtf{video-output}
		\end{scnindent}
		\scnsuperset{компонент вывода звука}
		\begin{scnindent}
			\scnidtf{sound-output}
		\end{scnindent}
		\scnsuperset{компонент вывода текста}
		\begin{scnindent}
			\scnidtf{text-output}
			\scnsuperset{заголовок}
			\begin{scnindent}
				\scnidtf{headline}
			\end{scnindent}
			\scnsuperset{параграф}
			\begin{scnindent}
				\scnidtf{paragraph}
			\end{scnindent}
			\scnsuperset{сообщение}
			\begin{scnindent}
				\scnidtf{message}
			\end{scnindent}
		\end{scnindent}
	\end{scnindent}
	\scnsuperset{декоративный компонент пользовательского интерфейса}
	\begin{scnindent}
		\scnidtf{decorative user interface component}
		\scnsuperset{разделитель}
		\begin{scnindent}
			\scnidtf{separator}
		\end{scnindent}
		\scnsuperset{пустое пространство}
		\begin{scnindent}
			\scnidtf{blank-space}
		\end{scnindent}
	\end{scnindent}
	\scnsuperset{контейнер}
	\begin{scnindent}
		\scnidtf{container}
		\scnsuperset{меню}
		\begin{scnindent}
			\scnidtf{menu}
		\end{scnindent}
		\scnsuperset{строка меню}
		\begin{scnindent}
			\scnidtf{menu-bar}
		\end{scnindent}
		\scnsuperset{панель инструментов}
		\begin{scnindent}
			\scnidtf{tool-bar}
		\end{scnindent}
		\scnsuperset{строка состояния}
		\begin{scnindent}
			\scnidtf{status-bar}
		\end{scnindent}
		\scnsuperset{таблично-строковый контейнер}
		\begin{scnindent}
			\scnidtf{table-row-container}
		\end{scnindent}
		\scnsuperset{списковый контейнер}
		\begin{scnindent}
			\scnidtf{list-container}
		\end{scnindent}
		\scnsuperset{таблично-клеточный контейнер}
		\begin{scnindent}
			\scnidtf{table-cell-container}
		\end{scnindent}
		\scnsuperset{древовидный контейнер}
		\begin{scnindent}
			\scnidtf{tree-container}
		\end{scnindent}
		\scnsuperset{панель вкладок}
		\begin{scnindent}
			\scnidtf{tab-pane}
		\end{scnindent}
		\scnsuperset{панель вращения}
		\begin{scnindent}
			\scnidtf{spin-pane}
		\end{scnindent}
		\scnsuperset{узловой контейнер}
		\begin{scnindent}
			\scnidtf{tree-node-container}
		\end{scnindent}
		\scnsuperset{панель прокрутки}
		\begin{scnindent}
			\scnidtf{scroll-pane}
		\end{scnindent}
		\scnsuperset{окно}
		\begin{scnindent}
			\scnidtf{window}
			\scnsuperset{модальное окно}
			\begin{scnindent}
				\scnidtf{modal-window}
			\end{scnindent}
			\scnsuperset{немодальное окно}
			\begin{scnindent}
				\scnidtf{non-modal-window}
			\end{scnindent}
		\end{scnindent}
	\end{scnindent}
\end{scnindent}	
\scnsuperset{интерактивный компонент пользовательского интерфейса}
\begin{scnindent}
	\scnidtf{interactive user interface component}
	\scnsuperset{компонент ввода данных}
	\begin{scnindent}
		\scnidtf{data-input-component}
		\scnsuperset{компонент ввода данных с прямой ответной реакцией}
		\begin{scnindent}
			\scnidtf{data-input-component-with-direct-feedback}
			\scnsuperset{компонент ввода текста с прямой ответной реакцией}
			\begin{scnindent}
                \scnidtf{text-input-component-with-direct-feedback}
				\scnsuperset{многострочное текстовое поле}
				\begin{scnindent}
					\scnidtf{multi-line-text-field}
				\end{scnindent}
				\scnsuperset{однострочное текстовое поле}
				\begin{scnindent}
					\scnidtf{single-line-text-field}
				\end{scnindent}
			\end{scnindent}
			\scnsuperset{ползунок}
			\begin{scnindent}
				\scnidtf{slider}
			\end{scnindent}
			\scnsuperset{область рисования}
			\begin{scnindent}
				\scnidtf{drawing-area}
			\end{scnindent}
			\scnsuperset{компонент выбора}
			\begin{scnindent}
				\scnidtf{selection-component}
				\scnsuperset{компонент выбора нескольких значений}
				\begin{scnindent}
					\scnidtf{selection-component-multiple-values}
				\end{scnindent}
				\scnsuperset{компонент выбора одного значения}
				\begin{scnindent}
					\scnidtf{selection-component-single-values}
				\end{scnindent}
			\end{scnindent}
			\scnsuperset{компонент выбора данных}
			\begin{scnindent}
				\scnidtf{selectable-data-representation}
				\scnsuperset{флаговая кнопка}
				\begin{scnindent}
					\scnidtf{check-box}
				\end{scnindent}
				\scnsuperset{радиокнопка}
				\begin{scnindent}
					\scnidtf{radio-button}
				\end{scnindent}
				\scnsuperset{переключатель}
				\begin{scnindent}
					\scnidtf{toggle-button}
				\end{scnindent}
				\scnsuperset{выбираемый элемент}
				\begin{scnindent}
					\scnidtf{selectable-item}
				\end{scnindent}
			\end{scnindent}
		\end{scnindent}	
		\scnsuperset{компонент ввода данных без прямой ответной реакции}
		\begin{scnindent}
			\scnidtf{data-input-component-without-direct-feedback}
			\scnsuperset{кнопка-счётчик}
			\begin{scnindent}
				\scnidtf{spin-button}
			\end{scnindent}
			\scnsuperset{компонент речевого ввода}
			\begin{scnindent}
				\scnidtf{speech-input}
			\end{scnindent}
			\scnsuperset{компонент ввода движений}
			\begin{scnindent}
				\scnidtf{motion-input}
			\end{scnindent}
		\end{scnindent}
	\end{scnindent}
	\scnsuperset{компонент для представления и взаимодействия с пользователем}
	\begin{scnindent}
		\scnidtf{presentation-manipulation-component}
		\scnsuperset{активирующий компонент}
		\begin{scnindent}
			\scnidtf{activating-component}
		\end{scnindent}
		\scnsuperset{компонент непрерывной манипуляции}
		\begin{scnindent}
			\scnidtf{continuous-manipulation-component}
			\scnsuperset{полоса прокрутки}
			\begin{scnindent}
				\scnidtf{scrollbar}
			\end{scnindent}
			\scnsuperset{компонент редактирования размера}
			\begin{scnindent}
				\scnidtf{resizer}
			\end{scnindent}
		\end{scnindent}
	\end{scnindent}
	\scnsuperset{компонент запроса действий}
	\begin{scnindent}
		\scnidtf{operation-trigger-component}
		\scnsuperset{компонент выбора команд}
		\begin{scnindent}
			\scnidtf{command-selection-component}
			\scnsuperset{кнопка}
			\begin{scnindent}
				\scnidtf{button}
			\end{scnindent}
			\scnsuperset{пункт меню}
			\begin{scnindent}
				\scnidtf{menu-item}
			\end{scnindent}
		\end{scnindent}
		\scnsuperset{компонент ввода команд}
		\begin{scnindent}
			\scnidtf{command-input-component}
		\end{scnindent}
	\end{scnindent}
\end{scnindent}

\scnheader{компонент пользовательского интерфейса для представления}
\scntext{пояснение}{\textit{компонент пользовательского интерфейса для представления} --- компонент пользовательского интерфейса, не подразумевающий взаимодействия с пользователем.}

\scnheader{компонент вывода}
\scntext{пояснение}{\textit{компонент вывода} --- компонент пользовательского интерфейса, предназначенный для представления информации.}

\scnheader{индикатор выполнения}
\scntext{пояснение}{\textit{индикатор выполнения} --- компонент пользовательского интерфейса, предназначенный для отображения процента выполнения какой-либо задачи.}

\scnheader{параграф}
\scntext{пояснение}{\textit{параграф} --- компонент пользовательского интерфейса, предназначенный для отображения блоков текста. Он отделяется от других блоков пустой строкой или первой строкой с отступом.}

\scnheader{декоративный компонент пользовательского интерфейса}
\scntext{пояснение}{\textit{декоративный компонент пользовательского интерфейса} --- компонент пользовательского интерфейса, предназначенный для стилизации интерфейса.}

\scnheader{контейнер}
\scntext{пояснение}{\textit{контейнер} --- компонент пользовательского интерфейса, задача которого состоит в размещении набора компонентов, включённых в его состав.}

\scnheader{меню}
\scntext{пояснение}{\textit{меню} --- компонент пользовательского интерфейса, содержащий несколько вариантов для выбора пользователем.}

\scnheader{строка меню}
\scntext{пояснение}{\textit{строка меню} --- горизонтальная полоса, содержащая ярлыки меню. Строка меню предоставляет пользователю место в окне, где можно найти большинство основных функций программы.}

\scnheader{панель инструментов}
\scntext{пояснение}{\textit{панель инструментов} --- компонент пользовательского интерфейса, на котором размещаются элементы ввода или вывода данных.}

\scnheader{панель вкладок}
\scntext{пояснение}{\textit{панель вкладок} --- контейнер, который может содержать несколько вкладок (секций) внутри, которые могут быть отображены, нажав на вкладке с названием в верхней части панели. Одновременно отображается только одна вкладка.}

\scnheader{окно}
\scntext{пояснение}{\textit{окно} --- обособленная область экрана, содержащая различные элементы пользовательского интерфейса. Окна могут располагаться поверх друг друга.}

\scnheader{модальное окно}
\scntext{пояснение}{\textit{модальное окно} --- окно, которое блокирует работу пользователя с системой до тех пор, пока пользователь окно не закроет.}

\scnheader{немодальное окно}
\scntext{пояснение}{\textit{немодальное окно} --- окно, которое позволяет выполнять переключение между данным окном и другим окном без необходимости закрытия окна.}

\scnheader{интерактивный компонент пользовательского интерфейса}
\scntext{пояснение}{\textit{интерактивный компонент пользовательского интерфейса} --- компонент пользовательского интерфейса, с помощью которого осуществляется взаимодействие с пользователем.}

\scnheader{флаговая кнопка}
\scntext{пояснение}{\textit{флаговая кнопка} --- компонент пользовательского интерфейса, позволяющий пользователю управлять параметром с двумя состояниями --- включено и отключено.}

\scnheader{радиокнопка}
\scntext{пояснение}{\textit{радиокнопка} --- компонент пользовательского интерфейса, который позволяет пользователю выбрать одну опцию из предопределенного набора.}

\scnheader{переключатель}
\scntext{пояснение}{\textit{переключатель} --- компонент пользовательского интерфейса, который позволяет пользователю переключаться между двумя состояниями.}

\scnheader{кнопка-счетчик}
\scntext{пояснение}{\textit{кнопка-счетчик} --- компонент пользовательского интерфейса, как правило, ориентированный вертикально, с помощью которого пользователь может изменить значение в прилегающем текстовом поле, в результате чего значение в текстовом поле увеличивается или уменьшается.}

\scnheader{полоса прокрутки}
\scntext{пояснение}{\textit{полоса прокрутки} --- компонент пользовательского интерфейса, который используется для отображения компонентов пользовательского интерфейса, больших по размеру, чем используемый для их отображения контейнер.}

\scnheader{кнопка}
\scntext{пояснение}{\textit{кнопка} --- компонент пользовательского интерфейса, при нажатии на который происходит программно связанное с этим нажатием действие либо событие.}

\scnheader{Стартовая страница пользовательского интерфейса Метасистемы IMS.ostis}
\scniselement{Визуальная часть пользовательского интерфейса Метасистемы IMS.ostis}
\begin{scnindent}
	\scnsubset{визуальная часть пользовательского интерфейса ostis-системы}
	\scnrelto{часть}{Пользовательский интерфейс Метасистемы IMS.ostis}
\end{scnindent}	
\scniselement{окно}
\scnrelfrom{иллюстрация}{\scnfileimage[40em]{Contents/part_ui/src/images/sd_ui/startPage.png}}
\begin{scnrelfromset}{декомпозиция}
    \scnitem{Панель навигации}
	\begin{scnindent}
		\scniselement{неатомарный компонент пользовательского интерфейса}
		\begin{scnrelfromset}{декомпозиция}
			\scnitem{Главное меню}
			\begin{scnindent}
				\scniselement{меню}
				\begin{scnrelfromset}{декомпозиция}
					\scnitem{Пункт меню для навигации по ключевым понятиям}
					\begin{scnindent}
						\scniselement{пункт меню}
					\end{scnindent}
					\scnitem{Пункт меню для выполнения команд просмотра базы знаний}
					\begin{scnindent}
						\scniselement{пункт меню}
					\end{scnindent}
					\scnitem{Компонент перехода в экспертный режим}
					\begin{scnindent}
						\scniselement{переключатель}
					\end{scnindent} 
				\end{scnrelfromset}
			\end{scnindent} 
			\scnitem{Компонент выбора языка}
			\begin{scnindent}
				\scniselement{компонент выбора одного значения}
			\end{scnindent} 
			\scnitem{Компонент авторизации}
			\begin{scnindent}
				\scniselement{кнопка}
			\end{scnindent} 
		\end{scnrelfromset}
	\end{scnindent}	 
    \scnitem{Блок истории запросов пользователя}
    	\begin{scnindent}
			\scniselement{неатомарный компонент пользовательского интерфейса}
		\end{scnindent}
    \scnitem{Основной блок}
	\begin{scnindent}
			\scniselement{неатомарный компонент пользовательского интерфейса}
		\begin{scnrelfromset}{декомпозиция}
			\scnitem{Главное}
			\begin{scnindent}
				\scniselement{окно}
			\end{scnindent}
			\scnitem{Панель инструментов}
			\begin{scnindent}
				\scniselement{неатомарный компонент пользовательского интерфейса}
				\begin{scnrelfromset}{декомпозиция}
					\scnitem{Кнопка отправки содержимого главного окна на печать}
					\begin{scnindent}
						\scniselement{кнопка}
					\end{scnindent} 
					\scnitem{Кнопка управления видимостью блока истории запросов пользователя}
					\begin{scnindent}
						\scniselement{кнопка}
					\end{scnindent} 
					\scnitem{Кнопка отображения ссылки на текущий запрос пользователя}
					\begin{scnindent}
						\scniselement{кнопка}
					\end{scnindent} 
					\scnitem{Поле поиска}
					\begin{scnindent}
						\scniselement{однострочное текстовое поле}
					\end{scnindent} 
				\end{scnrelfromset}
			\end{scnindent} 
		\end{scnrelfromset}
	\end{scnindent}
    \scnitem{Панель отображения информации об авторских правах}
	\begin{scnindent}
		\scniselement{неатомарный компонент пользовательского интерфейса}
	\end{scnindent}
\end{scnrelfromset}

\bigskip
\end{scnsubstruct}
\scnendcurrentsectioncomment
\end{SCn}
