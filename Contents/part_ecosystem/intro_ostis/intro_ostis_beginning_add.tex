\begin{scnrelfromset}{подвопрос}
    \scnfileitem{Недостатки современных интеллектуальных компьютерных систем}
    \scnfileitem{Недостатки современной технологии Искусственного интеллекта}
    \scnfileitem{Каким требованиям должна удовлетворять качественная технология разработки интеллектуальных компьютерных систем}
        \begin{scnrelfromset}{подвопрос}
            \scnfileitem{уточнить требования, представляемые к интеллектуальным компьютерным системам (что такое интеллектуальная компьютерная система)}
            \scnfileitem{уточнить, почему этого нет}
            \scnfileitem{как эти требования удовлетворить в рамках интеллектуальных компьютерных систем (принципы)}
            \scnfileitem{уточнить требования к технологии}
            \scnfileitem{понять, уточнить, почему, что мешает созданию технологии}
            \begin{scnrelfromset}{причина}
                \scnfileitem{сложность объекта}
                \scnfileitem{отсутствие понимания того, что задача такой сложности требует создания принципиально нового творческого коллектива с принципиально новой организацией взаимодействия}
            \end{scnrelfromset}
            \scnfileitem{как это сделать (принципы, лежащие в основе создания технологии интеллектуальных компьютерных систем)}
        \end{scnrelfromset}
    \scnfileitem{Что такое ИИ (как наука)}
    \begin{scnindent}
        \scniselement{научно-техническая дисциплина}
    \end{scnindent}
    \scnfileitem{Что такое интеллектуальная кибернетическая  система}
    \begin{scnindent}
        \scnsubset{кибернетическая система}
    \end{scnindent}
    \scnfileitem{Что такое технология проектирования и реализации интеллектуальная кибернетическая система}
    \scnfileitem{проблемы создания технологии проектирования}
    \scnfileitem{технология реализации от традиционных компьютеров к компьютерам, ориентированным на реализацию интеллектуальных кибернетических систем}
    \scnfileitem{Результат использования технологии проектирования и реализации --- это не отдельные интеллектуальные компьютерные системы и Экосистема из интеллектуальных компьютерных систем и людей}
    \scnfileitem{структура Экосистемы --- иерархическая система специализированных сообществ}
    \scnfileitem{Чем нас не устраивают те, интеллектуальные компьютерные системы, которые мы разрабатываем сейчас}
    \scnfileitem{Чем нас не устраивают современные технологии ИИ}
    \scnfileitem{Какие интеллектуальные компьютерные системы нам нужны}
    \scnfileitem{Какими свойствами и способностями мы хотели бы их наделить}
    \scnfileitem{высокая степень обучаемости в разных направлениях}
    \scnfileitem{расширение знаний без введения новых понятий}
    \scnfileitem{введение новых понятий без расширения многообразия видов знаний}
    \scnfileitem{расширение многообразия видов знаний}
    \scnfileitem{расширение моделей решения задач(новый вид методов + их интерпретация)}
    \scnfileitem{Какие технологии нам нужны}
    \scnfileitem{Почему таких икс и технологий ещё нет}
    \scnfileitem{Что мешает?}
    \scnfileitem{Что делать?}
    \scnfileitem{Какие недостатки имеют современные интеллектуальные системы}
    \scnfileitem{недостаточно высокий уровень интеллектуальности}
    \scnfileitem{нет эффективного взаимодействия(координации)}
    \scnfileitem{высокая степень обучаемости в разных направлениях}
    \scnfileitem{Какие недостатки имеют современные технологии Искусственного интеллекта}
    \scnfileitem{Какова трудоёмкость разработки выбранных икс}
    \scnfileitem{Какова трудоёмкость системной интеграции икс и их компонентов}
    \scnfileitem{Обеспечивается ли совместимость компонентов икс, разрабатываемых с помощью различных}
\end{scnrelfromset}
