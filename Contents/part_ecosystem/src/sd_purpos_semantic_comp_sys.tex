\begin{SCn}
    \scnsectionheader{Предметная область и онтология семантически совместимых интеллектуальных корпоративных ostis-систем различного назначения}
    \begin{scnsubstruct}

        \scnheader{Предметная область и онтология семантически совместимых интеллектуальный корпоративных ostis-систем различного назначения}
        \scniselement{предметная область}
        \begin{scnhaselementrole}{максимальный класс объектов исследования}
            {интеллектуальная комната данных}
        \end{scnhaselementrole}

        \scnheader{интеллектуальная комната данных}
        \scnidtf{Intelligent Data Room}
        \scnidtf{IDR}
        \scnidtf{система, которая позволяет отслеживать, анализировать и постепенно автоматизировать все процессы обработки данных в компании}
        \begin{scnrelfromset}{принципы работы}
            \scnfileitem{Интеллектуальные подсистемы (агенты) упорядочивают структуру ваших данных таким образом, что актуальная информация всегда доступна, а устаревшая информация автоматически архивируется или удаляется в соответствии с законами о хранении и защите данных.}
            \scnfileitem{Запросы к системе выполняются в виде простых инструкций, система помогает менеджерам вводить необходимую информацию, осуществляет частичную или полную автоматизацию обновления информации из баз данных, доступных через Интернет.}
            \scnfileitem{Искусственный интеллект (ИИ) выполняет структуризацию и классификацию документов и информации для принятия быстрых и правильных решений, автоматически обрабатывает документы и доступные базы данных для отбора ключевой информации, необходимой в данный момент и в будущем.}
            \scnfileitem{Существующее системное окружение на предприятии может быть легко подключено к искусственному интеллекту через открытые интерфейсы, вся информация остается доступной. Все ключевые системы данных синхронизируются с искусственным интеллектом, данные постоянно сравниваются друг с другом, чтобы избежать потерь.}
            \scnfileitem{Вся информация доступна в базе знаний, которая является источником данных для рабочих процессов, отчетности и комплексных проверок.}
            \scnfileitem{Таким образом, предлагаемая платформа на основе искусственного интеллекта позволяет представить всю компанию единым целостным образом.}
        \end{scnrelfromset}
        \begin{scnrelfromset}{достоинства внедрения}
            \scnfileitem{\textit{IDR} помогает собирать и оценивать информацию без преднамеренных искажений или ошибок, связанных с человеческим фактором.}
            \scnfileitem{Компания с \textit{IDR} полностью контролирует свои данные.}
            \scnfileitem{Система предоставляет только высококачественные, достоверные и актуальные данные.}
            \scnfileitem{Цифровое представление всех процессов компании обеспечивает интегрированную обработку информации внутри компании, что дает полную прозрачность управления, облегчает доступ ко всей информации и ее анализ.}
            \scnfileitem{Благодаря поддержке подсистем искусственного интеллекта все необходимые данные из документов, процессов и внешних источников могут быть извлечены, структурированы и грамотно оценены.}
            \scnfileitem{Искусственный интеллект предоставляет инструмент для интеллектуальной оцифровки и интеграции знаний вашей компании и взаимодействия между всеми заинтересованными сторонами в рамках вашей компании, как следствие - обеспечивает автоматическую поддержку соответствующих бизнес-процессов и устраняет локальные изолированные решения внутри компании, превращая ее в единую согласованную систему.}
        \end{scnrelfromset}
        \scntext{примечание}{Идея \textit{интеллектуальной комнаты данных} в общем случае может реализовываться двумя путями. Любой из них может реализовываться постепенно, с подключением к \textit{IDR} все новых и новых информационных ресурсов.}
        \scnsuperset{цифровой сотрудник}
        \scntext{пояснение}{Надстройка над уже существующими информационными ресурсами предприятия (различные базы данных, облачные и физические хранилища документов и т.д.). Для реализации \textit{цифрового сотрудника} необходимо в базе знаний \textit{IDR} в виде семейства онтологий описать метаинформацию об имеющихся информационных ресурсах (схемы баз данных, структуру и расположение документов и т.д.), а также механизмы доступа к этим ресурсам (например, их физическое расположение, языки запросов). На основе такого описания \textit{IDR} сможет автоматически построить необходимый набор запросов к нужным информационным ресурсам, интегрировать полученные ответы и выдать ответ пользователю в удобной ему форме. При этом пользователю системы не нужно знать, где именно и в какой форме хранится нужная ему информация, запрос к системе делается на языке, близком к естественному.}
        \begin{scnrelfromset}{достоинства}
            \scnfileitem{Нет необходимости в рамках базы знаний \textit{IDR} дублировать информацию, которая уже содержится в использовавшихся ранее информационных ресурсах, она может и дальше храниться в тех же местах и в той же форме. При этом данные могут быстро меняться, это никак не повлияет на работу системы.}
            \scnfileitem{Нет необходимости вносить изменения в уже налаженные процессы и используемое на предприятии ПО, резко переобучать людей и менять отлаженные схемы работы.}
            \scnfileitem{В общем случае реализация \textit{цифрового сотрудника } проще и дешевле в реализации и позволяет экспериментальным путем выявить наиболее больные места предприятия, где автоматизация информационных процессов и обеспечение их прозрачности позволит сэкономить наибольшее количество средств и минимизировать число ошибок.}
        \end{scnrelfromset}
        \begin{scnrelfromset}{недостатки}
            \scnfileitem{\textit{Цифровой сотрудник} не решает проблемы, связанные с дублированием информации, представленной в разной форме в разных местах, и только частично решает проблемы, связанные с ошибками при внесении новой или редактированием имеющейся информации в разнородные информационные ресурсы.}
            \scnfileitem{Из-за отсутствия унификации представления информации система \textit{IDR} ограничена в своих возможностях, в частности при верификации информации. Проверить корректность, непротиворечивость, полноту информации намного проще, если вся информация представлена в унифицированном виде (как по форме, так и по смыслу).}
        \end{scnrelfromset}
        \scnsuperset{полноценная комната данных}
        \scnidtf{полный цифровой двойник предприятия}
        \scntext{пояснение}{Реализация \textit{полноценной комнаты данных} предполагает полный перенос в базу знаний \textit{IDR} части или всей информации, хранящейся в электронном виде в информационных ресурсах предприятия. Для этого необходимо описывать в базе знаний как онтологии (системы понятий) предметной области предприятия, так и конкретные экземпляры и связи между ними. При этом очевидно, что если онтологии изменяются относительно редко и могут обновляться вручную, то конкретные экземпляры и их описание должно формироваться автоматически.}
        \begin{scnrelfromset}{достоинства}
            \scnfileitem{Полностью исключается необходимость дублирования информации в разных местах и в разных формах, таким образом снижается объем хранимой информации и минимизируется количество ошибок.}
            \scnfileitem{В онтологиях описывается только смысл информации, нет необходимости описывать существующую структуру информационных ресурсов предприятия и ориентироваться на нее.}
            \scnfileitem{Полностью задействуются возможности \textit{IDR} по верификации информации предприятия на предмет корректности, непротиворечивости и полноты.}
            \scnfileitem{Наращивание функциональных возможностей такой реализации значительно упрощается за счет гибкости подходов, лежащих в основе \textit{IDR}, в частности, подхода к разработке и структуризации базы знаний и многоагентного подхода к обработке информации.}
        \end{scnrelfromset}
        \begin{scnrelfromset}{недостатки}
            \scnfileitem{Требуются изменения в уже налаженных процессах на предприятии, отказ от части используемого в настоящий момент ПО, переобучение сотрудников, обслуживающих и развивающих информационные системы предприятия.}
            \scnfileitem{В общем случае \textit{полноценная комната данных} может оказаться более трудоемкой при первоначальной реализации, поскольку предполагает больший объем работ по формализации информации, обеспечению надежности ее хранения и доступа к ней, эффективности ее редактирования и дополнения.}
        \end{scnrelfromset}
        \scntext{примечание}{Оба варианта реализации интеллектуальной комнаты данных (\textit{полноценная комната данных} и \textit{цифровой сотрудник}) дают возможность описать в базе знаний \textit{IDR} не только информацию, которая уже активно используется на предприятии, но и дополнительные знания, которые могут оказаться полезными, например:
            \begin{scnitemize}
                \item Различные стандарты и документы, регламентирующие работу предприятия;
                \item Описание структуры предприятия, правил работы, основных контактов;
                \item Учебные материалы и руководства для новых сотрудников;
                \item и многое другое
            \end{scnitemize}}
        \bigskip
    \end{scnsubstruct}
    \scnendcurrentsectioncomment
\end{SCn}
