\begin{scnreltovector}{конкатенация сегментов}
    \scnitem{Структура деятельности в области Искусственного интеллекта}
    \scnitem{Текущее состояние и проблемы дальнейшего развития деятельности в области Искусственного интеллекта}
    \scnitem{Понятие Технологии OSTIS}
    \scnitem{Использование Технологии OSTIS для повышения качества человеческой деятельности в области Искусственного интеллекта}
    \scnitem{Понятие Экосистемы OSTIS}
\end{scnreltovector}
\begin{scnrelfromset}{рассматриваемые вопросы}
    \scnfileitem{Каковы основные стратегические цели (сверхзадачи) научно-технической деятельности в области \textit{Искусственного интеллекта}.}
    \scnfileitem{Какие проблемы являются на сегодняшний день актуальными для дальнейшего развития различных направлений \textit{Искусственного интеллекта} и для развития \textit{Искусственного интеллекта} в целом как общей (объединённой) \textit{научно-технической дисциплины}, а также для развития различных форм деятельности в этой области (научно-исследовательской деятельности создания технологий разработки интеллектуальных компьютерных систем, образовательной деятельности, бизнеса).}
    \scnfileitem{Какие проблемы являются на сегодняшний день актуальными для развития других \textit{научно-технических дисциплин} и являются ли эти проблемы аналогичными тем, которые актуальны для развития \textit{Искусственного интеллекта}.}
    \scnfileitem{Какие можно предложить подходы к решению указанных выше проблем и как для этого можно использовать создаваемый сейчас новый технологический уклад в области \textit{Искусственного интеллекта} (следующий уровень технологий искусственного интеллекта).}
    \scnfileitem{Как будет выглядеть на основе следующего уровня \textit{технологий Искусственного интеллекта} комплексная автоматизация всех \textit{видов человеческой деятельности}, а также взаимодействие различных \textit{видов человеческой деятельности}, т.е. как будет выглядеть архитектура \textit{smart-общества}.}
    \scnfileitem{Устраивает ли нас уровень семантической совместимости взаимопонимания между современными виртуальными компьютерными системами и что необходимо сделать для повышения этого уровня.}
    \scnfileitem{Устраивает ли нас уровень семантической совместимости взаимопонимания между современными интеллектуальными компьютерными системами их пользователями и что необходимо сделать для повышения этого уровня.}
\end{scnrelfromset}
\scntext{аннотация}{Предлагаемое вашему вниманию рассмотрение методологических проблем современного состояния работ в области \textit{Искусственного интеллекта} состоит из следующих частей:
    \begin{scnitemize}
        \item Анализ актуальных проблем, препятствующих дальнейшему развитию  \textit{Искусственного интеллекта} как \textit{научно-технической дисциплины}:
        \begin{scnitemizeii}
            \item Проблемы развития научных исследований в области \textit{Искусственного интеллекта}.
            \item Проблемы разработки технологий проектирования и реализации \textit{интеллектуальных компьютерных систем}.
            \item Проблемы формирования рынка \textit{интеллектуальных компьютерных систем}.
            \item Образовательные проблемы в области \textit{Искусственного интеллекта}.
            \item Проблемы развития бизнеса в области \textit{Искусственного интеллекта}.
        \end{scnitemizeii}
        \item Анализ проблем автоматизации сложных видов деятельности:
        \begin{scnitemizeii}
            \item научно-исследовательской деятельности в рамках различных научных дисциплин;
            \item создание \textit{технологий проектирования} и производства (реализации) сложных технических систем;
            \item \textit{инженерной деятельности} по разработке сложных технических систем;
            \item \textit{образовательной деятельности} по наукоёмким техническим специальностям.
        \end{scnitemizeii}
        \item Формулировка принципов, лежащих в основе \textit{Технологии OSTIS}, предназначенной для решения указанных выше проблем.
        \item Рассмотрение структуры \textit{Экосистемы OSTIS}, построенной по \textit{Технологии OSTIS} и обеспечивающей комплексную автоматизацию всех видов человеческой деятельности.
    \end{scnitemize}}
\begin{scnrelfromset}{используемые знаки общих понятий и иных сущностей}
    \scnitem{деятельность}
	    \begin{scnindent}
	    	\scnidtf{область деятельности}
	    	\scnsuperset{человеческая деятельность}
	    \end{scnindent}
    \scnitem{вид деятельности}
    \begin{scnindent}
        \scnhaselement{проектирование}
        \begin{scnindent}
            \scnidtf{проектная деятельность}
        \end{scnindent}
        \scnhaselement{производство}
        \begin{scnindent}
            \scnidtf{производственная деятельность}
        \end{scnindent}
        \scnhaselement{наука}
        \begin{scnindent}
            \scnidtf{научная деятельность}
        \end{scnindent}
    \end{scnindent}
    \scnitem{проект}
    \begin{scnindent}
        \scnsuperset{открытый проект}
    \end{scnindent}
    \scnitem{консорциум}
    \scnitem{технология}
    	\begin{scnindent}
        \scnsuperset{информационная технология}
        \begin{scnindent}
            \scnsuperset{технология искусственного интеллекта}
        \end{scnindent}
    \end{scnindent}
    \scnitem{кибернетическая система}
    \begin{scnindent}
        \scnsuperset{интеллектуальная система}
        \begin{scnindent}
            \scnsuperset{интеллектуальная компьютерная система}
            \begin{scnindent}
                \scnidtf{искусственная интеллектуальная система}
            \end{scnindent}
        \end{scnindent}
    \end{scnindent}
    \scnitem{конвергенция\scnsupergroupsign}
    \begin{scnindent}
        \scnidtf{уровень конвергенции (близости)}
        \scnsuperset{конвергенция кибернетических систем\scnsupergroupsign}
        %Ключевого знака в стандарте не было
        \begin{scnreltolist}{ключевой знак}
            \scnitem{\cite{Yankovskaya2017}}
            \scnitem{\cite{Palagin2013}}
            \scnitem{\cite{Yankovskaya2010}}
            \scnitem{\cite{Kovalchuk2011}}
        \end{scnreltolist}
    \end{scnindent}
    \scnitem{интеграция*}
    \begin{scnindent}
        \scnsuperset{интеграция кибернетических систем*}
        \scnsuperset{эклектичная интеграция*}
        \scnsuperset{глубокая интеграция*}
    \end{scnindent}
    \scnitem{интегрированная система}
    \begin{scnindent}
        \scnsuperset{эклектичная система}
        \scnsuperset{гибридная система}
    \end{scnindent}
    \scnitem{экосистема интеллектуальных компьютерных систем}
    \scnitem{рынок знаний}
    \begin{scnindent}
        \scnidtf{рыночная организация порождения эволюции и применения знаний}
    \end{scnindent}
    \scnitem{smart-общество}
    \begin{scnindent}
        \scnidtf{общество,в основе которого лежит экосистема интеллектуальных компьютерных систем и рынок знаний}
    \end{scnindent}
\end{scnrelfromset}
\begin{scnrelfromset}{ключевые знаки}
    \scnitem{Искусственный интеллект}
    \begin{scnindent}
        \scniselement{научно-техническая дисциплина}
        \begin{scnindent}
            \scnsubset{научно-техническая деятельность}
        \end{scnindent}
    \end{scnindent}
    \scnitem{интеллектуальная система}
    	\begin{scnindent}
        	\scnsuperset{интеллектуальная компьютерная система}
        \end{scnindent}
    \scnitem{Общая теория интеллектуальных систем}
    \scnitem{Базовая комплексная технология проектирования интеллектуальных компьютерных систем}
    \scnitem{Технология производства спроектированных интеллектуальных компьютерных систем}
    \scnitem{Специализированная инженерия в области Искусственного интеллекта}
    \scnitem{Образовательная деятельность в области Искусственного интеллекта}
    \scnitem{Бизнес-деятельность в области Искусственного интеллекта}

    \bigskip

    \scnitem{\scnkeyword{Технология OSTIS}}
    \scnitem{\scnkeyword{ostis-система}}
    \scnitem{смысловое преставление информации}
    \scnitem{агентно-ориентированная модель обработки информации в памяти}
    \scnitem{стандартизация ostis-систем}
    \scnitem{\scnkeyword{SC-код}}
    \scnitem{абстрактная sc-машина}
    \scnitem{конвергенция знаний в памяти}
    \scnitem{ostis-систем}
    \scnitem{конвергенция моделей решения задач в  ostis-системе}
    \scnitem{интеграция знаний в памяти  ostis-системы}
    \scnitem{интеграция моделей решения задач в  ostis-системе}
    \scnitem{ostis-сообщество}
    \scnitem{ostis-технология}
    \begin{scnindent}
        \scnsuperset{ostis-технология проектирования}
        \scnsuperset{ostis-технология производства}
        \scnsuperset{технология эксплуатации ostis-систем}
        \scnsuperset{технология реинжиниринга ostis-систем} 
    \end{scnindent}
    \scnitem{\scnkeyword{Ядро Технологии OSTIS}}  

    \bigskip

    \scnitem{OSTIS-портал научных знаний в области Искусственного интеллекта}
    \scnitem{Проект IMS.ostis}
    \scnitem{\scnkeyword{Метасистема IMS.ostis}}
    \scnitem{Проект Программной реализации универсальной абстрактной sc-машины}
    \scnitem{Проект разработки Универсального sc-компьютера}
    \scnitem{Специализированная инженерия, осуществляемая на основе Технологии OSTIS}
    \scnitem{Образовательная деятельность в области Искусственного интеллекта, осуществляемая на основе технологии OSTIS}
    \scnitem{\scnkeyword{Консорциум OSTIS}}

    \bigskip

    \scnitem{\scnkeyword{Экосистема OSTIS}}
    \begin{scnindent}
        \scnidtf{Симбиоз семантически совместимых и координирующих свою деятельность \textit{ostis-систем} и людей, направленный на существенное, качественное повышение уровня автоматизации всех \textit{видов человеческой деятельности}.}
        \scntext{примечание}{Семантически совместимая (понятийно согласованная) формализация всех(!) видов человеческой деятельности является органической частью \textit{Технологии OSTIS}(!). То есть формализуемые отраслевые стандарты \textit{всех видов человеческой деятельности} должны строго наследовать свойства всей системы базовых понятий и знаний, лежащих в основе Технологии OSTIS. Таким образом речь идет о строгой формальной модели \textit{Экосистемы OSTIS} как единого целого и здесь есть место всем приложениям, но приведенным в комплексную систему. Если к построению такой комплексной формальной модели \textit{всех видов человеческой деятельности} подходить системно, то все не так страшно, так как многие модели можно и нужно строить на основе аналогий, стратификации, наследования и свойств. Это придаст существенную динамику эволюции этих формальных моделей.\\
            К сожалению, современная наука психологически ориентирована на поиск отличий, на выявление принципиальной (научной) новизны своих результатов (что является необходимым фактором оценки этих решений). В этом ничего плохого нет, но для решения сиситемых проблем (в частности, для вывода \textit{Искусственного интеллекта} из кризисного состояния) необходимо существенно активизировать поиск сходств, аналогий, реализацию конвергентных процессов по построению комплексных интегрированных моделей. Это не менее значимые научные результаты, чем выявление принципиально новых свойств и закономерностей.}
    \end{scnindent}
    \scnitem{человеческая деятельность}
    \scnitem{вид человеческой деятельности}
    \scnitem{автоматизация человеческой деятельности}
    \scnitem{качество человеческой деятельности}
    \scnitem{субъект Экосистемы OSTIS}
    \scnitem{Рынок знаний, реализованный в рамках Экосистемы OSTIS}
    \scnitem{smart-общество}
\end{scnrelfromset}

\scntext{предисловие}{Анализ современного состояния работы в области \textit{Искусственного интеллекта} показывает то, что указанная область \textit{человеческой деятельности} находится в глубоком фундаментальным методологическом и трудновидимом кризисе. Поэтому основными целями данного раздела являются:
    \begin{scnitemize}
        \item выявление основных причин возникновения указанного кризиса;
        \item уточнение основных мер, направленных на устранение этого кризиса.
    \end{scnitemize}}
\scntext{основная цель}{Сформировать мотивацию и инфраструктуру для создания эволюции информационных технологий принципиально нового уровня, в основе которого лежат семантические совместимые \textit{интеллектуальные компьютерные системы}, способные согласовывать свои действия в заранее непредсказуемых обстоятельствах.}
\begin{scnindent}
    \scntext{примечание}{Сейчас актуально не столько обсуждать различные вопросы \textit{Искусственного интеллекта}, а обсуждать \uline{проблемы} и пути решения этих проблем. Нельзя делать вид, что всё хорошо.}
\end{scnindent}
	
\scnheader{Искусственный интеллект}
\scntext{примечание}{Современное кризисное состояние \textit{Искусственного интеллекта} вполне логично --- это естественный этап эволюции любых сложных систем и технологий:
    \begin{scnitemize}
        \item Сначала накопление большого количества конкретных решений;
        \item Анализ полученного многообразия и превращение его в стройную систему качественно более высокого уровня.
    \end{scnitemize}
    Кризисов не надо бояться --- их надо преодолевать. Диалектику и, в частности, переход количества в качество ещё никто не отменял.}
    
\scnheader{Современное состояние технологии Искусственного интеллекта}
\scntext{пояснение}{К настоящему моменту мы научились разрабатывать \textit{интеллектуальные компьютерные системы} самого различного назначения. Но для повышения уровня автоматизации всё более и более широких видов человеческой деятельности необходим \uline{качественный} переход к разработке не отдельных \textit{интеллектуальных компьютерных систем}, а целых комплексов самостоятельно взаимодействующих между собой \textit{интеллектуальных компьютерных систем}.Это требует фундаментального переосмысления теории технологии проектирования \textit{интеллектуальных компьютерных систем}. Эффективный переход количества в новое качество требует серьезных усилий.}
\scntext{эпиграф}{Необходим переход от зоопарка локальных идей, сервисов и информационных ресурсов к их системе.}
\scntext{эпиграф}{From data science to knowledge science.}
\scntext{эпиграф}{Одно дело --- создавать локальные шедевры и совсем другое дело --- двигаться ко всеобщей гармонии.}

\scnheader{Современное состояние информационных технологии}
\scntext{примечание}{Экспертам в процессе обсуждения инновационных вопросов приходится тратить много времени на формирование нового \uline{понятийного аппарата}. <...> Мировому сообществу есть смысл задуматься над созданием нового искусственного языка, <...> чтобы с учётом возможностей современного мира сформулировать новую среду экспертного общения.}
\scnrelfrom{автор}{Курбацкий А.Н.}

\scnheader{будущие технологии Искусственного интеллекта}
\scntext{примечание}{\uline{\textit{Смысл}} той \textit{информации}, которой оперирует \uline{каждая}(!) \textit{интеллектуальная компьютерная система}, а также \uline{\textit{смысл}} того, что она делает (какие \textit{задачи} она решает) должен быть четко формализован и является частью её \textit{базы знаний}. Формализация этого \textit{смысла} (в состав которой входит экспертное согласование системы используемых \textit{понятий}) представляет собой первый этап проектирования каждой \textit{интеллектуальной компьютерной системы}, обеспечивающий \textit{семантическую совместимость} (взаимопонимание) \textit{интеллектуальных компьютерных систем} и эффективное их взаимодействие (самостоятельную организацию коллективной деятельности).Таким образом, необходим \textit{\uline{универсальный} формальный язык}, который используется как экспертами, разработчиками, так и непосредственно самими \textit{интеллектуальными компьютерными системами}.}

\scnheader{Технология OSTIS}
\begin{scnrelfromset}{теоретический компонент}
    \scnfileitem{Комплексная семантическая теория интеллектуальных компьютерных систем (ostis-систем).}
    \scnfileitem{Комплексная семантическая теория человеческой деятельности как Экосистемы (симбиоза) с иерархическим комплексом интеллектуальных компьютерных систем.}
    \scnfileitem{Теория перманентной эволюции (реинжиниринга) указанной Экосистемы (с минимизацией этапов локальной приостановки).}
    \begin{scnindent}
        \scntext{примечание}{Cтандарты должны меняться быстро, а Экосистема должна быстро приводиться в соответствие с новыми стандартами.}
    \end{scnindent}
\end{scnrelfromset}

\scnheader{Подготовка специалистов в области Искусственного интеллекта}
\scntext{примечание}{Массовая подготовка высококвалифицированных \textit{специалистов в области Искусственного интеллекта}, способных преодолеть современное кризисное состояние \textit{Искусственного интеллекта}, фактически и является самым главным фактом преодоления указанного кризиса.\\
    Необходимым условием и эпицентром вывода \textit{Искусственного интеллекта} из кризисного состояния и повышения темпов эволюции технологий \textit{Искусственного интеллекта} является организация \textit{подготовки специалистов в области Искусственного интеллекта} на основе активного привлечения студентов, магистрантов и аспирантов к \uline{перманентному} процессу эволюции \textit{Технологии OSTIS}.\\
    Очевидно, этому должно способствовать объединение соответствующей учебно-методической базы для разных кафедр, осуществляющих такую подготовку.\\
    На современном этапе развития \textit{Искусственного интеллекта} требуется не просто подготовка специалистов в этой области --- а подготовка специалистов \uline{принципиально новой формации}, способных:
    \begin{scnitemize}
        \item рассматривать область \textit{Искусственного интеллекта} не просто как многообразие \textit{интеллектуальных компьютерных систем}, а как постоянно эволюционируемую \uline{\textit{Экосистему}} таких систем;
        \item эффективно участвовать в решении как фундаментальных, системных, технологических проблем, так и практических, прикладных проблем эволюции указанной \textit{Экосистемы}.
    \end{scnitemize}
    Все это требует существенного переосмысления организации учебного процесса и учебно-методического обеспечения.\\
    <<Часто, например, совершенствование программных систем сводится к программным заплаткам. Через какое-то время мы имеем программу со множеством заплаток, как правило уже громоздкую и малоэффективную. В итоге --- иногда её проще выбросить и создать новую.>> (\cite{Kurbatski})\\
    Современная разработка каждой сложной программной системы требует построения \uline{качественной} формальной (цифровой) модели объекта управления, объекта автоматизации, причем \textit{семантически совместимой} с соответствующими моделями в смежных системах.\\
    Здесь важна общая математическая культура и унификация такой формализации.\\
    В настоящее время методологический подход к инженерной деятельности при разработке компьютерных систем часто выглядит следующим образом: <<Поставьте мне четкую инженерную задачу и я ее выполню. Но ответственность за ее постановку я с себя снимаю и не хочу учитывать критерии качества постановки задачи на более высоком уровне>>.\\
    Для наукоемких проектов, реализуемых в рамках развивающихся технологий, это недопустимо.\\
    Каждый инженер должен \uline{понимать}, что он делает и каковы истинные более глубокие критерии качества его результата.\\
    Нужна принципиально новая психологическая установка.\\
    Необходимо учитывать не только желание заказчика, но и общие принципы и стандарты разрабатываемых \textit{интеллектуальных компьютерных систем}.\\
    В основе организации образовательной деятельности на современном этапе развития \textit{Искусственного интеллекта} лежит:
    \begin{scnitemize}
        \item четкое формальное описание того, чему мы учим (каким знаниям и навыкам) --- в нашем случае это описание текущей версии \textit{Стандарта OSTIS} и направлений эволюции этого стандарта;
        \item уточнение того, что должен делать студент, магистрант и любой специалист для быстрого и качественного приобретения этих знаний и навыков.
    \end{scnitemize}
    Нужна \uline{комплексная} учебная программа по специальности \textit{Искусственный интеллект}, а не мозаика отдельных учебных дисциплин. И, соответственно этому необходимо \uline{комплексное} учебно-методическое пособие, достаточно полно отражающее текущее состояние теории и технологии проектирования \textit{интеллектуальных компьютерных систем}.\\
    Использование проектного метода при подготовке специалистов в области Искусственного интеллекта предполагает составление систематизированного сборника упражнений и задач, в частности, направленных на эволюцию Технологии OSTIS и посильных для студентов специальности \textit{Искусственный интеллект}:
    \begin{scnitemize}
        \item представление конкретных фрагментов различных предметных областей и онтологий;
        \item представление конкретных специфицированных методов (пополнение библиотек используемых методов из разных предметных областей, например, из теории графов);
        \item спецификация библиографических источников (в контексте \textit{Базы знаний IMS.ostis});
        \item выявление синонимии, омонимии, противоречий;
        \item сравнительный анализ и обзор близких внешних публикаций.
    \end{scnitemize}
    Таким образом, фронт самостоятельных, весьма полезных и посильных для студентов работ весьма широк. Главное сформировать у студентов профессиональный интерес, познавательную активность, инициативность и самостоятельность.}
\scntext{проектный метод}{Для того, чтобы научиться разрабатывать \textit{интеллектуальные компьютерные системы}, необходимо приобрести достаточно большой опыт участия и \uline{завершения} разработки реально востребованных \textit{интеллектуальных компьютерных систем}.}
\scntext{проектный метод}{Для того, чтобы научиться разрабатывать и совершенствовать \textit{технологии искусственного интеллекта}, необходимо приобрести достаточно большой опыт успешного (!) участия в создании различных компонентов комплексной \textit{технологии Искусственного интеллекта}.}
\scntext{пояснение}{Итак, современного специалиста в области \textit{Искусственного интеллекта} необходимо учить:
    \begin{scnitemize}
        \item не только тому, как следует разрабатывать \textit{интеллектуальные компьютерные системы} с помощью имеющихся (существующих) \textit{методов} и \textit{средств}, т.е. с помощью имеющихся \textit{технологий},
        \item но и тому, как надо развивать (совершенствовать) имеющиеся \textit{технологии}.
    \end{scnitemize}
    \textit{Технология OSTIS} рассматривается нами не столько, как предлагаемая технология разработки \textit{интеллектуальных компьютерных систем}, а как предлагаемые \uline{принципы} построения технологии разработки \textit{интеллектуальных компьютерных систем} следующего поколения. Т.е. фактически мы предлагаем не саму технологию (\textit{Технологию OSTIS}), а участие в её создании и развитии, которое может привести даже к радикальным изменениям текущего состояния (текущей версии) этой \textit{технологии}. Это психологически снимает ощущение навязывания предлагаемой технологии и заменяет его на атмосферу партнерства, направленного на перманентную эволюцию указанной технологии. Такой подход создаст также условия для существенного повышения качества \textit{подготовки специалистов в области Искусственного интеллекта}, поскольку дает возможность осуществлять обучение путём непосредственного вовлечения студентов и магистрантов в реальные, практически значимые процессы разработки \textit{интеллектуальных компьютерных систем}, а также в процессы совершенствования (эволюции) соответствующих \textit{технологий}.}
\scnheader{Анализ методологических проблем современного состояния работ в области Искусственного интеллекта}
\begin{scnrelfromvector}{примечания}
    \scnfileitem{Самые тяжелые кризисные ситуации в научно-технической сфере --- это те, которые носят фундаментальный и не совсем очевидный характер. Развитие кибернетики, информатики и искусственного интеллекта подтверждает это. За впечатляющими практическими и теоретическими достижениями незаметно возрастает огромный вал накладных расходов при разработке сложных больших систем --- возрастает дублирование, нестыковки, несогласованности.}
    \scnfileitem{Нет ничего более грустного, чем созерцать активную творческую деятельность большого количества умных людей, которые по инерции, не отдавая себе отчета, накапливают проблемы, препятствующие дальнейшему качественному развитию этой деятельности. Вместо того, чтобы разгребать эти проблемы на благо всем.}
    \scnfileitem{Как только мы начнем серьезно относиться к \uline{формальному} уточнению и согласованию всего многообразия понятий, используемых в области Искусственного интеллекта и различных его приложений, как только мы начнем \uline{реальную}(!) \uline{совместную} работу по общей комплексной формальной теории интеллектуальных компьютерных систем и по созданию комплексной технологии их проектирования, многие современные проблемы \textit{Искусственного интеллекта} начнут решаться. Нет ничего практичнее хорошей теории.}
    \scnfileitem{Основной лейтмотив развития технологий \textit{Искусственного интеллекта} --- это не только создание компьютерной технологии разработки сей совместной \textit{интеллектуальной компьютерной системы}, но и создание \uline{Метатехнологии} перманентной \uline{эволюции}(!) такой технологии. Иначе --- эклектика, усугубляющая современный кризис. Для создания эффективно и самостоятельно взаимодействующих \textit{интеллектуальных компьютерных систем} несущественных мелочей не бывает --- дьявол кроется в деталях и тонкостях. Важен не столько инжиниринг, сколько реинжиниринг интеллектуальных компьютерных систем и человеческой деятельности в целом.}
    \scnfileitem{Решение рассматриваемых кризисных проблем требует:\\
	        \begin{scnitemize}
	            \item Существенного фундаментального общесистемного переосмысления всего того, что мы творим.
	            \item Осознания того, что кибернетика, информатика и искусственный интеллект --- это общая фундаментальная наука, требующая единого серьезного математического аппарата.
	            \item Осознания того, что сейчас требуется не расширяемость многообразия точек зрения, а учиться их согласовывать, совершенствуя соответствующие методы.
	        \end{scnitemize}}
    \scnfileitem{Нам необходимо переходить от автоматизации отдельных видов \textit{человеческой деятельности} к интегрированной автоматизации всего комплекса человеческой деятельности, к созданию и постоянной эволюции всей общечеловеческой \textit{экосистемы}, состоящей из самостоятельно взаимодействующих \textit{интеллектуальных компьютерных систем} как между собой, так и между людьми, автоматизацию деятельности которой они осуществляют. При этом надо помнить, что основные накладные расходы, основные проблемы, возникают на стыках при интеграции различных технических решений. Разработчик каждой подсистемы должен гарантировать отсутствие указанных накладных расходов.}
    \scnfileitem{Самое главное --- надо ориентироваться не на создание идеальной информационной \textit{экосистемы}, а на создание эффективной технологии, направленной на перманентную эволюцию(!) указанной экосистемы.}
    \scnfileitem{Уникальность современного кризиса в области кибернетики, информатики и искусственного интеллекта заключается в том, что, несмотря на глобальность этого кризиса, абсолютно реально создать локальный эпицентр по разрешению этого кризиса --- в частности, в Республике Беларусь. Для этого есть все предпосылки --- специальность \textit{Искусственный интеллект}, опыт разработки компьютерных систем, достаточный научный уровень.}
\end{scnrelfromvector}
