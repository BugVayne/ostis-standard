\begin{SCn}
\scnsectionheader{Общие направления развития Cтандарта OSTIS}
\begin{scnstruct}
    \scnheader{Направления развития Cтандарта OSTIS}

    \begin{scneqtoset}
        \scnfileitem{Все sc-идентификаторы, входящие в состав scg-текстов,
            scn-текстов и различных иллюстраций должны иметь одинаковый
            шрифт и размер. За
            исключением, возможно, размера sc-идентификаторов в
            scg-текстах. (См.,
            например, страницы 443-448 стандарта-2021)}
        \scnfileitem{Уточнить Алфавит SCg-кода}
    \end{scneqtoset}

    \scnheader{Cтандарт OSTIS}
    \begin{scnrelfromset}{направления перманентного развития}
        \scnfileitem{В рамках титульных спецификаций разделов Стандарта OSTIS
            постоянно уточнять и детализировать семантические связи каждого
            раздела с
            другими разделами}
        \scnfileitem{Каждая новая именуемая (идентифицируемая) сущность,
            описываемая в Стандарте OSTIS (в первую очередь, каждое
            понятие), должна быть
            подробно специфицирована в соответствующем разделе}
    \end{scnrelfromset}

    \begin{scnrelfromset}{направления текущего этапа развития}
        \scnfileitem{Доработать в текущую версию Стандарта OSTIS и ввести:
            \begin{scnitemize}
                \item Оглавление
                \item Титульную спецификацию Стандарта OSTIS
            \end{scnitemize}}
        \scnfileitem{Привести текущую версию Стандарта OSTIS в соответствие с
            новой версией оглавления Стандарта OSTIS}
        \scnfileitem{Включить расширенный материал статьи В.В. Голенкова и
            соавторов в Springer-2021 в текущую версию Стандарта OSTIS}
        \scnfileitem{Включить в Стандарт OSTIS все методические рекомендации
            по развитию Стандарта OSTIS (правила построения различных
            фрагментов, правила
            идентификации sc-элементов, правила спецификации sc-элементов
            --- точнее
            обозначение сущностей, направления развития)}
    \end{scnrelfromset}

    \scnheader{Cтандарт OSTIS --- Общий орг-план}
    \begin{scnrelfromset}{направления развития}
        \scnfileitem{Доработать правила оформления Стандарта OSTIS:
            \begin{scnitemize}
                \item Правила структуризации, типология компонентов,
                семантические связи
                между компонентами
                \item Правила идентификации sc-элементов (именования
                различных
                классифицированных сущностей)
                \item Правила спецификации различных классов sc-элементов
            \end{scnitemize}}
        \scnfileitem{Существенно расширить библиографию Стандарта OSTIS и
            библиографии всех разделов. \scnqqi{Преобразовать простой список
            библиографических
            источников в библиографическую спецификацию соответствующего
            текста}}
        \scnfileitem{Чётко сформировать общий план доработки всего текста
            Стандарта OSTIS, а также конкретные планы доработки каждого
            раздела}
        \scnfileitem{Содержание всех работ, опубликованных авторами Стандарта
            OSTIS по Технологии OSTIS должны быть формализованы и включены
            в
            соответствующие разделы Стандарта OSTIS (имеются в виду отчёты
            по лабораторным
            работам студентов, расчётные работы, курсовые проекты,дипломные
            проекты,
            диссертации, статьи в материалах конференций OSTIS, в сборниках
            Springer, в
            журнале Онтология проектирования и других изданиях). Более
            того, если первичной
            публикацией новых материалов будет их включение в состав
            Стандарта OSTIS, то
            это существенно повысит результативность работ по развитию
            Стандарта OSTIS}
        \scnfileitem{Включить в текст монографии все статьи Голенкова В.В. на
            конференциях OSTIS и другие публикации (в том числе книги)
            \begin{scnitemize}
                \item Это актуально для работы над изданием Стандарта
                OSTIS(в эту версию
                Стандарта OSTIS  надо собрать абсолютно всё, что нами
                сделано)
            \end{scnitemize}}
        \scnfileitem{Увеличить число ссылок на библиографические источники из
            текста Стандарта OSTIS}
    \end{scnrelfromset}

    \scnheader{Cтандарт OSTIS}
    \begin{scnrelfromset}{направления развития}
        \scnfileitem{Некоторые материалы текущего состояния ряда разделов
            целесообразно перенести в дочерние разделы (если имеющаяся
            детализация этих
            материалов более уместна для дочерних разделов).Это, например,
            касается
            некоторых сегментов раздела \scnqqi{\nameref{intro_ostis}} (в
            частности сегмента об
            Экосистеме OSTIS)}
        \scnfileitem{Дополнить этот список разделов}
        \scnfileitem{Совершенствовать стратификацию}
    \end{scnrelfromset}

    \scnheader{Правила организации развития исходного текста Стандарта OSTIS}
    \scnidtf{Правила организации коллективной деятельности по развитию
        исходного
        текста Стандарта OSTIS}
    \scnheader{Cтандарт OSTIS}
    \begin{scnrelfromset}{правила построения}
        \scnfileitem{Каждому разделу приписать одного ответственного редактора
            и возможно несколько соавторов (ответственный редактор является
            единственным
            автором)}
        \scnfileitem{LaTeX + макросы}
        \scnfileitem{GitHub --- структуризация файлов ostis-ai}
        \scnfileitem{Предложения/рецензирования/включение}
        \scnfileitem{Рецензирование}
        \scnfileitem{Извлечение и конвертирование в pdf-файл любого раздела
            или группы разделов}
        \scnfileitem{Просмотр pdf-файла}
        \scnfileitem{Конвертирование в scs}
        \scnfileitem{Загрузка в sc-память}
        \scnfileitem{Просмотр базы знаний}
    \end{scnrelfromset}
\end{scnstruct}
\end{SCn}