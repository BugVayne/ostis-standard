\begin{SCn}
    \scnsectionheader{Предметная область и онтология внутреннего языка ostis-систем}
        \begin{scnsubstruct}
            \begin{scnreltovector}{конкатенация сегментов}
                \scnitem{Основные положения внутреннего языка ostis-систем}
                \scnitem{Описание Ядра SC-кода}
                \scnitem{SC-код как синтаксическое расширение Ядра SC-кода}
                \scnitem{Использование SC-кода для формального описания собственного синтаксиса}
            \end{scnreltovector}
            
            \scnsegmentheader{Основные положения внутреннего языка ostis-систем}
            \begin{scnsubstruct}
    
                \scnheader{SC-код}
    
                \scnidtf{Язык унифицированного смыслового представления знаний в памяти \textit{интеллектуальных компьютерных систем}}
                \scnidtf{Внутренний язык \textit{ostis-систем}}
                \scnrelto{внутренний язык}{ostis-система}
                \scntext{эпиграф}{Информация содержится не в самих знаках, а в конфигурации связей между ними.}
                \scntext{эпиграф}{Он вскочил на коня и поскакал во все стороны.}
                \scntext{основной внешний идентификатор sc-элемента}{\textbf{SC-код}}
                \scniselement{имя собственное}
                \scntext{часто используемый неосновной внешний идентификатор sc-элемента}{sc-текст}
                \scniselement{имя нарицательное}
                \scniselement{абстрактный язык}
                \begin{scnindent}
                    \scnidtf{Язык, для которого не уточняется способ представления символов (синтаксически элементарных фрагментов), входящих в состав текстов этого языка, а задается только \uline{алфавит*} этих символов, то есть семейство классов символов, считающихся синтаксически эквивалентными друг другу.}
                    \begin{scnindent}
                        \scntext{примечание}{Каждому абстрактному языку можно поставить в соответствие целое семейство \textit{реальных языков}, обеспечивающих \uline{изоморфное} реальное представление текстов указанного абстрактного языка путем уточнения способов представления (изображения, кодирования) символов, входящих в состав этих текстов, а также путем уточнения правил установления синтаксической эквивалентности этих символов. Очевидно, что во всём остальном синтаксис и денотационная семантика указанных реальных языков полностью совпадает с синтаксисом и денотационной семантикой соответствующего абстрактного языка.}
                        \begin{scnindent}
                            \scntext{примечание}{Для \textit{SC-кода} как абстрактного языка необходима разработка как минимум трех синтаксически и семантически эквивалентных ему реальных языков: (1) язык кодирования текстов \textit{SC-кода} в памяти традиционных компьютеров; (2) язык кодирования текстов \textit{SC-кода} в семантической ассоциативной памяти; (3) \textit{Ядро SCg-кода} --- язык, синтаксически и семантически эквивалентный \textit{SC-коду} и обеспечивающий графическое представление текстов \textit{SC-кода}.}
                        \end{scnindent}
                    \end{scnindent}
                \end{scnindent}
                \scniselement{графовый язык}
                \begin{scnindent}
                    \scntext{пояснение}{язык, каждый текст которого 
                        \begin{scnitemize}
                            \item задается множеством входящих в него элементарных фрагментов (символов), которое, в свою очередь, состоит
                            \begin{scnitemizeii}
                                \item из множества узлов (вершин), возможно, синтаксически разного вида
                                \item из множества связок, которые также могут принадлежать разным синтаксически выделяемым классам
                            \end{scnitemizeii}
                            \item задается в общем случае несколькими отношениями инцидентности связок с компонентами этих связок (при этом указанными компонентами в общем случае могут быть не только вершины, но и другие связки)
                        \end{scnitemize}}
                \end{scnindent}
                \scnidtf{Универсальный язык, обеспечивающий внутреннее представление и структуризацию \uline{всех}(!), используемых ostis-системой в процессе своего функционирования.}
                \scnidtf{Универсальный язык, являющийся результатом унификации (уточнения) синтаксиса и денотационной семантики семантических сетей.}
                \scntext{пояснение}{Универсальность SC-кода обеспечивается и тем, что элементами текстов SC-кода могут быть знаки описываемых сущностей \uline{любого} вида, в том числе, и  знаки связей между описываемыми сущностями и/или их знаками}\scntext{следствие}{Тексты SC-кода являются графовыми структурами расширенного вида, в которых знаки описываемых связей могут соединять не только вершины (узлы) графовой структуры, но и знаки других связей.}
                \scnidtf{Базовый универсальный язык внутреннего представления знаний в памяти ostis-систем.}
                \scnidtf{Базовый внутренний язык ostis-систем.}
                \scnidtf{Максимальный внутренний язык ostis-систем, по отношению к которому все остальные (специализированные) внутренние языки являются его подъязыками (подмножествами)}
                \scnidtf{Множество всевозможных текстов SC-кода}
                \scniselement{имя собственное}
                \scnidtf{текст SC-кода}
                \scniselement{имя нарицательное}
                \begin{scnrelfromvector}{принципы, лежащие в основе}
                    \scnitem{\textit{Знаки} (обозначения) всех \textit{сущностей}, описываемых в \textit{sc-текстах} (текстах \textit{\textbf{SC-кода}}) представляются в виде синтаксически элементарных (атомарных) фрагментов \textit{sc-текстов} и, следовательно, не имеющих внутренней структуры, не состоящих из более простых фрагментов \textit{текста}, как, например, имена (термины), которые представляют \textit{знаки} описываемых \textit{сущностей} в привычных \textit{языках} и состоят из \textit{букв}.}
                    \scnitem{\textit{Имена} (термины), \textit{естественно-языковые тексты} и другие информационные конструкции, не являющиеся \textit{sc-текстами}, могут входить в состав \textit{sc-текста}, но только как \textit{файлы}, описываемые (специфицируемые) \textit{sc-текстами}. Таким образом, в состав базы знаний \textit{интеллектуальной компьютерной системы}, построенной на основе \textit{\textbf{SC-кода}}, могут входить \textit{имена} (термины), обозначающие некоторые описываемые \textit{сущности} и представленные соответствующими \textit{файлами}. Каждый \mbox{\textit{sc-элемент}} будем называть внутренним обозначением некоторой \textit{сущности}, а \textit{имя} этой \textit{сущности}, представленное соответствующим файлом, будем называть \textit{внешним идентификатором} (внешним обозначением) этой \textit{сущности}. При этом каждый именуемый (идентифицируемый) \mbox{\textit{sc-элемент}} связывается дугой, принадлежащей отношению \scnqqi{быть \textit{\textbf{внешним идентификатором*}}}, с \textit{узлом}, содержимым которого является \textit{файл} идентификатора (в частности, \textit{имени}), обозначающего ту же \textit{сущность}, что и указанный выше \textit{sc-элемент}. \textit{Внешним идентификатором} может быть не только \textit{имя} (термин), но и иероглиф, пиктограмма, озвученное имя, жест. Особо отметим, что \textit{внешние идентификаторы} описываемых \textit{сущностей} в \textit{интеллектуальной компьютерной системе}, построенной на основе \textit{\textbf{SC-кода}}, используются только (1) для анализа информации, поступающей в эту систему из вне из различных источников, и ввода (понимания и погружения) этой информации в \textit{базу знаний}, а также (2) для синтеза различных \textit{сообщений}, адресуемых различным субъектам (в т.ч. пользователям).}
                    \scnitem{Тексты \textit{\textbf{SC-кода}} (\textit{sc-тексты}) имеют в общем случае нелинейную (графовую) структуру, поскольку (1) \textit{знак} каждой описываемой сущности входит в состав \textit{sc-текста} однократно и (2) каждый такой \textit{знак} может быть инцидентен неограниченному числу других \textit{знаков}, поскольку каждая описываемая \textit{сущность} может быть связана неограниченным числом связей с другими описываемыми \textit{сущностями}.}
                    \scnitem{\textit{База знаний}, представленная текстом \textit{\textbf{SC-кода}}, является \textit{графовой структурой} специального вида, алфавит элементов которой включает в себя множество \textit{узлов}, множество \textit{ребер}, множество \textit{дуг}, множество \textit{базовых дуг} --- дуг специально выделенного типа, обеспечивающих структуризацию \textit{баз знаний}, а также множество специальных \textit{узлов}, каждый из которых имеет содержимое, являющееся \textit{файлом}, хранящимся в памяти \textit{интеллектуальной компьютерной системы}. Структурная особенность данной \textit{графовой структуры} заключается в том, что ее \textit{дуги} и \textit{ребра} могут связывать не только \textit{узел} с \textit{узлом}, но и \textit{узел} с \textit{ребром} или \textit{дугой}, \textit{ребро} или \textit{дугу} с другим \textit{ребром} или \textit{дугой}.}
                    \scnitem{\uline{Все элементы} (\textit{sc-элементы}) указанной выше \textit{графовой структуры} (текста \textit{\textbf{SC-кода}}), т.е. все ее узлы (\textit{sc-узлы}), ребра (\textit{sc-ребра}) и дуги (\textit{sc-дуги}) являются обозначениями различных сущностей. При этом ребро является обозначением бинарной неориентированной связки между двумя сущностями, каждая из которых либо представлена в рассматриваемой графовой структуре соответствующим знаком, либо является самим этим знаком. Дуга является обозначением бинарной ориентированной связки между двумя сущностями. Дуга специального вида (\textit{\textbf{базовая дуга}}) является знаком связи между узлом, обозначающим некоторое множество элементов рассматриваемой графовой структуры, и одним из элементов этой графовой структуры, который принадлежит указанному множеству. Узел, имеющий содержимое (узел, для которого содержимое существует, но может в текущий момент быть неизвестным) является знаком файла, который является содержимым этого узла. Узел, не являющийся знаком файла, может обозначать какой-либо материальный объект, первичный абстрактный объект(например, число, точку в некотором абстрактном пространстве), какую-либо бинарную связь, какое-либо множество (в частности, понятие, структуру, ситуацию, событие, процесс). При этом сущности, обозначаемые элементами рассматриваемой графовой структуры, могут быть постоянными (существующими всегда) и временными (сущностями, которым соответствует отрезок времени их существования). Кроме того, сущности, обозначаемые элементами рассматриваемой графовой структуры, могут быть константными (конкретными) сущностями и переменными (произвольными) сущностями. Каждому элементу рассматриваемой графовой структуры, являющемуся обозначением переменной сущности, ставится в соответствие область возможных значений этого обозначения. Область возможных значений каждого переменного ребра является подмножеством множества всевозможных константных ребер, область возможных значений каждой переменной дуги является подмножеством множества всевозможных константных дуг, область возможных значений каждого переменного узла является подмножеством множества всевозможных константных узлов.}
                    \scnitem{В рассматриваемой графовой структуре, являющейся представлением базы знаний в \textit{\textbf{SC-коде}}, могут, но не должны существовать разные элементы графовой структуры, обозначающие одну и ту же сущность. Если пара таких элементов обнаруживается, то эти элементы склеиваются (отождествляются). Таким образом, синонимия внутренних обозначений в базе знаний интеллектуальной компьютерной системы, построенной на основе \textit{\textbf{SC-кода}}, запрещена. При этом синонимия внешних обозначений считается нормальным явлением. Формально это означает, что из некоторых элементов рассматриваемой графовой структуры выходит несколько дуг, принадлежащих отношению \scnqqi{быть \textit{\textbf{внешним идентификатором*}}}. Из всех указанных дуг, принадлежащих отношению \scnqqi{быть \textit{\textbf{внешним идентификатором*}}} и выходящих из одного элемента рассматриваемой графовой структуры, обязательно выделяется одна (очень редко две) путем включения их в число дуг, принадлежащих отношению \scnqqi{быть \textit{\textbf{основным внешним идентификатором*}}}. Это означает, что указываемый таким образом внешний идентификатор не является омонимичным, т.е. не может быть использован как внешний идентификатор, соответствующий другому элементу рассматриваемой графовой структуры.}
                    \scnitem{Кроме файлов, представляющих различные внешние обозначения (имена, иероглифы, пиктограммы), в памяти интеллектуальной компьютерной системе, построенной на основе \textit{\textbf{SC-кода}}, могут хранится файлы различных текстов (книг, статей, документов, примечаний, комментариев, пояснений, чертежей, рисунков, схем, фотографий, видео-материалов, аудио-материалов).}
                    \scnitem{\uline{Любую сущность}, требующую описания, в тексте \textit{\textbf{SC-кода}} можно обозначить в виде \textit{sc-элемента}. Это являетс яодним из факторов, обеспечивающих универсальность \textit{\textbf{SC-кода}}. Особо подчеркнем, что sc-элементы являются не просто обозначениями различных описываемых сущностей, а обозначениями, которые являются элементарными (атомарными) фрагментами знаковой конструкции, т.е. фрагментами, детализация структуры которых не требуется для \scnqq{прочтения} и понимания этой знаковой конструкции.}
                    \scnitem{Текст \textit{\textbf{SC-кода}}, как и любая другая графовой структура, является абстрактным математическим объектом, не требующим детализации (уточнения) его кодирования в памяти компьютерной системы (например, в виде матрицы смежности, матрицы инцидентности, списковой структуры). Но такая детализация потребуется для технической реализации памяти, в которой хранятся и обрабатываются \textit{sc-тексты}.}
                    \scnitem{Важнейшим дополнительным свойством \textit{\textbf{SC-кода}} является то,что он удобен не просто для внутреннего представления знаний в памяти интеллектуальной компьютерной системы, но также и для внутреннего представления информации в памяти компьютеров, специально предназначенных для интерпретации семантических моделей интеллектуальных компьютерных систем. Т.е., \textit{\textbf{SC-код}} определяет синтаксические, семантические и функциональные принципы организации памяти компьютеров нового поколения, ориентированных на реализацию интеллектуальных компьютерных систем, --- принципы организации графодинамической ассоциативной семантической памяти.}
                    \scnitem{\textit{\textbf{SC-код}} рассматривается нами как объединение трех его подъязыков, в число которых входит \textit{\textbf{Ядро SC-кода}}, подъязык \textit{\textbf{SC-кода}}, обеспечивающий представление текстов \textit{\textbf{SC-кода}} (\textit{sc-текстов}) в форме орграфов классического вида, являющихся подразбиениями текстов \textit{\textbf{Ядра SC-кода}} и, соответственно, использующих \uline{явное} представление пар инцидентности элементов sc-текстов (sc-элементов), синтаксическое \textit{\textbf{Расширение Ядра SC-кода}}, обеспечивающее представление в памяти ostis-системы информационных конструкций инородного для \textit{\textbf{SC-кода}} вида.}
                \end{scnrelfromvector}
                \scnsourcecommentinline{Завершили Описание принципов, лежащих в основе \textit{SC-кода}}
                
                \scnheader{SC-код}
                \scntext{примечание}{Следует особо подчеркнуть, что  унификация и максимально возможное упрощение  \textbf{\textit{синтаксиса}} и \textbf{\textit{денотационной семантики}} внутреннего языка интеллектуальных компьютерных систем прежде всего необходимы потому, что подавляющий объем \textbf{\textit{знаний}}, хранимых в составе  базы знаний интеллектуальной компьютерной системы, представляют собой \textbf{\textit{метазнания}}, описывающие свойства других знаний. К \textit{метазнаниям}, в частности, следует отнести и различного вида логические высказывания и всевозможного вида программы, описания методов (навыков), обеспечивающих решение различных классов задач. Необходимо исключить зависимость формы представляемого знания от вида этого знания. Форма (структура) внутреннего представления знания любого вида должна зависеть \uline{только}(!) от смысла этого знания. Более того, конструктивное (формальное) развитие теории интеллектуальных компьютерных систем невозможно без уточнения (унификации, стандартизации) и обеспечения семантической совместимости различных видов знаний, хранимых в базе знаний интеллектуальной компьютерной  системы.  Очевидно, что многообразие форм представления семантически эквивалентных знаний делает разработку общей теории  интеллектуальных компьютерных систем практически невозможной.}
                
                \scnheader{SC-пространство}
                \scntext{примечание}{Понятие \textit{SC-пространства} наряду с понятием \textit{SC-кода} играет важнейшую роль для уточнения и формализации понятия смысла информационных конструкций, для унификации смыслового представления информации и для максимально возможного исключения субъективизма в трактовке понятия смысла. Смысл информационной конструкции в конечном счете определяется (1) конфигурацией смыслового представления этой конструкции и (2) и местоположением (контекстом) смыслового представления указанной информационной конструкции в рамках смыслового пространства, т.е. в рамках объединенного смыслового представления \uline{всевозможных} информационных конструкций, либо в рамках объединенного смыслового представления информации, накопленной к заданному моменту времени некоторым индивидуальным субъектом или коллективом субъектов. Таким объединенным смысловым представлением информации, в частности, является смысловое представление глобальной базы всех знаний, накопленных человечеством к текущему моменту.}\scntext{пояснение}{Объединение (вместилище) \uline{всевозможных} унифицированных семантических сетей (текстов SC-кода)}\scntext{примечание}{При теоретико-множественном объединении текстов \textit{SC-кода} семантически эквивалентные (синонимичные) элементы (синтаксически элементарные фрагменты) этих текстов считаются совпадающими элементами и при объединении указанных текстов склеиваются.}\scnrelto{объединение}{SC-код}
                \scnidtf{Унифицированное смысловое пространство}
                \scntext{достоинство}{Важнейшим достоинством \textit{SC-пространства} является возможность уточнения таких понятий, как понятие аналогичности (сходства и отличия) различных описываемых внешних сущностей, аналогичности унифицированных семантических сетей (текстов \textit{SC-кода}), понятие семантической близости описываемых сущностей (в том числе, и текстов \textit{SC-кода}).}
                \bigskip
            \end{scnsubstruct}
            \scnsourcecommentinline{Завершили Сегмент \scnqqi{Основные положения внутреннего языка ostis-систем}}
    
            \newpage
    
            \scnsegmentheader{Описание Ядра SC-кода}
            \begin{scnsubstruct}
                \scnstructheader{Синтаксис Ядра SC-кода}
                \begin{scnsubstruct}
                    \scnheader{Синтаксис Ядра SC-кода}
                    \scntext{примечание}{\textit{Синтаксис Ядра SC-кода} задается: (1) \textit{Алфавитом Ядра SC-кода}, (2) Отношением \textit{инцидентности sc-коннекторов*}, (3) Отношением \textit{инцидентности входящих sc-дуг*}}\scnrelto{синтаксис}{Ядро SC-кода}
                    
                    \scnheader{Ядро SC-кода}
                    \scnrelfrom{множество всех элементов конструкций данного языка}{sc-элемент}
                        \begin{scnindent}
                            \scnidtf{элемент конструкции \textit{Ядра SC-кода}}
                            \scnidtf{синтаксически элементарный (атомарный) фрагмент дискретной информационной конструкции, принадлежащей \textit{Ядру SC-кода}}
                            \scnidtf{Класс элементов конструкций \textit{Ядра SC-кода}}
                            \scnidtf{Множество всех элементов всевозможных конструкций \textit{Ядра SC-кода}}
                        \end{scnindent}
                    \scnrelfrom{алфавит}{Алфавит Ядра SC-кода\scnsupergroupsign}
                    
                    \scnheader{Алфавит Ядра SC-кода\scnsupergroupsign}    
                    \scnidtf{Множество (Семейство) всех классов синтаксически эквивалентных sc-элементов Ядра SC-кода}
                    \scnidtf{класс синтаксически эквивалентных sc-элементов Ядра SC-кода}
                    \scnidtf{класс синтаксически эквивалентных элементов конструкций Ядра SC-кода}
                    \scnidtf{элемент Алфавита Ядра SC-кода}
                    \scnidtf{синтаксический тип sc-элемента Ядра SC-кода}
                    \begin{scneqtoset}
                        \scnitem{sc-узел общего вида}
                        \scnitem{sc-ребро общего вида}
                        \scnitem{sc-дуга общего вида}
                        \scnitem{базовая sc-дуга}
                    \end{scneqtoset}
                    
                    \scnheader{sc-элемент}
                    \scnrelfrom{разбиение}{Алфавит Ядра SC-кода\scnsupergroupsign}
                    \begin{scnindent}
                        \scntext{примечание}{\textit{Алфавит Ядра SC-кода} является одним из признаков классификации sc-элементов.}
                        \scntext{примечание}{В процессе обработки текстов \textit{Ядра SC-кода} синтаксический тип \textit{sc-элементов} может меняться --- \textit{sc-узел} может трансформироваться в \textit{sc-ребро}, \textit{sc-ребро} --- в \textit{sc-дугу}, \textit{sc-дуга} общего вида --- в \textit{базовую sc-дугу}.}
                    \end{scnindent}
                    
                    \scnheader{синтаксически выделяемый класс sc-элементов в рамках Ядра SC-кода\scnsupergroupsign}
    
                    \scnidtf{класс \textit{sc-элементов}, определяемый на основе \textit{Алфавита Ядра SC-кода}}
                    \scnhaselement{sc-коннектор}
                    \scnhaselement{sc-дуга}
                    \scnsuperset{Алфавит Ядра SC-кода\scnsupergroupsign}
                    
                    \scnheader{sc-дуга}
    
                    \begin{scnsubdividing}
                        \scnitem{sc-дуга общего вида}
                        \scnitem{базовая sc-дуга}
                    \end{scnsubdividing}
                    
                    \scnheader{sc-коннектор}
    
                    \begin{scnsubdividing}
                        \scnitem{sc-ребро общего вида}
                        \scnitem{sc-дуга общего вида}
                    \end{scnsubdividing}
                \end{scnsubstruct}
    
                \scnstructheader{Синтаксическая классификация sc-элементов в рамках Ядра SC-кода}
                \begin{scnsubstruct}
                    
                    \scnheaderlocal{sc-элемент}
                    \begin{scnsubdividing}
                        \scnitem{sc-узел общего вида}
                        \scnitem{sc-коннектор}
                        \begin{scnindent}
                            \begin{scnsubdividing}
                                \scnitem{sc-ребро общего вида}
                                \scnitem{sc-дуга}
                                \begin{scnindent}
                                    \begin{scnsubdividing}
                                        \scnitem{sc-дуга общего вида}
                                        \scnitem{базовая sc-дуга}
                                    \end{scnsubdividing}
                                    \end{scnindent}
                            \end{scnsubdividing}
                        \end{scnindent}
                    \end{scnsubdividing}
                    \bigskip
                    \scnsourcecommentinline{Завершили представление \textit{Синтаксической классификации sc-элементов в рамках Ядра SC-кода}}
                    \scntext{примечание}{Все классы \textit{sc-элементов}, входящие в состав синтаксической классификации sc-элементов являются синтаксически выделяемыми классами \textit{sc-элементов}}
                    
                    \scnheader{инцидентность sc-коннекторов*}
    
                    \scnidtftext{определение}{Бинарное ориентированное отношение, первым компонентом каждой ориентированной пары которого является некоторый sc-коннектор, а вторым компонентом является один из sc-элементов, соединяемых указанным sc-коннектором с некоторым другим sc-элементом, который указывается в другой паре инцидентности для этого же sc-коннектора}
                    
                    \scnheader{инцидентность входящих sc-дуг*}
    
                    \scnidtftext{определение}{Бинарное ориентированное отношение, первым компонентом каждой ориентированной пары которого является некоторая sc-дуга, а вторым компонентом --- sc-элемент, в который указанная sc-дуга входит, т.е. sc-элемент, который является вторым компонентом, соединяемым (связываемым) указанной sc-дугой}
                    
                    \scnheader{Ядро SC-кода}
    
                    \scnrelfrom{синтаксические правила}{\scnstructidtf{Синтаксические правила Ядра SC-кода}}
                    \begin{scnindent}
                        \begin{scnhassubset}
                            \scnitem{
                                \begin{scnset}
                                    \scnitem{инцидентность sc-коннекторов*}
                                    \begin{scnindent}
                                        \scnsuperset{инцидентность входящих sc-дуг*}
                                        \scniselement{бинарное ориентированное отношение}
                                    \end{scnindent}
                                \end{scnset}
                            }
                            \scnitem{Для каждого sc-коннектора существует две и только две пары \textit{инцидентности sc-коннекторов*}, указанный sc-коннектор является первым связующим компонентом. При этом для каждой sc-дуги из двух указанных пар инцидентности \uline{одна} должна принадлежать отношению инцидентности \textit{входящей sc-дуги*}.}
                            \scnitem{Пары инцидентности sc-коннекторов могут быть \uline{кратными}. То есть sc-коннектор может соединять (связывать) sc-элемент с самим собой. Такие sc-коннекторы будем называть петлевыми sc-коннекторами (петлевыми sc-ребрами и петлевыми sc-дугами).}
                            \scnitem{Само \textit{Отношение инцидентности sc-коннекторов*} и, следовательно, \textit{Отношение инцидентности входящих sc-дуг*} не имеет кратных пар инцидентности. То есть sc-коннектор не может быть инцидентен самому себе.}
                            \scnitem{В область определения \textit{Отношения инцидентности sc-коннекторов*} и \textit{Отношения инцидентности входящих sc-дуг*} входят не только sc-узлы общего вида, но и sc-коннекторы. Это значит, что sc-коннектор может соединять (связывать) не только sc-узел с sc-узлом, но также sc-узел с sc-коннектором и даже sc-коннектор с sc-коннектором.}
                        \end{scnhassubset}
                    \end{scnindent}
                    \scnsourcecommentinline{Завершили перечень синтаксических правил Ядра SC-кода}
                    \bigskip
                \end{scnsubstruct}
                \scnsourcecommentinline{Завершили изложение \textit{Синтаксиса Ядра SC-кода}}
    
                \scnstructheader{Денотационная семантика Ядра SC-кода}
                \begin{scnsubstruct}
                    
                    \scnheader{Ядро SC-кода}
    
                    \scnrelfrom{денотационная семантика}{Денотационная семантика Ядра SC-кода}
                    \begin{scnindent}
                        \scnidtf{Описание соответствия информационных конструкций, принадлежащих \textit{Ядру SC-кода}, и сущностей, описываемых этими конструкциями}
                    \end{scnindent}
                    
                    \scnheader{параметр, заданный на множестве sc-элементов}
    
                    \scnhaselement{Алфавит Ядра SC-кода\scnsupergroupsign}
                    \scnhaselement{Алфавит SC-кода\scnsupergroupsign}
                    \scnhaselement{Структурная типология sc-элементов\scnsupergroupsign}
                    \scnhaselement{Типология sc-элементов по признаку константности\scnsupergroupsign}
                    \scnhaselement{Типология sc-элементов по признаку постоянства обозначаемой сущности\scnsupergroupsign}
                    \scnhaselement{Типология sc-элементов по признаку доступности sc-элемента в процессе эксплуатации и эволюции базы знаний\scnsupergroupsign}
                \end{scnsubstruct}
    
                \scnstructheader{Семантическая классификация sc-элементов}
                \begin{scnsubstruct}
                    
                    \scnheader{sc-элемент}
                    \scnidtf{обозначение описываемой сущности}
                    \scnrelfrom{разбиение}{\scnkeyword{Структурная типология sc-элементов\scnsupergroupsign}}
                    \begin{scnindent}
                        \begin{scneqtoset}
                            \scnitem{обозначение терминальной сущности}
                            \begin{scnindent}
                                \begin{scnsubdividing}
                                    \scnitem{обозначение материальной сущности}
                                    \begin{scnindent}
                                        \scntext{примечание}{К материальным сущностям относятся физические тела, поля, биологические объекты, технические системы и многое другое.}
                                    \end{scnindent}
                                    \scnitem{обозначение абстрактной терминальной сущности}
                                    \begin{scnindent}
                                        \scntext{примечание}{Примерами абстрактных терминальных сущностей являются предельно малые физические тела, точки различных пространств, числа.}
                                    \end{scnindent}
                                    \scnitem{обозначение дискретной информационной конструкции, не принадлежащей SC-коду}
                                    \begin{scnindent}
                                        \scnidtf{обозначение информационной конструкции, не являющейся конструкцией \textit{SC-кода} и тем более \textit{Ядра SC-кода}}
                                        \scnidtf{обозначение \scnqq{инородной} для \textit{SC-кода} информационной конструкции}
                                        \scnsuperset{обозначение файла}
                                        \begin{scnindent}
                                            \scnidtf{обозначение внешней информационной конструкции, представленной в электронной форме}
                                        \end{scnindent}
                                    \end{scnindent}
                                \end{scnsubdividing}
                            \end{scnindent}
                            \scnitem{обозначение множества}
                            \begin{scnindent}
                                \begin{scnsubdividing}
                                    \scnitem{обозначение связки}
                                    \scnitem{обозначение класса}
                                    \scnitem{обозначение структуры}
                                \end{scnsubdividing}
                            \end{scnindent}
                        \end{scneqtoset}
                    \end{scnindent}
                    
                    \scnheader{обозначение связки}
                    \begin{scnsubdividing}
                        \scnitem{обозначение небинарной связки}
                        \scnitem{обозначение sc-пары}
                        \begin{scnindent}
                            \begin{scnsubdividing}
                                \scnitem{обозначение неориентированной sc-пары}
                                \scnitem{обозначение ориентированной пары неизвестной направленности}
                                \scnitem{обозначение ориентированной sc-пары}
                                \begin{scnindent}
                                    \scnsuperset{обозначение sc-пары принадлежности}
                                    \begin{scnindent}
                                        \begin{scnsubdividing}
                                            \scnitem{обозначение позитивной sc-пары принадлежности}
                                            \scnitem{обозначение негативной sc-пары принадлежности}
                                            \scnitem{обозначение нечеткой sc-пары принадлежности}
                                        \end{scnsubdividing}
                                    \end{scnindent}
                                \end{scnindent}
                            \end{scnsubdividing}
                        \end{scnindent}
                    \end{scnsubdividing}
                    
                    \scnheader{sc-элемент}
                    \scnrelfrom{разбиение}{\scnkeyword{Типология sc-элементов по признаку константности\scnsupergroupsign}}
                        \begin{scnindent}
                        \begin{scneqtoset}
                            \scnitem{sc-константа}
                            \begin{scnindent}
                                \scnidtf{константный sc-элемент}
                                \scnidtf{обозначение конкретной (фиксированной) сущности}
                            \end{scnindent}
                            \scnitem{sc-переменная}
                            \begin{scnindent}
                                \scnidtf{переменный sc-элемент}
                                \scnidtf{обозначение произвольной сущности из некоторого множества сущностей}
                                \scnidtf{sc-элемент, имеющий (принимающий) произвольное значение из некоторого множества sc-элементов}
                            \end{scnindent}
                        \end{scneqtoset}
                    \end{scnindent}
                    \scnrelfrom{разбиение}{\scnkeyword{Типология sc-элементов по постоянству обозначаемых сущностей\scnsupergroupsign}}
                    \begin{scnindent}
                        \begin{scneqtoset}
                            \scnitem{обозначение постоянной сущности}
                            \scnitem{обозначение временной сущности}
                            \begin{scnindent}
                                \scnidtf{обозначение нестационарной сущности, факт существования которой зависит от времени}
                                \begin{scnsubdividing}
                                    \scnitem{обозначение прошлой сущности}
                                    \begin{scnindent}
                                        \scnidtf{обозначение сущности, существовавшей до текущего момента времени}
                                    \end{scnindent}
                                    \scnitem{обозначение настоящей сущности}
                                    \begin{scnindent}
                                        \scnidtf{обозначение сущности, существующей в текущий момент времени}
                                    \end{scnindent}
                                    \scnitem{обозначение будущей сущности}
                                    \begin{scnindent}
                                        \scnidtf{обозначение сущности, существование которой прогнозируется или планируется в будущем}
                                    \end{scnindent}
                                \end{scnsubdividing}
                            \end{scnindent}
                        \end{scneqtoset}
                    \end{scnindent}
                    \scnrelfrom{разбиение}{\scnkeyword{Типология sc-элементов по признаку доступности sc-элемента в процессе эксплуатации и эволюции базы знаний\scnsupergroupsign}}
                    \begin{scnindent}
                        \begin{scneqtoset}
                            \scnitem{удаленный sc-элемент}
                            \begin{scnindent}
                                \scnidtf{sc-элемент, считающийся логически удаленным, но присутствующим в описании истории эксплуатации и эволюции базы знаний}
                            \end{scnindent}
                            \scnitem{настоящий sc-элемент}
                            \begin{scnindent}
                                \scnidtf{sc-элемент, входящий в состав эксплуатируемой части базы знаний}
                            \end{scnindent}
                            \scnitem{будущий sc-элемент}
                            \begin{scnindent}
                                \scnidtf{sc-элемент, планируемый для включения в состав эксплуатируемой части базы знаний}
                            \end{scnindent}
                        \end{scneqtoset}
                    \end{scnindent}
                    
                    \scnheader{обозначение множества}
                    \scnidtf{обозначение множества sc-элементов}
                    \begin{scnsubdividing}
                        \scnitem{произвольное множество}
                        \begin{scnindent}
                        	\scnidtf{sc-переменная, обозначающая произвольное множество из некоторого семейства множеств}
                            \scnidtf{переменное множество}
                        \end{scnindent}
                        \scnitem{множество}
                        \begin{scnindent}
                        	\scnidtf{конкретное (константное, фиксированное) множество sc-элементов}
                        \end{scnindent}
                    \end{scnsubdividing}
                    
                    \scnheader{множество}
                    \scnidtf{множество sc-элементов}
                    \begin{scnsubdividing}
                        \scnitem{множество sc-констант}
                        \begin{scnindent}
                        	\scnidtf{множество, элементами которого являются только sc-константы}
                        \end{scnindent}
                        \scnitem{множество sc-переменных}
                        \begin{scnindent}
                            \scnidtf{множество, элементами которого являются только sc-переменные}
                            \scnsuperset{sc-переменная}
                            \begin{scnindent}
                                \scnidtf{множество, элементами которого являются всевозможные sc-переменные и только они}
                                \scnsuperset{произвольное множество}
                                \begin{scnindent}
                                    \scnidtf{sc-переменная, значениями которой являются всевозможные обозначения множеств и только они}
                                \end{scnindent}
                            \end{scnindent}
                        \end{scnindent}
                        \scnitem{множество sc-констант и sc-переменных}
                        \begin{scnindent}
                        	\scnidtf{множество, в число элементов которого входят как sc-константы, так и sc-переменные}
                            \scnsuperset{обозначение множества}
                            \begin{scnindent}
                                \scnidtf{множество, элементами которого являются всевозможные \mbox{sc-переменные} и \mbox{sc-константы}, обозначающие множества и только они}
                            \end{scnindent}
                        \end{scnindent}
                    \end{scnsubdividing}
                    
                    \scnheader{обозначение связки}
                    \scnidtf{обозначение связи между sc-элементами и/или обозначаемыми ими сущностями}
                    \begin{scnsubdividing}
                        \scnitem{произвольная связка}
                        \begin{scnindent}
                        	\scnidtf{sc-переменная, значениями которой являются обозначения связок}
                        \end{scnindent}
                        \scnitem{связка}
                        \begin{scnindent}
                        	\scnidtf{конкретная связка sc-элементов}
                        \end{scnindent}
                    \end{scnsubdividing}
                    \scnsuperset{обозначение sc-пары}
                    \begin{scnindent}
                        \scnidtf{обозначение связки двух sc-элементов либо одного sc-элемента с самим собой}
                        \scnsuperset{sc-пара}
                        \scnidtf{конкретная sc-пара}
                        \scnsubset{sc-константа}
                        \scnsuperset{sc-коннектор}
                        \scnsuperset{ориентированная sc-пара}
                        \begin{scnindent}
                            \scnsuperset{sc-пара принадлежности}
                            \begin{scnindent}
                                \begin{scnsubdividing}
                                    \scnitem{позитивная sc-пара принадлежности}
                                    \begin{scnindent}
                                        \scnsuperset{позитивная постоянная sc-пара принадлежности}
                                        \begin{scnindent}
                                            \scnsuperset{базовая sc-дуга}
                                        \end{scnindent}
                                    \end{scnindent}
                                    \scnitem{негативная sc-пара принадлежности}
                                    \scnitem{нечеткая sc-пара принадлежности}
                                \end{scnsubdividing}
                            \end{scnindent}
                        \end{scnindent}
                    \end{scnindent}

                    \scnheader{обозначение класса}
                    \scnidtf{обозначение множества sc-элементов, которые в соответствующем смысле эквивалентны друг другу, т.е. имеют одинаковые свойства}
                    \begin{scnsubdividing}
                        \scnitem{произвольный класс}
                        \begin{scnindent}
                        	\scnsubset{sc-переменная}
                            \scniselement{sc-константа}
                        \end{scnindent}
                        \scnitem{класс}
                        \begin{scnindent}
                        	\scnsubset{sc-константа}
                        \end{scnindent}
                    \end{scnsubdividing}
                    
                    \scnheader{класс}
                    \begin{scnsubdividing}
                        \scnitem{класс терминальных сущностей}
                        \scnitem{класс множеств}
                        \begin{scnindent}
                            \begin{scnsubdividing}
                                \scnitem{класс связок}
                                \begin{scnindent}
                                    \scnsuperset{sc-отношение}
                                \end{scnindent}
                                \scnitem{класс классов}
                                \begin{scnindent}
                                    \scnsuperset{параметр}
                                \end{scnindent}
                                \scnitem{класс структур}
                                \begin{scnindent}
                                    \scnsuperset{sc-язык}
                                    \begin{scnindent}
                                        \scnidtf{специализированный язык, являющийся подъязыком SC-кода, и обеспечивающий представление всевозможных знаний в рамках соответствующей предметной области, которая, в свою очередь, специфицируется соответствующей комплексной онтологией}
                                    \end{scnindent}
                                \end{scnindent}
                            \end{scnsubdividing}
                        \end{scnindent}
                    \end{scnsubdividing}
                    \scnhaselement{обозначение множества}
                    \begin{scnindent}
                        \scnsuperset{множество}
                    \end{scnindent}
                    \scnhaselement{множество}
                    \scnhaselement{обозначение связки}
                    \begin{scnindent}
                        \scnsuperset{связка}
                    \end{scnindent}
                    \scnhaselement{связка}
                    \scnhaselement{обозначение класса}
                    \begin{scnindent}
                        \scnsuperset{класс}
                    \end{scnindent}
                    \scnhaselement{класс}
                    \scnhaselement{обозначение структуры}
                    \begin{scnindent}
                        \scnsuperset{структура}
                    \end{scnindent}
                    \scnhaselement{структура}
                    \scnhaselement{обозначение дискретной информационной конструкции}
                    \begin{scnindent}
                        \scnsuperset{дискретная информационная конструкция}
                        \begin{scnindent}
                            \scnsuperset{файл}
                            \begin{scnindent}
                                \scnsuperset{файл ostis-системы}
                                \begin{scnindent}
                                    \scnsuperset{внутренний файл ostis-системы}
                                \end{scnindent}
                            \end{scnindent}
                        \end{scnindent}
                    \end{scnindent}
                        \scnsuperset{обозначение структуры}
                    \scntext{примечание}{Все семантически и синтаксически выделяемые классы sc-элементов, а также всевозможные подклассы этих классов являются экземплярами (элементами) \textit{класса}}
                    
                    \scnheader{обозначение структуры}
                    \scnidtf{обозначение множества, не являющегося ни связкой, ни классом}
                    \begin{scnsubdividing}
                        \scnitem{произвольная структура}
                        \begin{scnindent}
                        	\scnsubset{sc-переменная}
                        \end{scnindent}
                        \scnitem{структура}
                        \begin{scnindent}
                        	\scnidtf{конкретная структура}
                            \scnsubset{sc-константа}
                        \end{scnindent}
                    \end{scnsubdividing}
                \end{scnsubstruct}
                \scnsourcecommentinline{Завершили представление \textit{Семантической классификации sc-элементов}}
    
                \newpage
            
                \scnstructheader{Соотношение между семантически и синтаксически выделяемыми классами sc-элементов в рамках Ядра SC-кода}
                \begin{scnsubstruct}
                
                    \scnheader{семантически выделяемый класс sc-элементов}
                    \begin{scnsubstruct}
                        \scnidtf{класс sc-элементов, определяемый сущностями, которые обозначаются этими sc-элементами, также доступностью (активностью использования) sc-элементов в процессе эксплуатации и эволюции базы знаний}
                        \scniselement{обозначение терминальной сущности}
                        \begin{scnindent}
                            \scnsuperset{\scnkeyword{sc-узел общего вида}}
                        \end{scnindent}
                        \scniselement{обозначение небинарной связки}
                        \begin{scnindent}
                            \scnsuperset{\scnkeyword{sc-узел общего вида}}
                        \end{scnindent}
                        \scniselement{обозначение sc-пары}
                        \begin{scnindent}
                            \scnrelboth{пара пересекающихся множеств}{sc-узел общего вида}
                            \scnsuperset{\scnkeyword{sc-коннектор}}
                            \scntext{примечание}{\textit{обозначение sc-пары} может быть представлено либо \textit{sc-узлом общего вида}, либо \textit{sc-коннектором}. При этом каждый \textit{sc-коннектор} представляет собой \textit{обозначение sc-пары}.}
                        \end{scnindent}
                        \scniselement{обозначение неориентированной sc-пары}
                        \begin{scnindent}
                            \scnidtf{обозначение бинарной неориентированной связи между sc-элементами}
                            \begin{scnrelbothlist}{пара пересекающихся множеств}
                                \scnitem{\scnkeyword{sc-узел общего вида}}
                                \scnitem{\scnkeyword{sc-ребро общего вида}}
                            \end{scnrelbothlist}
                            \scntext{примечание}{\textit{обозначение неориентированной sc-пары} может быть представлено либо \textit{sc-узлом общего вида}, либо \textit{sc-ребром}. При этом не каждое \textit{sc-ребро} представляет обозначение \textit{неориентированный sc-пары}. Некоторые из них представляют \textit{обозначения ориентированных sc-пар неизвестной направленности}.}
                        \end{scnindent}
                        \scniselement{обозначение ориентированной sc-пары неизвестной направленности}
                        \begin{scnindent}
                            \begin{scnrelbothlist}{пара пересекающихся множеств}
                                \scnitem{\scnkeyword{sc-узел общего вида}}
                                \scnitem{\scnkeyword{sc-ребро общего вида}}
                            \end{scnrelbothlist}
                        \end{scnindent}
                        \scniselement{обозначение ориентированной sc-пары}
                        \begin{scnindent}
                            \scnidtf{обозначение бинарной ориентированной связи между sc-элементами}
                            \begin{scnrelbothlist}{пара пересекающихся множеств}
                                \scnitem{\scnkeyword{sc-узел общего вида}}
                                \scnitem{\scnkeyword{sc-ребро общего вида}}
                            \end{scnrelbothlist}
                            \scnsuperset{sc-дуга общего вида}
                        \end{scnindent}
                        \scniselement{константная постоянная позитивная sc-пара принадлежности}
                        \begin{scnindent}
                            \begin{scnrelbothlist}{пара пересекающихся множеств}
                                \scnitem{\scnkeyword{sc-узел общего вида}}
                                \scnitem{\scnkeyword{sc-ребро общего вида}}
                                \scnitem{\scnkeyword{sc-дуга общего вида}}
                            \end{scnrelbothlist}
                            \scnsuperset{\scnkeyword{базовая sc-дуга}}
                            \begin{scnreltoset}{пересечение множеств}
                                \scnitem{sc-константа}
                                \scnitem{обозначение постоянной сущности}
                                \scnitem{обозначение sc-пары принадлежности}
                            \end{scnreltoset}
                        \end{scnindent}
                        \scniselement{обозначение класса}
                        \begin{scnindent}
                            \scnsubset{\scnkeyword{sc-узел общего вида}}
                        \end{scnindent}
                        \scniselement{обозначение структуры}
                        \begin{scnindent}
                            \scnsubset{\scnkeyword{sc-узел общего вида}}
                        \end{scnindent}
                    \end{scnsubstruct}
                    \scnsourcecommentinline{Завершили \textit{Описание Соотношения между семантически и синтаксически выделяемыми классами sc-элементов в рамках Ядра SC-кода}. В этом описании жирным курсивом выделены идентификаторы (имена) синтаксически выделяемых классов sc-элементов}
                    \newpage
                    
                    \scnheader{Ядро SC-кода}
                    \begin{scnsubstruct}
                        \scnrelfrom{семантические правила}{\scnstructidtf{Семантические правила Ядра SC-кода}}
                        \begin{scnindent}
                            \begin{scnhassubset}
                                \scnfileitem{Каждый sc-элемент является знаком (обозначением) некоторой описываемой сущности.}
                                \scnfileitem{Любая сущность может быть обозначена sc-элементом и, соответственно, описана в виде конструкции Ядра SC-кода.}
                                \scnfileitem{С помощью sc-элементов можно описать любые связи между sc-элементами и/или между сущностями, которые обозначаются этими sc-элементами. При этом указанные связи трактуются как множества связываемых sc-элементов и обозначаются sc-ребрами, sc-дугами, а в случае небинарных связей --- sc-узлами.}
                                \scnfileitem{Поскольку каждый \mbox{sc-коннектор} семантически трактуется как обозначение пары \mbox{sc-элементов}, связываемых (соединяемых) этим \mbox{sc-коннектором}, каждая пара инцидентности \mbox{sc-коннектора} семантически интерпретируется как обозначение пары принадлежности, связывающей \mbox{sc-коннектор} с одним из элементов обозначаемой им пары \mbox{sc-элементов}.}
                                \scnfileitem{\uline{Любая} описываемая сущность может быть обозначена sc-узлом общего вида, но обратное неверно, т.к. некоторые сущности могут быть обозначены sc-ребрами общего вида, sc-дугами общего вида, базовыми sc-дугами.}
                                \scnfileitem{Каждое sc-ребро является обозначением либо бинарной неориентированной связи между sc-элементами, либо бинарной ориентированной связи неизвестной направленности между sc-элементами.}
                                \scnfileitem{Любая бинарная неориентированная связь между sc-элементами может быть обозначена sc-ребром, но обратное неверно.}
                            \end{scnhassubset}
                        \end{scnindent}
                    \end{scnsubstruct}
                    \scnsourcecommentinline{Завершили описание \textit{Денотационной семантики Ядра SC-кода}}
                    
                    \scnheader{Правила синтаксической трансформации sc-элементов в рамках Ядра SC-кода}
                    \scnidtf{Правила модификации синтаксического типа sc-элементов в рамках Ядра SC-кода}
                    \begin{scnhassubset}
                        \scnfileitem{Если \textit{sc-узел общего вида} является \textit{обозначением sc-пары}, то он трансформируется в \textit{sc-коннектор}}
                        \scnfileitem{Если \textit{sc-узел общего вида} является \textit{обозначением неориентированной sc-пары} или \textit{обозначением ориентированной sc-пары неизвестной направленности}, то он трансформируется в \textit{sc-ребро общего вида}}
                        \scnfileitem{Если \textit{sc-узел общего вида} или \textit{sc-ребро общего вида} являются \textit{обозначением ориентированной sc-пары} и при этом дополнительно указана направленность этой sc-пары, то она трансформируется в \textit{sc-дугу общего вида}.}
                        \scnfileitem{Если \textit{sc-узел общего вида} или \textit{sc-ребро общего вида} или \textit{sc-дуга общего вида} являются \textit{константными постоянными позитивными парами принадлежности}, то они трансформируются в \textit{базовую sc-дугу}.}
                    \end{scnhassubset}
                    
                    \scnheader{следует отличать*}
                    \begin{scnhaselementset}
                        \scnitem{синтаксически выделяемый класс sc-элементов в рамках Ядра SC-кода}
                        \scnitem{синтаксически выделяемый класс sc-элементов в рамках SC-кода}
                        \scnitem{семантически выделяемый класс sc-элементов}
                    \end{scnhaselementset}
                    \bigskip
                \end{scnsubstruct}
            \end{scnsubstruct}
            \scnsourcecommentinline{Завершили Сегмент \textit{\scnqqi{Описание Ядра SC-кода}}}
    
            \scnsegmentheader{SC-код как синтаксическое расширение Ядра SC-кода}
            \begin{scnsubstruct}
                \scnstructheader{Сравнение SC-кода и Ядра SC-кода}
                \begin{scnhassubset}
                    \scnitem{SC-код}
                    \begin{scnindent}
                        \scnrelto{синтаксическое расширение языка}{Ядро SC-кода}
                        \scnidtf{Синтаксическое расширение Ядра SC-кода}
                        \begin{scnindent}
                            \scntext{примечание}{Синтаксическое расширение Ядра SC-кода заключается во введении дополнительного класса синтаксически эквивалентных элементарных фрагментов конструкций Ядра SC-кода --- sc-элементов, обозначающих \scnqq{внутренние} файлы, хранимые в памяти ostis-системы}
                        \end{scnindent}
                        \scnrelboth{семантическая эквивалентность языков}{Ядро SC-кода}
                        \scntext{примечание}{Семантическая эквивалентность \textit{SC-кода} и \textit{Ядра SC-кода} является следствием того, что \textit{SC-код} является \uline{синтаксическим} расширением \textit{Ядра SC-кода}}
                        \scnidtf{Результат введения в \textit{Ядро SC-кода} sc-узлов, имеющих содержимое и обозначающих файлы, хранимые в памяти ostis-системы, т.е. внутренние файлы ostis-системы}
                        \scntext{примечание}{Результатом просмотренного расширения \textit{Ядра SC-кода} является расширение \textit{Алфавита Ядра SC-кода}}
                        \scntext{примечание}{Все \textit{файлы}, представляющие собой электронные образы инородных для \textit{SC-кода} информационных конструкций, можно представить в \textit{SC-коде} с помощью графовых структур, в которых \textit{sc-элементы} обозначают буквы текстов или пиксели изображений. Но такой вариант кодирования внешних для \textit{ostis-системы} информационных конструкций не дает возможности непосредственно использовать накопленный человечеством арсенал электронных информационных ресурсов.}
                        \scntext{примечание}{Важнейшим видом внутренних \textit{файлов ostis-систем} являются внутренние файлы \textit{внешних идентификаторов sc-элементов} (в частности, имен sc-элементов), представляющих \textit{sc-элементы} в текстах внешних языков (в том числе, в текстах \textit{SCs-кода} и \textit{SCn-кода})}
                        \bigskip
                    \end{scnindent}
                    \scnitem{\scnnonamednode}
                    \begin{scnindent}
                         \begin{scneqtoset}
                            \scnitem{Ядро SC-кода}
                            \scnitem{SC-код}
                        \end{scneqtoset}
                        \scntext{сравнение}{Множество всех элементов конструкций \textit{Ядра SC-кода} и Множество всех элементов конструкций \textit{SC-кода} полностью совпадают, т.к. для каждого элемента конструкции \textit{Ядра SC-кода} существует синонимичный ему элемент конструкции \textit{SC-кода} и наоборот. Из этого следует, что семантическая классификации \textit{элементов информационных конструкций} \textit{SC-кода} и \textit{Ядра SC-кода} также полностью совпадают.Семантика \textit{SC-кода} ничем не отличается от семантики \textit{Ядра SC-кода}. То есть все, что может быть обозначено и описано текстами \textit{SC-кода}, может быть обозначено и описано текстами \textit{Ядра SC-кода}. Отличие \textit{SC-кода} от \textit{Ядра SC-кода} заключается только в том, что в \textit{SC-код} добавляется новый синтаксически выделяемый класс sc-элементов --- класс sc-элементов, являющихся знаками конкретных (константных) файлов, хранимых в памяти ostis-системы. Такие \textit{внутренние файлы} необходимы для того, чтобы в \textit{памяти ostis-системы} можно было хранить и обрабатывать \textit{информационные конструкции}, не являющиеся текстами \textit{SC-кода}, что необходимо при вводе (восприятии) информации, поступающей извне, а также при генерации \textit{информационных конструкций}, передаваемых другим субъектам. Включение в \textit{SC-код} специальных \uline{синтаксически} выделяемых \textit{sc-узлов}, обозначающих хранимые в \textit{sc-памяти} электронные образы (файлы) различного вида \textit{информационных конструкций}, не являющихся конструкциями \textit{SC-кода}, дает возможность непосредственно в \textit{памяти ostis-системы}, то есть в одной и той же запоминающий среде обрабатывать не только конструкции \textit{SC-кода}, но и \scnqq{инородные} для него конструкции, что для необходимо для реализации \textit{интерфейса ostis-системы}, обеспечивающего ее взаимодействие с \textit{внешней средой}. Без такой реализации \textit{интерфейса ostis-системы} невозможно реализовать синтаксический анализ, семантический анализ и понимание, а также невозможно реализовать синтез (генерацию) внешних информационных конструкций, принадлежащих заданному внешнему языку и семантически эквивалентных заданному смыслу. Поскольку все синтаксические и семантические свойства \textit{SC-кода} и \textit{Ядра SC-кода} являются весьма близкими, при описании \textit{SC-кода} акцентируется внимание на его отличия от \textit{Ядра SC-кода}, а также на более детальное рассмотрение семантической классификации элементов.}
                    \end{scnindent}
                \end{scnhassubset}
                \scnsourcecommentinline{Завершили \textit{Сравнение SC-кода и Ядра SC-кода}}
                \newpage
                    
                \scnstructheader{Синтаксис SC-кода}
                \begin{scnsubstruct}
    
                    \scnheader{SC-код}
                    \scnrelfrom{синтаксис}{синтаксис SC-кода}
                    \begin{scnindent}
                        \scntext{примечание}{\textit{Синтаксис SC-кода} отличается от \textit{Синтаксиса Ядра SC-кода} только тем, что в \textit{Алфавит \mbox{SC-кода}} дополнительно вводится класс sc-узлов, являющихся знаками \textit{файлов}, хранимых в памяти \textit{\mbox{ostis-системы}}}
                    \end{scnindent}
                    \scnrelfrom{множество всех экземпляров конструкций данного языка}{sc-элемент}
                    \begin{scnindent}
                        \scnidtf{элемент конструкции SC-кода}
                        \scntext{примечание}{Множество всех элементов конструкций SC-кода совпадает со множеством всех элементов конструкций Ядра SC-кода. Просто в конструкциях SC-кода некоторые sc-элементы, имеющие \scnqq{синтаксическую метку} (синтаксический тип) \textit{sc-узла общего вида}, будут иметь \scnqq{метку} sc-узла, являющегося знаком \textit{внутреннего файла},  хранимого в памяти \textit{ostis-системы}}
                    \end{scnindent}
                    \scntext{пояснение}{\textit{Синтаксис SC-кода} задается
                    \begin{scnitemize}
                        \item типологией (алфавитом) sc-элементов (атомарных фрагментов текстов SC-кода);
                        \item правилами соединения (инцидентности) sc-элементов (например, sc-элементы каких типов не могут быть инцидентными друг другу);
                        \item типологией конфигураций sc-элементов (связки, классы, структуры), связями между конфигурациями sc-элементов (в частности, теоретико-множественными)
                    \end{scnitemize}
                    }
                    
                    \scnheader{Алфавит SC-кода}
                    \scnrelto{алфавит}{SC-код}
                    \scnrelfrom{разбиение}{sc-элемент}
                    \begin{scneqtoset}
                    	\scnitem{(Алфавит Ядра SC-кода $\cup$ \scnkeyword{внутренний файл ostis-системы})}
                    \end{scneqtoset}
                    \begin{scneqtoset}
                        \scnitem{sc-узел общего вида}
                        \scnitem{\textit{\scnkeyword{внутренний файл ostis-системы}}} 
                        \scnitem{sc-ребро общего вида}
                        \scnitem{sc-дуга общего вида}
                        \scnitem{базовая sc-дуга}
                    \end{scneqtoset}
                    
                    
                    \scnheader{Алфавит SC-кода}
                    \scnidtf{Алфавит sc-элементов в рамках SC-кода}
                    \scnidtf{Семейство всех максимальных множеств синтаксически эквивалентных (в рамках SC-кода) sc-элементов}
                    \scnidtf{Семейство классов синтаксически эквивалентных sc-элементов SC-кода}
                    \scnidtf{Семейство всех множеств, в каждое из которых входят все синтаксически эквивалентные друг другу (в рамках SC-кода) sc-элементы и только они}
                    \scnidtf{Фактор-множество отношения \scnqq{синтаксическая эквивалентность sc-элементов в рамках SC-кода}}
                    \scneq{фактор-множество*(синтаксическая эквивалентность sc-элементов в рамках SC-кода*)}
                    \begin{scnindent}
                        \scniselement{сложный внешний идентификатор sc-элемента}
                    \end{scnindent}
                    \scnidtf{Семейство множеств sc-элементов, являющихся результатом разбиения максимального множества \mbox{sc-элементов} SC-кода (класса всевозможных sc-элементов) по признаку синтаксической эквивалентности sc-элементов}
                    \scnidtf{Признак (параметр) синтаксической эквивалентности sc-элементов}
                    
                    \scnheader{внутренний файл ostis-системы}
                    \scnidtf{sc-узел, имеющий содержимое}
                    \scnidtf{sc-ссылка}
                    \scnidtf{множество всевозможных sc-узлов, имеющих содержимое, хранимое в памяти ostis-системы}
                    \scnidtf{внутренний файл, хранимый в памяти ostis-системы}
                    \scnidtf{внутренний файл для заданной ostis-системы (той ostis-системы, в памяти которой хранится sc-узел, обозначающий этот файл)}
                    \scnidtf{sc-узел, являющийся знаком конкретного файла, хранимого в той же sc-памяти (в качестве содержимого sc-узла), в которой находится и сам указанный sc-узел}
                    \scnidtf{файл, знак которого находится в той же sc-памяти, в которой находится и сам файл}
                    \scnidtf{свой файл ostis-системы}
                    \scnsubset{внутренняя информационная конструкция}
                    
                    \scnheader{синтаксически выделяемый класс sc-элементов в рамках SC-кода}
                    \scnidtf{класс sc-элементов, определяемый на основе Алфавита SC-кода}
                    \scnsuperset{Алфавит SC-кода}
                    \scnhaselement{sc-узел, не являющийся знаком внутреннего файла ostis-системы}
                \end{scnsubstruct}
                \scnstructheader{Синтаксическая классификация sc-элементов в рамках SC-кода}
                \begin{scnsubstruct}                   
                    \scnheaderlocal{sc-элемент}
                    \begin{scnsubdividing}
                        \scnitem{sc-узел общего вида}
                        \begin{scnindent}
                            \begin{scnsubdividing}
                                \scnitem{sc-узел, не являющийся знаком внутреннего файла ostis-системы}
                                \scnitem{внутренний файл ostis-системы}
                            \end{scnsubdividing}
                        \end{scnindent}
                        \scnitem{sc-коннектор}
                        \begin{scnindent}
                            \begin{scnsubdividing}
                                \scnitem{sc-ребро общего вида}
                                \scnitem{sc-дуга}
                                \begin{scnindent}
                                    \begin{scnsubdividing}
                                        \scnitem{sc-дуга общего вида}
                                        \scnitem{базовая sc-дуга}
                                    \end{scnsubdividing}
                                \end{scnindent}
                            \end{scnsubdividing}
                        \end{scnindent}
                    \end{scnsubdividing}
                    \bigskip
                    \scntext{примечание}{Данная \textit{Синтаксическая классификация sc-элементов} от \textit{Синтаксической классификации sc-элементов Ядра SC-кода} отличается только дополнительным уточнением синтаксической типологии \textit{sc-узлов}}
                \end{scnsubstruct}
                \scnsourcecommentinline{Завершили представление \textit{Синтаксиса SC-кода}}
    
                \scnstructheader{Денотационная семантика SC-кода}
                \begin{scnsubstruct}
                    
                    \scnheader{Денотационная семантика SC-кода}
                    \scntext{аннотация}{\textit{Денотационную семантику SC-кода} рассмотрим как расширение и уточнение \textit{Денотационной семантики Ядра SC-кода} (смотрите предыдущий сегмент \scnqq{\textit{Описание Ядра SC-кода}}). Изложение построим как последовательное уточнение следующих понятий:
                    \begin{scnitemize}
                        \item \textit{sc-переменная}
                        \item \textit{обозначение дискретной информационной конструкции}
                        \item \textit{дискретная информационная конструкция} (рассмотрим различные параметры и отношения, заданные на множестве дискретных информационных конструкций)
                        \item \textit{знание} (как частный вид дискретных информационных конструкций)
                        \item \textit{файл} (как \textit{sc-константа}, являющаяся \textit{обозначением файла})
                        \item \textit{внутренний файл ostis-системы}
                        \item \textit{структура} (как \textit{дискретная информационная конструкция}, принадлежащая \textit{SC-коду})
                    \end{scnitemize}
                    }
                    
                    \scnheader{SC-код}
                    \scnrelfrom{денотационная семантика}{Денотационная семантика SC-кода}
                    \begin{scnindent}
                        \scntext{пояснение}{\textit{Денотационная семантика SC-кода} задается
                        \begin{scnitemize}
                            \item семантической интерпретацией sc-элементов и их конфигураций;
                            \item семантической интерпретацией инцидентности sc-элементов;
                            \item иерархической системой \textit{предметных областей};
                            \item структурой используемых понятий в каждой предметной области (исследуемые классы объектов, исследуемые отношения, исследуемые классы объектов отношений из смежных предметных областей, ключевые экземпляры исследуемых классов объектов);
                            \item \textit{онтологиями предметных областей};
                        \end{scnitemize}
                        }
                    \end{scnindent}
                \end{scnsubstruct}    
    
                \scnstructheader{Классификация sc-переменных}
                \begin{scnsubstruct}
                    
                    \scnheader{sc-переменная}
                    \scnidtf{sc-элемент, представляющий собой обозначение произвольной (переменной) сущности из некоторого дополнительно уточняемого множества обозначений других сущностей, которые считаются возможными значениями указанной произвольной сущности}
                    \scnrelto{область задания}{значение переменной*}
                    \begin{scnindent}
                        \scnidtf{Бинарное ориентированное отношение, связывающее sc-переменные с их возможными значениями*}
                        \scntext{пояснение}{Это одно из отношений, заданных на множестве sc-переменных}
                    \end{scnindent}
                    \scnrelfrom{разбиение}{\scnkeyword{Структурная типология sc-переменных}}
                    \begin{scnindent}
                        \begin{scneqtoset}
                            \scnitem{произвольная терминальная сущность}
                                \begin{scnindent}
                                    \scnidtf{sc-переменная, обозначающая терминальную сущность}
                                    \scnidtf{sc-переменная, значением или значением значения и т.д. которой является терминальная сущность}
                                    \scnidtf{sc-переменная, \scnqq{конечным} значением которой является терминальная сущность}
                                    \scnidtf{обозначение произвольной терминальной сущности}
                                \end{scnindent}
                            \scnitem{произвольное множество sc-элементов}
                        \end{scneqtoset}
                    \end{scnindent}
                    \begin{scnsubdividing}
                        \scnitem{sc-переменная, у которой логический уровень всех ее значений одинаков}
                        \begin{scnindent}
                        	\scnsuperset{первичная sc-переменная}
                            \begin{scnindent}
                                \scnidtf{sc-переменная, все значения которой являются sc-константами}
                            \end{scnindent}
                            \scnsuperset{вторичная sc-переменная}
                            \begin{scnindent}
                                \scnidtf{sc-переменная, все значения которой являются первичными sc-переменными}
                            \end{scnindent}
                            \scnsuperset{sc-переменная третьего уровня}
                            \begin{scnindent}    
                                \scnidtf{sc-переменная, все значения которой являются вторичными sc-переменными}
                            \end{scnindent}
                        \end{scnindent}
                        \scnitem{sc-переменная, значения которой имеют различный логический уровень}
                    \end{scnsubdividing}
                    \begin{scnsubdividing}
                        \scnitem{sc-переменная, у которой синтаксический тип всех её значений одинаков}
                        \begin{scnindent}
                            \scnsuperset{переменный sc-узел}
                            \begin{scnindent}
                                \scnidtf{sc-переменная, все значения которой являются sc-узлами}
                            \end{scnindent}
                            \scnsuperset{переменное sc-ребро}
                            \scnsuperset{переменная sc-дуга}
                        \end{scnindent}
                        \scnitem{sc-переменная, значения которой имеют различный синтаксический тип}
                    \end{scnsubdividing}
                    \scnsourcecommentinline{Завершили \textit{Классификации sc-переменных}}
                    \bigskip
                    
                    \scnheader{обозначение дискретной информационной конструкции}
                    \begin{scnsubdividing}
                        \scnitem{обозначение дискретной информационной конструкции, не принадлежащей SC-коду}
                        \scnitem{\scnkeyword{обозначение структуры}}
                        \begin{scnindent}
                            \scnidtf{обозначение дискретной информационной конструкции, принадлежащей SC-коду}
                            \scnidtf{обозначение sc-конструкции}
                        \end{scnindent}
                    \end{scnsubdividing}
                    \begin{scnsubdividing}
                        \scnitem{произвольная дискретная информационная конструкция}
                        \begin{scnindent}
                        	\scnidtf{sc-переменная, обозначающая дискретную информационную конструкцию}
                        \end{scnindent}
                        \scnitem{\scnkeyword{дискретная информационная структура}}
                        \begin{scnindent}
                            \scnidtf{sc-константа, обозначающая конкретную дискретную информационную конструкцию}
                        \end{scnindent}
                    \end{scnsubdividing}
                \end{scnsubstruct}
    
                \scnstructheader{Описание параметров и отношений, заданных на дискретных информационных конструкциях}
                \begin{scnsubstruct}
                    
                    \scnheader{параметр, заданный на множестве дискретных информационных конструкций\scnsupergroupsign}
                    \scnhaselement{типология дискретных информационных конструкций, определяемая их носителем\scnsupergroupsign}
                    \begin{scnindent}
                        \scnhaselement{некомпьютерная форма представления дискретных информационных конструкций\scnsupergroupsign}
                        \scnhaselement{файл}
                        \begin{scnindent}
                            \scnidtf{компьютерная форма предcтавления дискретных информационных конструкций в линейной адресной памяти}
                        \end{scnindent}
                        \scnhaselement{структура}
                        \begin{scnindent}
                            \scnidtf{компьютерная форма представления дискретных информационных конструкций в графодинамической ассоциативной памяти}
                            \scnidtf{представление дискретных информационных конструкций в виде конструкций SC-кода в памяти ostis-систем}
                        \end{scnindent}
                    \end{scnindent}
                    \scnhaselement{типология дискретных информационных конструкций, определяемая их соотношением с памятью ostis-систем\scnsupergroupsign}
                    \begin{scnindent}
                        \scnhaselement{внешняя дискретная информационная конструкция ostis-системы}
                        \begin{scnindent}
                            \scnidtf{дискретная информационная конструкция, которая находится вне памяти той ostis-системы, в которой находится sc-узел, обозначающий эту информационную конструкцию}
                            \begin{scnsubdividing}
                                \scnitem{некомпьютерная форма представления дискретных информационных конструкций}
                                \begin{scnindent}
                                    \scntext{примечание}{Очевидно, что информационные конструкции такого вида принципиально не могут быть внутренними информационными конструкциями ostis-систем, хранимыми в их памяти.}
                                \end{scnindent}
                                \scnitem{внешний файл ostis-системы}
                                \begin{scnindent}
                                    \begin{scnsubdividing}
                                        \scnitem{файл компьютерной системы, которая не является ostis-системой}
                                        \begin{scnindent}
                                            \scnidtf{файл, который не хранится в памяти данной ostis-системы, но о которой известно, какая система, не являющаяся ostis-системой, им \scnqq{владеет} и как его \scnqq{скачать}}
                                            \scnidtf{внешний файл ostis-системы, принадлежащий компьютерной системе, которая не является ostis-системой}
                                        \end{scnindent}
                                        \scnitem{файл другой ostis-системы}
                                        \begin{scnindent}
                                            \scnidtf{файл, который не является внутренним файлом данной ostis-системы, в памяти которой находится знак этого файла, но является внутренним знаком другой ostis-системы}
                                            \scnidtf{внешний файл ostis-системы, принадлежащий другой ostis-системе}
                                        \end{scnindent}
                                    \end{scnsubdividing}
                                \end{scnindent}
                                \scnitem{внешняя структура ostis-системы}
                                \begin{scnindent}
                                    \scnidtf{структура, хранимая в памяти другой ostis-системы}
                                    \scnidtf{структура другой ostis-системы}
                                \end{scnindent}
                            \end{scnsubdividing}
                        \end{scnindent}
                    \end{scnindent}
                    \bigskip
                    \scnhaselement{внутренняя информационная конструкция ostis-системы}
                    \begin{scnindent}
                        \scnidtf{внутренняя для заданной ostis-системы информационная конструкция}
                        \scnidtf{внутренняя информационная конструкция той ostis-системы, в памяти (sc-памяти) которой хранится знак (sc-узел) этой информационной конструкции}
                        \scntext{примечание}{Внутренние информационные конструкции ostis-систем (т.е. конструкции, обрабатываемые в их памяти) могут быть только дискретными, хотя и не обязательно знаковыми.}
                        \begin{scnsubdividing}
                            \scnitem{внутренний файл ostis-системы}
                            \scnitem{внутренняя структура}
                            \begin{scnindent}
                                \scnidtf{структура, которой в памяти данной ostis-системы соответствует не только знак этой структуры, но и она сама}
                                \scnidtf{структура, хранимая и обрабатываемая в памяти данной ostis-системы}
                                \scnidtf{внутренняя структура ostis-системы}
                            \end{scnindent}
                        \end{scnsubdividing}
                        \begin{scnsubdividing}
                            \scnitem{сформированная внутренняя информационная конструкция ostis-системы}
                            \scnitem{частично сформированная внутренняя информационная конструкция ostis-системы}
                            \scnitem{внутренняя информационная конструкция ostis-системы на начальной стадии формирования}
                        \end{scnsubdividing}
                    \end{scnindent}
                    \scnhaselement{типология дискретных информационных конструкций, определяемая правилами, которым они должны удовлетворять\scnsupergroupsign}
                    \scnhaselement{информационная конструкция Русского языка}
                    \scnhaselement{информационная конструкция Английского языка}
                    \scnhaselement{структура}
                    \begin{scnindent}
                        \scnidtf{информационная конструкция SC-кода}
                    \end{scnindent}
                    \scnhaselement{sc.g-конструкция}
                    \begin{scnindent}
                        \scnidtf{информационная конструкция SCg-кода}
                    \end{scnindent}
                    \scnhaselement{sc.s-конструкция}
                    \begin{scnindent}
                        \scnidtf{информационная конструкция SCs-кода}
                    \end{scnindent}
                    \scnhaselement{sc.n-конструкция}
                    \begin{scnindent}
                        \scnidtf{информационная конструкция SCn-кода}
                    \end{scnindent}
                    \scnhaselement{наличие синтаксической связности\scnsupergroupsign}
                    \begin{scnindent}
                        \scnhaselement{синтаксически связная дискретная информационная конструкция}
                        \begin{scnindent}
                            \scnidtftext{определение}{дискретная информационная конструкция, у которой для каждой пары её элементов существует маршрут, соединяющий эти элементы и проходящий по связям их инцидентности}
                        \end{scnindent}
                        \scnhaselement{синтаксически несвязная дискретная информационная конструкция}
                        \scntext{примечание}{Можно оценивать \scnqq{силу} синтаксической связности --- наличие и число \scnqq{мостов} в графе инцидентности элементов дискретной информационной конструкции, наличие и число точек \scnqq{сочленения}, минимальное число элементов конструкции, удаление которых приводит к несвязности. Можно также оценивать уровень синтаксической несвязности дискретной информационной конструкции числом компонентов связности этой конструкции.}
                    \end{scnindent}
                    \scnhaselement{наличие семантической связности\scnsupergroupsign}
                    \begin{scnindent}
                        \scntext{примечание}{Свойством семантической связности могут обладать только знаковые конструкции.}
                        \scnhaselement{семантически связная дискретная информационная конструкция}
                        \begin{scnindent}
                            \scntext{определение}{Это конструкция, которая обладает следующим свойством: для любой ее декомпозиции на два синтаксически правильных компонента всегда найдется пара синонимичных знаков, один из которых находится в одном компоненте, а другой --- в другом.}
                        \end{scnindent}
                        \scnhaselement{семантически несвязная дискретная информационная конструкция}
                    \end{scnindent}
                    
                    \scnheader{отношение, заданное на множестве дискретных информационных конструкций\scnsupergroupsign}
                    \scnhaselement{дискретная информационная конструкция заданного языка*}
                    \begin{scnindent}
                        \scniselement{отношение, заданное на множестве языков\scnsupergroupsign}
                        \begin{scnsubdividing}
                            \scnitem{синтаксически неправильная дискретная информационная конструкция заданного языка*}
                            \begin{scnindent}
                                \begin{scnreltoset}{объединение}
                                    \scnitem{синтаксически некорректная дискретная информационная конструкция заданного языка*}
                                    \scnitem{синтаксически нецелостная дискретная информационная конструкция заданного языка*}
                                \end{scnreltoset}
                            \end{scnindent}
                            \scnitem{\scnkeyword{текст заданного языка*}}
                            \begin{scnindent}
                                \begin{scnreltoset}{пересечение}
                                    \scnitem{синтаксически корректная дискретная информационная конструкция заданного языка*}
                                    \scnitem{синтаксически целостная дискретная информационная конструкция заданного языка*}
                                \end{scnreltoset}
                                \begin{scnsubdividing}
                                    \scnitem{семантически неправильный текст заданного языка*}
                                    \begin{scnindent}
                                        \begin{scnreltoset}{объединение}
                                            \scnitem{семантически некорректный текст заданного языка*}
                                            \scnitem{семантически нецелостный текст заданного языка*}
                                        \end{scnreltoset}
                                    \end{scnindent}
                                    \scnitem{знание, представленное в заданном языке*}
                                    \begin{scnindent}
                                        \begin{scnreltoset}{пересечение}
                                            \scnitem{семантически корректный текст заданного языка*}
                                            \scnitem{семантически целостный текст заданного языка*}
                                        \end{scnreltoset}
                                    \end{scnindent}
                                \end{scnsubdividing}
                            \end{scnindent}
                        \end{scnsubdividing}
                    \end{scnindent}
                    \scnsourcecommentinline{Завершили \textit{Описание параметров и отношений, заданных на дискретных информационных конструкциях}}
                    
                    \scnheader{обозначение файла}
                    \begin{scnsubdividing}
                        \scnitem{произвольный файл}
                        \begin{scnindent}
                        	\scnidtf{sc-переменная, каждым значением которой является обозначение файла}
                            \scnidtf{обозначение произвольного файла}
                            \scnidtf{sc-переменнная, обозначающая файл}
                        \end{scnindent}
                        \scnitem{\scnkeyword{файл}}
                        \begin{scnindent}
                        	\scnidtf{знак конкретного файла}
                            \scnidtf{sc-константа, обозначающая конкретный файл}
                        \end{scnindent}
                    \end{scnsubdividing}
                    \begin{scnsubdividing}
                        \scnitem{обозначение внешнего файла ostis-системы}
                        \begin{scnindent}
                            \begin{scnsubdividing}
                                \scnitem{произвольный внешний файл ostis-системы}
                                \scnitem{внешний файл ostis-системы}
                            \end{scnsubdividing}
                        \end{scnindent}
                        \scnitem{обозначение внутреннего файла ostis-системы}
                        \begin{scnindent}
                            \begin{scnsubdividing}
                                \scnitem{произвольный внутренний файл ostis-системы}
                                \scnitem{внутренний файл ostis-системы}
                            \end{scnsubdividing}
                        \end{scnindent}
                    \end{scnsubdividing}
                    
                    \scnheader{файл}
                    \scnidtf{sc-узел, обозначающий файл}
                    \scnidtf{знак файла}
                    \begin{scnsubdividing}
                        \scnitem{ея-файл}
                        \begin{scnindent}
                        	\scnidtf{естественно-языковой файл}
                        \end{scnindent}
                        \scnitem{файл, являющийся текстом формального языка}
                        \begin{scnindent}
                        	\scnsuperset{sc.g-файл}
                            \scnsuperset{sc.s-файл}
                            \scnsuperset{sc.n-файл}
                        \end{scnindent}
                        \scnitem{файл, отражающий процесс изменения sc.g-текста}
                        \scnitem{графический файл}
                        \scnitem{файл-изображение}
                        \scnitem{видео-файл}
                        \scnitem{аудио-файл}
                    \end{scnsubdividing}
                    \begin{scnsubdividing}
                        \scnitem{файл-экземпляр}
                        \begin{scnindent}
                        	\scnidtf{файл, являющийся конкретным электронным документом или электронным образом конкретной внешней информационной конструкции}
                        \end{scnindent}
                        \scnitem{файл-образец}
                        \begin{scnindent}
                        	\scnidtf{файл-класс ostis-системы}
                            \scnidtf{файл, являющийся одновременно также и знаком множества всевозможных экземпляров (копий) этого файла}
                        \end{scnindent}
                    \end{scnsubdividing}
                    \begin{scnsubdividing}
                        \scnitem{внешний файл ostis-системы}
                        \scnitem{\scnkeyword{внутренний файл ostis-системы}}
                    \end{scnsubdividing}
                    
                    \scnheader{внутренний файл ostis-системы}
                    \scniselement{синтаксически выделяемый класс sc-элементов в рамках SC-кода}
                    \scniselement{семантически выделяемый класс sc-элементов в рамках SC-кода}
                    \scntext{примечание}{Данный класс sc-элементов, являющихся знаками файлов, хранимых в памяти ostis-систем, в отличие от других синтаксически выделяемых классов sc-элементов, представляет собой одновременно  синтаксически и семантически выделяемый класс sc-элементов. Это обусловлено (1) тем, что каждый экземпляр данного класса sc-элементов является знаком конкретного файла, хранимого в памяти ostis-системы, и (2) тем, что каждый файл, хранимый в памяти ostis-системы, может и должен быть обозначен только таким sc-элементом, который является экземпляром рассматриваемого класса sc-элементов.}
                    \scntext{примечание}{sc-узел может быть знаком файла, находящегося в памяти другой ostis-системы (не в той, в которой хранится этот sc-узел). Но в этом случае указанный sc-узел не будет принадлежать рассматриваемому классу sc-узлов.}
                    \scnidtf{знак файла ostis-системы, хранимого в \scnqq{моей} памяти}
                    \scntext{примечание}{Следует отличать синтаксическую эквивалентность файлов, семантическую эквивалентность файлов и совпадение файлов (когда речь идет об одном и том же файле). Т.е. копия файла и один и тот же файл --- это разные вещи.}
                    
                \end{scnsubstruct}
                \scnstructheader{Классификация структур}
                \begin{scnsubstruct}
    
                    \scnheader{структура}
                    \scnidtf{структура, элементами которой являются sc-элементы}
                    \scnidtf{множество sc-элементов (множество), не являющиеся ни связкой (связкой sc-элементов), ни классом (множеством всех sc-элементов, эквивалентных в определенном смысле)}
                    \scnidtf{знак конкретной (константной) структуры}
                    \scnidtf{Класс всех тех и только тех sc-элементов, каждый из которых является знаком конкретной структуры}
                    \scnidtf{Знак класса всех sc-элементов, являющихся знаками конкретных структур}
                    \scnidtf{Константный sc-элемент (точнее, sc-узел), являющийся знаком конкретного класса всех sc-элементов, являющихся знаками конкретных структур}
                    \scnidtf{sc-конструкция}
                    \scnidtf{информационная конструкция, принадлежащая SC-коду}
                    \scnsuperset{\scnkeyword{sc-текст}}
                    \begin{scnindent}
                        \scnidtf{структура, удовлетворяющая синтаксическим правилам SC-кода}
                        \scnsuperset{\scnkeyword{знание}}
                        \begin{scnindent}
                            \scnidtf{семантически корректный и целостный sc-текст}
                        \end{scnindent}
                    \end{scnindent}        
                    \scnrelfrom{разбиение}{\scnkeyword{Наличие sc-переменных, входящих в состав структуры}}
                    \begin{scnindent}
                        \begin{scneqtoset}
                            \scnitem{структура, в составе которой sc-переменные не входят}
                            \scnitem{структура, в состав которой входят sc-переменные}
                            \begin{scnindent}
                                \scntext{примечание}{Такие структуры при представлении логических высказываний в SC-коде являются аналогами атомарных логических формул.}
                            \end{scnindent}
                        \end{scneqtoset}
                    \end{scnindent}
                    \scnrelfrom{разбиение}{\scnkeyword{Темпоральная характеристика структур}}
                    \begin{scnindent}
                        \begin{scneqtoset}
                            \scnitem{ситуативная структура}
                            \begin{scnindent}
                                \scnidtf{ситуация, представленная в SC-коде}
                                \scnidtf{ситуация}
                                \scnidtf{структура, в состав которой входят знаки временных сущностей и которая сама является временной сущностью (при этом время существования такой структуры совпадает с временем одновременного существования всех временных сущностей, знаки которых входят в состав этой ситуативной структуры).}
                                \begin{scnsubdividing}
                                    \scnitem{ситуация во внешней среде}
                                    \scnitem{ситуация в sc-памяти}
                                \end{scnsubdividing}
                            \end{scnindent}
                            \scnitem{структура, не содержащая знаков временных сущностей}
                            \scnitem{динамическая структура}
                            \begin{scnindent}
                                \scntext{пояснение}{В отличие от ситуативной структуры конфигурация динамической структуры меняется во времени в зависимости от момента появления и момента завершения существования каждой временной сущности (в том числе временной связи), знак которой входит в состав динамической структуры. Каждой динамической структуре можно поставить в соответствие темпоральную последовательность состояний (ситуаций) и событий.}
                            \end{scnindent}
                        \end{scneqtoset}
                        
                    \end{scnindent}
                    \scnrelfrom{разбиение}{\scnkeyword{Наличие связности структур}}
                    \begin{scnindent}
                        \begin{scneqtoset}
                            \scnitem{связная структура}
                            \begin{scnindent}
                                \scnrelfrom{разбиение}{Связность структур\scnsupergroupsign}
                                \begin{scnindent}
                                    \scnidtf{Минимальное число sc-элементов, удаление которых преобразует связную структуру в несвязную}
                                    \scniselement{одно-связная структура}
                                    \begin{scnindent}
                                        \scnrelto{разбиение}{Признак классификации структур по числу точек сочленения\scnsupergroupsign}
                                        \scnrelto{разбиение}{Признак классификации структур по числу мостов\scnsupergroupsign}
                                    \end{scnindent}
                                \end{scnindent}
                            \end{scnindent}
                            \scnitem{несвязная структура}
                            \begin{scnindent}
                                \scnrelto{разбиение}{Признак классификации структур по числу компонентов связности\scnsupergroupsign}
                                \scnsubset{тривиальная структура}
                                \begin{scnindent}
                                    \scnidtf{структура, в состав элементов которой sc-коннекторы не входят}
                                \end{scnindent}
                            \end{scnindent}
                        \end{scneqtoset}
                        \scntext{примечание}{Важнейшей особенностью SC-кода является то, что для конструкций SC-кода (для структур) нет необходимости противопоставлять синтаксическую и семантическую связность, то есть все синтаксически связные структуры являются также и семантически связными и наоборот.}
                    \end{scnindent}
                    \scnrelfrom{разбиение}{\scnkeyword{Рефлексивность структур}}
                    \begin{scnindent}
                        \begin{scneqtoset}
                            \scnitem{рефлексивная структура}
                            \begin{scnindent}
                                \scnidtf{структура, в число элементов которой входит sc-узел, обозначающий саму эту структуру}
                                \scnsubset{рефлексивное множество}
                            \end{scnindent}
                            \scnitem{нерефлексивная структура}
                        \end{scneqtoset}
                    \end{scnindent}
                    \scnrelfrom{разбиение}{\scnkeyword{Целостность структур по связкам}}
                    \begin{scnindent}
                        \begin{scneqtoset}
                            \scnitem{структура, содержащая все компоненты всех своих связок}
                            \scnitem{структура, не содержащая все компоненты всех своих связок}
                        \end{scneqtoset}
                    \end{scnindent}
                    \begin{scnsubdividing}
                        \scnitem{синтаксически неправильная структура}
                        \begin{scnindent}
                        	\scnidtf{синтаксически неправильно построенная структура}
                            \begin{scnreltoset}{объединение}
                                \scnitem{синтаксически некорректная структура}
                                \begin{scnindent}
                                    \scnidtf{структура, содержащая фрагменты, противоречащие \textit{Синтаксическим правилам SC-кода} (ошибочные фрагменты)}
                                \end{scnindent}
                                \scnitem{синтаксически нецелостная структура}
                                \begin{scnindent}
                                    \scnidtf{структура, в которой имеется синтаксически выявленная недостаточность, неполнота (то есть имеется некоторое количество информационных дыр)}
                                \end{scnindent}
                            \end{scnreltoset}
                            \scntext{примечание}{Разделение \textit{Синтаксических правил SC-кода} на правила анализа синтаксической корректности и правила анализа синтаксической целостности (полноты) существенно упрощает процедуру синтаксического анализа \textit{структур}.}
                        \end{scnindent}
                        \scnitem{\scnkeyword{sc-текст}}
                        \begin{scnindent}
                            \scnidtf{синтаксически правильная структура}
                            \scnidtf{синтаксически правильно построенная структура}
                            \begin{scnreltoset}{пересечение}
                                \scnitem{синтаксически корректная структура}
                                \scnitem{синтаксически целостная структура}
                            \end{scnreltoset}
                            \begin{scnsubdividing}
                                \scnitem{семантически неправильный sc-текст}
                                \begin{scnindent}
                                    \begin{scnreltoset}{объединение}
                                        \scnitem{семантически некорректный sc-текст}
                                        \scnitem{семантически нецелостный sc-текст}
                                    \end{scnreltoset}
                                \end{scnindent}
                                \scnitem{\scnkeyword{знание}}
                                \begin{scnindent}
                                    \scnidtf{семантически правильно построенный sc-текст}
                                    \begin{scnreltoset}{пересечение}
                                        \scnitem{семантически корректный sc-текст}
                                        \scnitem{семантически целостный sc-текст}
                                    \end{scnreltoset}
                                \end{scnindent}
                            \end{scnsubdividing}
                        \end{scnindent}
                    \end{scnsubdividing}
                    \scnsourcecommentinline{Завершили \textit{Классификацию структур}}
                    
                    \scnheader{знание}
                    \scnidtf{дискретная информационная конструкция, являющаяся знанием, представленная в некотором (дополнительно уточняемом) языке}
                    \scnrelto{второй домен}{знание, представленное в заданном языке*}
                    \scnsubset{знаковая конструкция}
                    \scntext{примечание}{Каждое знание является знаковой конструкцией, но не каждая знаковая конструкция является знанием, а только та, смысловое представление которой удовлетворяет определенным требованиям корректности и целостности.}
                    \scniselement{семантически выделяемый класс дискретных информационных конструкций\scnsupergroupsign}
                    
                    \scnheader{следует отличать*}
                    \begin{scnhaselementset}
                        \scnitem{\scnnonamednode}
                        \begin{scnindent}
                            \begin{scneqtoset}
                                \scnitem{дискретная информационная конструкция}
                                \scnitem{текст}
                                \scnitem{знание}
                            \end{scneqtoset}
                            \scniselement{следует отличать*}
                        \end{scnindent}
                        \scnitem{\scnnonamednode}
                        \begin{scnindent}
                            \begin{scneqtoset}
                                \scnitem{структура}
                                \scnitem{sc-текст}
                                \scnitem{знание}
                            \end{scneqtoset}
                            \scniselement{следует отличать*}
                        \end{scnindent}
                    \end{scnhaselementset}
                \end{scnsubstruct}
                \scnsourcecommentinline{Завершили представление \textit{Денотационной семантики SC-кода}}
                \bigskip
    
            \end{scnsubstruct}
            \scnendsegmentcomment
    
            \scnsegmentheader{Использование SC-кода для формального описания собственного синтаксиса}
            \begin{scneqtoset}
                \bigskip
                \scnfileitem{В предыдущем сегменте \scnqqi{\textit{SC-код как синтаксическое расширение Ядра SC-кода}} рассмотрен \textit{Синтаксис SC-кода} путём:
                \begin{scnitemize}
                    \item введения \textit{синтаксически выделяемых классов sc-элементов} в рамках \textit{SC-кода};
                    \item описания \textit{теоретико-множественных связей} между указанными классами \textit{sc-элементов} (к такому описанию, в частности, относится \textit{Синтаксическая классификация sc-элементов в рамках SC-кода});
                    \item введения двух отношений инцидентности \textit{sc-элементов} --- \textit{Отношения инцидентности sc-коннекторов*} и \textit{Отношения инцидентности входящих sc-дуг*};
                    \item описания \textit{Синтаксических правил SC-кода}, которые, прежде всего, описывают формальные свойства указанных выше отношений инцидентности \textit{sc-элементов}.
                \end{scnitemize}
                Однако для того, чтобы получить возможность \uline{все} (!) \textit{Синтаксические правила SC-кода} записать средствами самого \textit{SC-кода}, необходимо иметь \uline{явное} представление \textit{пар} отношений инцидентности \textit{sc-элементов} в виде \textit{sc-дуг}, принадлежащим этим отношениям. В случае, если указанные \textit{sc-дуги} инцидентности являются \textit{sc-переменными}, логико-семантических проблем не возникнет. И этого, кстати, вполне достаточно, чтобы \textit{Синтаксические правила SC-кода}, сформулированные в виде \textit{логических высказываний}, записать средствами \textit{SC-кода}. Но, если разрешить \textit{sc-дугам} инцидентности быть \textit{sc-константами}, то, во-первых, в \textit{Синтаксические правила SC-кода} необходимо добавить Правило удаления \textit{константной sc-дуги инцидентности}, если эта инцидентность представлена неявно, а, во-вторых, в \textit{Правила синтаксической трансформации sc-элементов} необходимо добавить Правило трансформации (замены) \textit{константной sc-дуги инцидентности} на неявное представление этой инцидентности.В теоретическом и, возможно, даже в практическом плане может быть интересна такая синтаксическая модификация (синтаксическое расширение) \textit{SC-кода}, в котором: 
                \begin{scnitemize}
                    \item \uline{все} неявно представленные \textit{пары инцидентности sc-элементов} заменяются на \textit{константные sc-дуги инцидентности} --- неявно представленными \textit{парами инцидентности} остаются \uline{только} \textit{пары инцидентности} константных \textit{sc-дуг} инцидентности с компонентами этих \textit{sc-дуг};
                    \item В \textit{Алфавит SC-кода} вводятся два новых \textit{синтаксически выделяемых класса sc-элементов} --- \textit{класс sc-дуг инцидентности sc-коннекторов}, а также \textit{класс sc-дуг инцидентности входящих sc-дуг}.
                \end{scnitemize}
                В результате такого преобразования конструкций \textit{SC-кода} конструкции \textit{SC-кода} перестают быть графовыми конструкциями нетрадиционного вида, в которых рёбра, гиперрёбра, дуги могут быть инцидентны другим рёбрам, гиперребрам и дугам, а становятся классическими графами с двумя типами дуг (с \textit{sc-дугами инцидентности sc-коннекторов} и с \textit{sc-дугами инцидентности входящих sc-дуг}) и с пятью типами вершин (с вершинами, представляющими \textit{sc-узлы общего вида}, с вершинами, представляющими \textit{sc-узлы}, являющиеся знаками \textit{внутренних файлов ostis-системы}, с вершинами, представляющими \textit{sc-рёбра общего вида}, с вершинами, представляющими \textit{sc-дуги общего вида}, с вершинами, представляющими \textit{базовые sc-дуги}).Рассмотренное преобразование конструкций \textit{SC-кода} в теории графов называется поздразделением или подразбиением графа (Смотрите \cite{Trudeau1993}).}
            \end{scneqtoset}
            \scnendsegmentcomment
            \bigskip
        \end{scnsubstruct}
    \scnendcurrentsectioncomment
    
    \end{SCn}
    \scsubsubsection{Предметная область и онтология синтаксиса внутреннего языка ostis-систем}\label{sd_sc_code_syntax}
    \scsubsubsection{Предметная область и онтология базовой денотационной семантики внутреннего языка ostis-систем}\label{sd_sc_code_semantic}