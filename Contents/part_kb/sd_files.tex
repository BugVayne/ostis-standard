\begin{SCn}
    \scnsectionheader{\currentname}
    \begin{scnsubstruct}
    	%не нашла такой ПО в стандарте
        \scnheader{Предметная область файлов, внешних информационных конструкций и внешних языков ostis-систем}
        \scniselement{предметная область}
        \begin{scnhaselementrole}{максимальный класс объектов исследования}
            {файл}
        \end{scnhaselementrole}
        \begin{scnhaselementrolelist}{класс объектов исследования}
            \scnitem{внешний язык}
            \scnitem{естественный язык}
            \scnitem{Русский язык}
            \scnitem{Английский язык}
            \scnitem{изображение}
            \scnitem{класс синтаксически эквивалентных информационных конструкций}
            \scnitem{максимальный класс синтаксически эквивалентных информационных конструкций}
        \end{scnhaselementrolelist}
        \begin{scnhaselementrolelist}{исследуемое отношение}
            \scnitem{трансляция sc-текста*}
        \end{scnhaselementrolelist}
        
        \scnheader{файл}
        \scnidtf{знак файла}
        \scnidtf{sc-знак файла}
        \scnidtf{знак информационной конструкции, внешней по отношению к sc-памяти}
        \scnidtf{sc-ссылка}
        \scntext{пояснение}{Под \textbf{\textit{файлом}} понимается любая информационная конструкция, внешняя по отношению к \textit{sc-памяти}, т.е., не являющаяся \textit{sc-текстом}. При этом каждому \textbf{\textit{файлу}} может быть поставлен в соответствие семантически эквивалентный \textit{sc-текст}.}
        
        \scnheader{внешний язык}
        \scnidtf{язык, внешний по отношению к sc-памяти}
        \scnrelto{семейство подмножеств}{файл}
        \scntext{пояснение}{Под \textbf{\textit{внешним языком}} понимается множество \textit{файлов}, имеющих общую синтаксическую структуру.}
        
        \scnheader{естественный язык}
        \scnidtf{язык диалога с пользователем}
        \scnsubset{внешний язык}
        \scnhaselement{Русский язык}
        \scnhaselement{Английский язык}
        \scntext{пояснение}{Под конкретным \textbf{\textit{естественным языком}} понимается некоторое множество \textit{файлов} (например, идентификаторов, естественно-языковых пояснений и т.д.), которые используются при диалоге с тем или иным пользователем, режим ведения которого он может выбрать. В этом смысле некоторые фрагменты, такие как, например обозначения \textbf{sin}, \textbf{cos}, \textbf{a.e.} и т.п., могут входить в несколько \textbf{\textit{естественных языков}}, поскольку используются при диалоге, но исторически являться фрагментами другого языка.}
        
        \scnheader{Русский язык}
        
        \scnheader{Английский язык}
        
        \scnheader{трансляция sc-текста*}
        \scntext{пояснение}{Связки отношения \textbf{\textit{трансляция sc-текста*}} связывают некоторый \textit{sc-текст} и \textit{файл}, который является семантическим эквивалентом этого \textit{sc-текста} на некотором внешнем языке (в том числе, например, языке геометрических чертежей, математических формул и т.д.).}
        
        \scnheader{изображение}
        \scnidtf{графический файл}
        \scnidtf{графическая несимвольная информационная конструкция}
        \scnsubset{файл}
        
        \scnheader{класс синтаксически эквивалентных информационных конструкций}
        \scnsuperset{максимальный класс синтаксически эквивалентных информационных конструкций}
        \scntext{пояснение}{\textbf{\textit{класс синтаксически эквивалентных информационных конструкций}} - \textit{класс} информационных конструкций имеющих общие синтаксические свойства без учета различий в форматах их кодирования, в используемых шрифтах, в форматировании и размещении.}
        \scnheader{максимальный класс синтаксически эквивалентных информационных конструкций}
        \scntext{пояснение}{\textbf{\textit{максимальный класс синтаксически эквивалентных информационных конструкций}} - \textit{класс} всевозможных конструкций, имеющих общие синтаксические свойства без учета различий в форматах их кодирования, в используемых шрифтах, в форматировании и размещении.}
    \end{scnsubstruct}
    \scnendcurrentsectioncomment
\end{SCn}
