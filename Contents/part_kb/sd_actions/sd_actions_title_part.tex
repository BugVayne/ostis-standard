\begin{SCn}
    \scniselement{раздел}
    \scniselement{предметная область и онтология}
    \begin{scnreltovector}{конкатенация сегментов}
        \scnitem{Уточнение понятия воздействия и понятия действия. Типология воздействий и действий}
        \scnitem{Уточнение понятия задачи. Типология задач}
        \scnitem{Уточнение семейства параметров и отношений, заданных на множестве воздействий, действий и задач}
        \scnitem{Предметная область и онтология субъектно-объектных спецификаций воздействий}
        \scnitem{Уточнение понятий плана сложного действия, класса задач, метода}
        \scnitem{Уточнение понятия навыка, понятия класса методов и понятия модели решения задач}
        \scnitem{Уточнение понятия деятельности, понятия вида деятельности и понятия технологии}
    \end{scnreltovector}
    \bigskip
    \begin{scnhaselementrolelist}{исследуемый класс первичных объектов исследования}
        \scnitem{воздействие}
        \begin{scnindent}
            \scnidtf{\textit{процесс} воздействия одних \textit{сущностей} на другие}
            \scnsubset{процесс}
        \end{scnindent}
        \scnitem{действие}
        \begin{scnindent}
            \scnidtf{\textit{процесс}, \scnqq{осознанно} и целенаправленно выполняемый (управляемый) некоторой \textit{кибернетической системой}}
            \scnsubset{воздействие}
            \scnsubset{процесс}
        \end{scnindent}
        \bigskip
        \scnitem{неосознанное воздействие}
        \bigskip
        \scnitem{действие, выполняемое в памяти субъекта действия}
        \scnitem{действие, выполняемое во внешней среде субъекта действия}
        \scnitem{рецепторное действие}
        \begin{scnindent}
            \scnidtf{действие, выполняемое рецептором субъекта действия}
        \end{scnindent}
        \scnitem{эффекторное действие}
        \begin{scnindent}
            \scnidtf{действие, выполняемое эффектором субъекта действия}
        \end{scnindent}
        \bigskip
        \scnitem{элементарное действие}
        \begin{scnindent}
            \scnidtf{действие, выполнение которого не требует его декомпозиции на взаимосвязанные поддействия}
        \end{scnindent}
        \scnitem{сложное действие}
        \scnitem{легко выполнимое сложное действие}
        \begin{scnindent}
            \scnidtf{сложное действие, которое известно, как выполнять}
        \end{scnindent}
        \scnitem{интеллектуальное действие}
        \begin{scnindent}
            \scnidtf{сложное действие, для которого априори не известно, как его выполнять}
        \end{scnindent}
        \scnitem{индивидуальное действие}
        \begin{scnindent}
            \scnidtf{действие, выполняемое индивидуальной кибернетической системой}
        \end{scnindent}
        \scnitem{коллективное действие}
        \begin{scnindent}
            \scnidtf{действие, выполняемое коллективом кибернетических систем (многоагентной системой)}
        \end{scnindent}
        \bigskip
        \scnitem{планируемое действие}
        \scnitem{инициированное действие}
        \scnitem{выполняемое действие}
        \begin{scnindent}
            \scnidtf{активное действие}
        \end{scnindent}
        \scnitem{прерванное действие}
        \begin{scnindent}
            \scnidtf{выполняемое действие, находящееся в состоянии прерывания}
        \end{scnindent}
        \scnitem{выполненное действие}
        \scnitem{отмененное действие}
        \scnitem{действие с очень высоким приоритетом}
        \scnitem{действие с высоким приоритетом}
        \scnitem{действие со средним приоритетом}
        \scnitem{действие с низким приоритетом}
        \scnitem{действие с очень низким приоритетом}
    \end{scnhaselementrolelist}
    \bigskip
    \begin{scnhaselementrolelist}{исследуемый класс классов первичных объектов исследования}
        \scnitem{осмысленность воздействия\scnsupergroupsign}
        \scnitem{длительность воздействия\scnsupergroupsign}
        \scnitem{место выполнения действия\scnsupergroupsign}
        \scnitem{функциональная сложность действия\scnsupergroupsign}
        \scnitem{многоагентность действия\scnsupergroupsign}
        \begin{scnindent}
            \scnidtf{коллективность субъекта действия}
        \end{scnindent}
        \scnitem{текущее состояние действия\scnsupergroupsign}
        \scnitem{приоритет действия\scnsupergroupsign}
        \begin{scnindent}
            \scnidtf{важность действия\scnsupergroupsign}
        \end{scnindent}
        \scnitem{срочность действия\scnsupergroupsign}
        \bigskip
        \scnitem{класс действий}
        \scnitem{класс функционально эквивалентных действий\scnsupergroupsign}
        \scnitem{класс логически эквивалентных действий\scnsupergroupsign}
        \scnitem{класс семантических эквивалентных задач\scnsupergroupsign}
        \scnitem{класс логически эквивалентных задач\scnsupergroupsign}
        \scnitem{класс задач, для которого существует общий метод их решения\scnsupergroupsign}
        \scnitem{класс аналогичных семантически элементарных процессов воздействия\scnsupergroupsign}
        \begin{scnindent}
            \scnidtf{класс однотипных семантически элементарных воздействий\scnsupergroupsign}
        \end{scnindent}
    \end{scnhaselementrolelist}
    \bigskip
    \begin{scnhaselementrolelist}{исследуемый класс классов}
        \scnitem{отношение, заданное на множестве* (действие)}
        \begin{scnindent}
            \scnidtf{отношение, заданное на множестве действий}
        \end{scnindent}
        \scnitem{отношение, заданное на множестве* (задача)}
        \scnitem{параметр, заданный на множестве* (действие)}
        \scnitem{параметр, заданный на множестве* (задача)}
    \end{scnhaselementrolelist}
    \scnsourcecomment{Здесь указаны классы классов, которые не являются классами классов \uline{первичных} объектов исследования}
    \bigskip
    \begin{scnhaselementrolelist}{исследуемое отношение, заданное на множестве первичных объектов исследования}
        \scnitem{воздействующая сущность\scnrolesign}
        \scnitem{воздействуемая сущность\scnrolesign}
        \scnitem{посредник\scnrolesign}
        \scnitem{медиатор\scnrolesign}
        \scnitem{субъект\scnrolesign}
    	\begin{scnindent}
            \scnidtf{быть субъектом заданного действия}
        \end{scnindent}
        \scnitem{спецификация воздействия*}
    	\begin{scnindent}
            \scnsuperset{спецификация действия*}
        \end{scnindent}
        \scnitem{спецификация действия*}
    	\begin{scnindent}
            \scnsuperset{задача*}
            \begin{scnindent}
	            \begin{scnsubdividing}
		            \scnitem{декларативная формулировка задачи*}
		            \scnitem{процедурная формулировка задачи*}
	            \end{scnsubdividing}
	        \end{scnindent}
            \scnsuperset{план сложного действия*}
            \scnsuperset{декларативная спецификация выполнения сложного действия*}
            \scnsuperset{протокол*}
            \scnsuperset{результативная часть протокола*}
        \end{scnindent}
        \scnitem{декларативная формулировка задачи*}
        \scnitem{процедурная формулировка задачи*}
        \scnitem{план сложного действия*}
        \scnitem{декларативная спецификация выполнения сложного действия*}
        \scnitem{протокол*}
        \scnitem{результативная часть протокола*}
    \end{scnhaselementrolelist}
    \bigskip
    \begin{scnhaselementrolelist}{исследуемое отношение}
        \scnitem{спецификация класса действий*}
        \scnitem{спецификация метода*}
        \scnitem{спецификация класса методов*}
        \scnitem{спецификация деятельности*}
        \scnitem{спецификация вида деятельности*}
    \end{scnhaselementrolelist}
    \begin{scnindent}
    	\scnsourcecomment{Здесь указаны исследуемые отношения, которые заданы не на множестве первичных объектов исследования}	
    \end{scnindent}
    \bigskip
    \begin{scnhaselementrolelist}{исследуемый класс структур, специфицирующих первичные объекты исследования}
        \scnitem{задача}
    	\begin{scnindent}
            \scnidtf{формулировка задачи}
            \scnidtf{спецификация действия}
            \scnidtf{структура (sc-конструкция), содержащая в идеале достаточную информацию для выполнения соответствующего (специфицируемого) действия}
        \end{scnindent}
        \scnitem{декларативная формулировка задачи}
        \begin{scnindent}
            \scnidtf{семантическая спецификация действия}
        \end{scnindent}
        \scnitem{процедурная формулировка задачи}
        \begin{scnindent}
            \scnidtf{функциональная спецификация действия}
        \end{scnindent}
        \scnitem{план сложного действия}
        \begin{scnindent}
            \scnidtf{план выполнения сложного действия}
        \end{scnindent}
        \scnitem{процедурный план сложного действия}
        \scnitem{непроцедурный план сложного действия}
        \begin{scnindent}
            \scnidtf{декларативный план сложного действия}
            \scnidtf{иерархическая система подзадач заданной сложной задачи}
        \end{scnindent}
    \end{scnhaselementrolelist}
    \bigskip
    \begin{scnhaselementrolelist}{исследуемый класс структур}
        \scnitem{метод}
        \begin{scnindent}
            \scnidtf{спецификация класса сложных действий}
        \end{scnindent}
        \scnitem{денотационная семантика метода}
        \scnitem{операционная семантика метода}
        \scnitem{навык}
        \scnitem{модель решения задач}
        \scnitem{технология}
    \end{scnhaselementrolelist}
    \begin{scnindent}
    	\scnsourcecomment{Здесь указаны классы структур, не являющихся спецификациями первичных объектов исследования}
    \end{scnindent}
    \bigskip
    \begin{scnhaselementrolelist}{вводимое, но не исследуемое понятие}
        \scnitem{действие, выполняемое в памяти ostis-системы}
        \scnitem{действие, выполняемое ostis-системой в своей внешней среде}
        \scnitem{рецептурное действие ostis-системы}
        \scnitem{эффекторное действие ostis-системы}
        \scnitem{sc-агент}
    	\begin{scnindent}
            \scnidtf{внутренний субъект ostis-системы}
            \scnidtf{субъект, реализующий действия, выполняемые в памяти ostis-системы}
        \end{scnindent}
    \end{scnhaselementrolelist}
    \bigskip
    \begin{scnhaselementrolelist}{используемое понятие, исследуемое в другой предметной области и онтологии}
        \scnitem{кибернетическая система}
        \begin{scnindent}
            \scnidtf{сущность, обладающая способностью быть субъектом различного вида действий}
        \end{scnindent}
        \scnitem{компьютерная система}
        \begin{scnindent}
            \scnidtf{искусственная кибернетическая система}
            \scnsubset{кибернетическая система}
        \end{scnindent}
        \scnitem{интеллектуальная компьютерная система}
        \begin{scnindent}
            \scnsubset{компьютерная система}
            \scnsuperset{ostis-система}
        \end{scnindent}
        \scnitem{человек}
        \begin{scnindent}
            \scnsubset{кибернетическая система}
        \end{scnindent}
        \scnitem{ostis-система}
        \scnitem{спецификация*}
        \begin{scnindent}
            \scnidtf{быть спецификацией (описанием, семантической окрестностью заданной сущности*)}
            \scnidtf{семантическая окрестность*}
        \end{scnindent}
    \end{scnhaselementrolelist}
   
    \scnheader{следует отличать*}
    \begin{scnhaselementset}
        \scnitem{\scnnonamednode}
        \begin{scnindent}
            \begin{scneqtovector}
                \scnitem{действие}
                \scnitem{класс действий}
            \end{scneqtovector}
        \end{scnindent}
        \scnitem{\scnnonamednode}
        \begin{scnindent}
            \begin{scneqtovector}
                \scnitem{метод}
                \scnitem{класс методов}
            \end{scneqtovector}
        \end{scnindent}
        \scnitem{\scnnonamednode}
        \begin{scnindent}
            \begin{scneqtovector}
                \scnitem{деятельность}
                \scnitem{вид деятельности}
            \end{scneqtovector}
        \end{scnindent}
    \end{scnhaselementset}
    \begin{scnindent}
	    \scnsubset{семейство подклассов*}
	    \scntext{примечание}{Все сущности, принадлежащие рассмотренным \textit{понятиям}, требуют достаточно детальной \textit{спецификации}. При этом не следует путать сами сущности и их \textit{спецификации}. Так, например, не следует путать \textit{действие} и \textit{задачу}, которая специфицирует (уточняет) это \textit{действие}. Особое место среди указанных понятий занимает понятие \textit{метода}, т.к. каждый конкретный \textit{метод}, с одной стороны, является \textit{спецификацией} соответствующего \textit{класса действий}, а, с другой стороны, сам нуждается в \textit{спецификации}, которая уточняет либо \textit{декларативную семантику} этого \textit{метода} (т.е. обобщенную декларативную формулировку класса задач, решаемых с помощью этого \textit{метода}), либо \textit{операционную семантику} этого \textit{метода}, (т.е. множество \textit{методов}, обеспечивающих \textit{интерпретацию} данного специфицируемого \textit{метода}) и тем самым преобразует специфицируемый \textit{метод} в \textit{навык}.}
    \end{scnindent}
    	
    \scnheader{следует отличать*}
    \begin{scnhaselementvector}
        \scnitem{первый домен*(спецификация*)}
        \begin{scnindent}
            \scnidtf{специфицируемая сущность}
            \scnidtf{сущность, использование которой требует вполне определенной ее спецификации}
            \scnsuperset{действие}
            \scnsuperset{класс действий}
            \scnsuperset{метод}
            \scnsuperset{класс методов}
            \scnsuperset{деятельность}
            \scnsuperset{вид деятельности}
        \end{scnindent}
        \scnitem{второй домен*(спецификация*)}
    	\begin{scnindent}
            \scnidtf{спецификация}
            \scnsuperset{задача}
            \begin{scnindent}
	            \scnsuperset{декларативная формулировка задачи}
	            \begin{scnindent}
	            	\scnidtf{семантическая формулировка задачи}
	            \end{scnindent}
	            \scnsuperset{процедурная формулировка задачи}
	            \begin{scnindent}
	            	\scnidtf{функциональная формулировка задачи}
	            \end{scnindent}
	        \end{scnindent}
            \scnsuperset{план действия}
            \begin{scnindent}
	            \scnidtf{план}
	            \scnidtf{план выполнения действия}
	    	\end{scnindent}
            \scnsuperset{декларативная спецификация выполнения действий}
            \begin{scnindent}
            	\scnidtf{иерархическая система подзадач}
            \end{scnindent}
            \scnsuperset{протокол}
            \scnsuperset{результативная часть протокола}
            \scnsuperset{обобщенная декларативная формулировка класса задач}
            \scnsuperset{метод}
            \scnsuperset{декларативная семантика метода}
            \scnsuperset{операционная семантика метода}
            \scnsuperset{модель решения задач}
        \end{scnindent}
    \end{scnhaselementvector}
	\begin{scnindent}
    	\scntext{примечание}{При этом следует отличать:
	        \begin{scnitemize}
	            \item спецификацию конкретного \textit{действия} (\textit{задачу}, \textit{план}, \textit{декларативную спецификацию выполнения действия}, \textit{протокол}, \textit{результативную часть протокола});
	            \item спецификацию конкретной \textit{деятельности} (\textit{контекст}*, \textit{множество используемых методов}*);
	            \item спецификацию \textit{класса действий} (\textit{обобщенную декларативную формулировку класса задач}, \textit{метод});
	            \item спецификацию \textit{вида деятельности} (\textit{технологию});
	            \item спецификацию \textit{метода} (\textit{декларативную семантику метода}, \textit{операционную семантику метода});
	            \item спецификацию \textit{класса методов} (\textit{модель решения задач}).
	        \end{scnitemize}}
    \end{scnindent}
    
    \scnheader{следует отличать*}
    \begin{scnhaselementset}
        \scnitem{\scnnonamednode}
        \begin{scnindent}
            \begin{scneqtovector}
		      \scnitem{действие}
		      \scnitem{класс действий, метод}
            \end{scneqtovector}
        \end{scnindent}
        \scnitem{\scnnonamednode}
        \begin{scnindent}
            \begin{scneqtovector}
             \scnitem{деятельность}
             \scnitem{вид деятельности, технология}
            \end{scneqtovector}
        \end{scnindent}
    \end{scnhaselementset}
    
    \bigskip
    
    %в стандарте нет этого
    \begin{scnrelfromlist}{библиографический источник}
        \scnitem{\cite{Martynov1984}}
        \begin{scnindent}
            \scnciteannotation{Martynov1984}
    	\end{scnindent}
        \scnitem{\cite{Ikeda1998}}
    	\begin{scnindent}
            \scnciteannotation{Ikeda1998}
            \scnrelfrom{ключевой знак}{онтология классов задач}
        	\begin{scnindent}
	            \scnidtf{задачная онтология}
	            \scnidtf{онтология классов задач, решаемых в данной предметной области}
        	\end{scnindent}
        \end{scnindent}
        \scnitem{\cite{Studer1996}}
        \begin{scnindent}
            \scnciteannotation{Studer1996}
     	\end{scnindent}
        \scnitem{\cite{Benjamins1999}}
        \begin{scnindent}
            \scnciteannotation{Benjamins1999}
        \end{scnindent}
        \scnitem{\cite{Chandrasekaran1999}}
        \begin{scnindent}
            \scnciteannotation{Chandrasekaran1999}
        \end{scnindent} 
        \scnitem{\cite{Chandrasekaran1998}}
        \begin{scnindent}
            \scnciteannotation{Chandrasekaran1998}
        \end{scnindent}
        \scnitem{\cite{Fensel1998Reuse}}
        \begin{scnindent}
            \scnciteannotation{Fensel1998Reuse}
        \end{scnindent}
        \scnitem{\cite{Kemke2001}}
        \begin{scnindent}
            \scnciteannotation{Kemke2001}
    	\end{scnindent}
        \scnitem{\cite{Tu1995}}
        \begin{scnindent}
            \scnciteannotation{Tu1995}
        \end{scnindent}
        \scnitem{\cite{Trypuz2007}}
        \begin{scnindent}
            \scnciteannotation{Trypuz2007}
        \end{scnindent}
        \scnitem{\cite{Fang2019}}
        \begin{scnindent}
            \scnciteannotation{Fang2019}
        \end{scnindent}
        \scnitem{\cite{Fensel1997}}
        \scnitem{\cite{McBride2021}}
        \begin{scnindent}
            \scnciteannotation{McBride2021}
        \end{scnindent}
        \scnitem{\cite{Crowther2020}}
        \begin{scnindent}
            \scnciteannotation{Crowther2020}
        \end{scnindent}
        \scnitem{\cite{McCann1998}}
        \begin{scnindent}
            \scnciteannotation{McCann1998}
        \end{scnindent}
        \scnitem{\cite{Yan2014}}
        \begin{scnindent}
            \scnciteannotation{Yan2014}
        \end{scnindent}
        \scnitem{\cite{Ansari2018}}
        \scnitem{\cite{Crubezy2004}}
        \begin{scnindent}
            \scnciteannotation{Crubezy2004}
        \end{scnindent}
        \scnitem{\cite{Coelho1996}}
    \end{scnrelfromlist}
\end{SCn}
