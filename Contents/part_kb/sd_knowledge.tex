\begin{SCn}
    \scnsectionheader{\currentname}
    \begin{scnsubstruct}
        \begin{scnrelfromlist}{дочерний раздел}
            \scnitem{Предметная область и онтология множеств
                ~\\\scnidtf{Предметная область и онтология \textit{знаний о множествах}}
                \scntext{примечание}{\textit{знания о множествах} являются \uline{частным видом} \textit{знаний} и, следовательно, общие свойства сущностей, описываемых знаниями, могут наследоваться \textit{Предметной областью и онтологией множеств}}}
            \scnitem{Предметная область и онтология связок и отношений}
            \scnitem{Предметная область и онтология параметров, величин и шкал}
            \scnitem{Предметная область и онтология чисел и числовых структур}
            \scnitem{Предметная область и онтология структур}
            \scnitem{Предметная область и онтология темпоральных сущностей}
            \scnitem{Предметная область и онтология темпоральных сущностей баз знаний ostis-систем}
            \scnitem{Предметная область и онтология семантических окрестностей}
            \scnitem{Предметная область и онтология предметных областей}
            \scnitem{Предметная область и онтология онтологий}
            \scnitem{Предметная область и онтология логических формул, высказываний и формальных теорий}
            \scnitem{Предметная область и онтология внешних информационных конструкций и файлов ostis-систем}
            \scnitem{Глобальная предметная область действий и задач и соответствующая ей онтология методов и технологий}
        \end{scnrelfromlist}
        \scnheader{Предметная область знаний и баз знаний ostis-систем}
        \scniselement{предметная область}
        \begin{scnhaselementrolelist}{максимальный класс объектов исследования}
            \scnitem{знание}
        \end{scnhaselementrolelist}
        \begin{scnhaselementrolelist}{исследуемый класс классов}
            \scnitem{вид знаний}
            \scnitem{отношение, заданное на множестве знаний}
        \end{scnhaselementrolelist}
        \scnheader{знание}
        \scnidtf{синтаксически корректная (для соответствующего языка) и семантически целостная информационная конструкция}
        \scnsubset{информационная конструкция}
        \scniselementrole{класс объектов исследования}{\nameref{intro_lang}}
        \scnrelfrom{покрытие}{вид знаний
            ~\\\scnidtf{Множество \uline{всевозможных} видов знаний}
            \scntext{примечание}{Тот факт, что семейство \textit{видов знаний} является \textit{покрытием} Множества всевозможных \textit{знаний}, означает то, что каждое \textit{знание} принадлежит по крайней мере одному выделенному нами \textit{виду знаний}}}
        \scnheader{вид знаний}
        \scnhaselement{спецификация}
        \scnidtf{описание заданной сущности}
        \scnsuperset{спецификация материальной сущности}
        \scnsuperset{спецификация обратной сущности, не являющейся множеством}
        \scnsuperset{спецификация геометрической точки}
        \scnsuperset{спецификация числа}
        \scnsuperset{спецификация множества}
        \scnsuperset{спецификация связи}
        \scnsuperset{спецификация структуры}
        \scnsuperset{спецификация класса}
        \scnsuperset{спецификация класса сущностей, не являющихся множествами}
        \scnsuperset{спецификация отношения}
        \scnidtf{спецификация класса связей (связок)}
        \scnsuperset{спецификация класса классов}
        \scnsuperset{спецификация параметра}
        \scnsuperset{спецификация класса структур}
        \scnsuperset{спецификация понятий}
        \scnsuperset{пояснение}
        \scnsuperset{определение}
        \scnsuperset{утверждение}
        \scnidtf{утверждение, описывающее свойства экземпляров (элементов) специфицируемого понятия}
        \scnidtf{закономерность}
        \scnsuperset{семантическая окрестность}
        \scnsuperset{однозначная спецификация}
        \scnsuperset{сравнительный анализ}
        \scnsuperset{достоинства}
        \scnsuperset{недостатки}
        \scnsuperset{структура специфицируемой сущности}
        \scnsuperset{принципы, лежащие в основе}
        \scnsuperset{обоснование предлагаемого решения}
        \scnidtf{аргументация предлагаемого решения}
        \scnhaselement{сравнение}
        \scnhaselement{высказывание}
        \scnsuperset{фактографическое высказывание}
        \scnsuperset{закономерность}
        \scnhaselement{формальная теория}
        \scnhaselement{предметная область}
        \scnhaselement{предметная область и онтология
            ~\\\scnidtf{предметная область и её онтология}
            ~\\\scnidtf{предметная область и соответствующая ей объединенная онтология}
        }
        \scnhaselement{метазнание}
        \scnidtf{спецификация знания}
        \scnsuperset{аннотация}
        \scnsuperset{введение}
        \scnsuperset{предисловие}
        \scnsuperset{заключение}
        \scnsuperset{онтология}
        \scnsuperset{онтология предметной области}
        \scnsuperset{структурная онтология предметной области}
        \scnsuperset{теоретико-множественная онтология предметной области}
        \scnsuperset{логическая онтология предметной области}
        \scnsuperset{терминологическая онтология предметной области}
        \scnsuperset{объединенная онтология предметной области}
        \scnhaselement{задача}
        \scnidtf{спецификация действия}
        \scnhaselement{план}
        \scnhaselement{протокол}
        \scnhaselement{результативная часть протокола}
        \scnhaselement{метод}
        \scnhaselement{технология}
        \scnhaselement{история использования предметной области и её онтологии по решению информационных задач}
        \scnhaselement{история использования предметной области и её онтологии по решению задач во внешней среде}
        \scnhaselement{история эволюции предметной области и её онтологии}
        \scnhaselement{база знаний}
        \scnidtf{совокупность знаний, хранимых в памяти интеллектуальной компьютерной системы и \uline{достаточных} для того, чтобы указанная система удовлетворяла соответствующим предъявляемым к ней требованиям (в частности, чтобы она имела соответствующий уровень интеллекта)}
        \scnidtf{систематизированная совокупность знаний, хранимая в памяти интеллектуальной компьютерной системы и достаточная для обеспечения целенаправленного (целесообразного, адекватного) функционирования (поведения) этой системы как в своей внешней среде, так и в своей внутренней среде (в собственной базе знаний)}
        \begin{scnrelfromset}{обобщенная декомпозиция}
            \scnitem{согласованная часть базы знаний
                ~\\\scnidtf{часть базы знаний, признанная коллективом авторов на текущий момент}
            }
            \scnitem{история эксплуатации базы знаний}
            \scnitem{история эволюции базы знаний}
            \scnitem{план эволюции базы знаний
                ~\\\scnidtf{система специфицированных и согласованных действий авторов базы знаний, направленных на повышение её качества}
            }
        \end{scnrelfromset}
        \scntext{примечание}{Основным факторами, определяющими качество интеллектуальной компьютерной системы, являются:
            \begin{scnitemize}
                \item качественная структуризация (систематизация) и \uline{стратификация} базы знаний интеллектуальной компьютерной системы, а также
                \item систематизация и стратификация \uline{деятельности}, которая осуществляется интеллектуальной компьютерной системой и спецификация которой является важнейшей частью базы знаний этой системы (Смотрите Раздел \textit{Глобальная предметная область действий и задач и соответствующая ей онтология методов и технологий}).
            \end{scnitemize}
        }
        \scntext{примечание}{Даже небольшой перечень \textit{видов знаний} свидетельствует об огромном многообразии \textit{видов знаний}}
        \scnheader{знание}
        \begin{scnsubdividing}
            \scnitem{декларативное знание
                ~\\\scnidtf{\textit{знание}, имеющее \uline{только} \textit{денотационную семантику}, которая представляется в виде семантической \textit{спецификации} системы \textit{понятий}, используемых в этом \textit{знании}}
            }
            \scnitem{процедурное знание
                ~\\\scnidtf{\textit{знание}, имеющее не только \textit{денотационную семантику}, но и \textit{операционную семантику}, которая представляется в виде семейства \textit{спецификаций агентов}, осуществляющих интерпретацию \textit{процедурного знания}, направленную на решение некоторой инициированной \textit{задачи}}
                \scnidtf{функционально интерпретируемое знание, обеспечивающее решение либо конкретной задачи, либо некоторого множества инициируемых задач}
                \scnsuperset{задача}
                \scnidtf{формулировка конкретной задачи}
                \scnsuperset{декларативная формулировка задачи}
                \scnsuperset{процедурная формулировка задачи}
                \scnsuperset{план}
                \scnidtf{план решения конкретной задачи}
                \scnidtf{контекст конкретной задачи, предоставляющий всю информацию для решения всех подзадач для указанной конкретной задачи}
                \scnidtf{описание системы подзадач некоторой задачи}
                \scnsuperset{метод}
                \scnidtf{обобщенное описание плана решения любой задачи из некоторого заданного класса задач}
                \scnsuperset{навык}
                \scnidtf{метод, детализированный до уровня элементарных подзадач}
            }
        \end{scnsubdividing}
        \scnheader{отношение, заданное на множестве знаний}
        \scnhaselement{дочернее знание*}
        \scnidtf{знание, которое от материнского знания наследует все описанные там свойства объектов исследования}
        \scntext{примечание}{Факт наследования свойств описываемых объектов от материнского знания подчеркивается использованием прилагательного дочернее в sc-идентификаторе данного отношения, заданного на множестве знаний}
        \scnsuperset{дочерний раздел*}
        \scnidtf{частный раздел*}
        \scnsuperset{дочерняя предметная область и онтология*}
        \scnhaselement{спецификация*}
        \scnidtf{быть знанием, которое является спецификацией (описанием) заданной сущности}
        \scntext{примечание}{специфицируемой сущностью может быть сущность любого вида, в том числе, и другое знание}
        \scnhaselement{онтология*}
        \scnidtf{быть семантической спецификацией заданного знания*}
        \scnhaselement{семантическая эквивалентность*}
        \scnhaselement{следовательно*}
        \scnidtf{логическое следствие*}
        \scnhaselement{логическая эквивалентность*}
        \bigskip
    \end{scnsubstruct}
    \scnendcurrentsectioncomment
\end{SCn}
