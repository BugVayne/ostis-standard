\begin{SCn}
    \scnsectionheader{Предметная область и онтология семантических окрестностей}
    \begin{scnsubstruct}
        \scnheader{Предметная область семантических окрестностей}
        \scniselement{предметная область}
        \begin{scnhaselementrole}{максимальный класс объектов исследования}
            {семантическая окрестность}
        \end{scnhaselementrole}
        \begin{scnhaselementrolelist}{класс объектов исследования}
            \scnitem{семантическая окрестность по инцидентным коннекторам}
            \scnitem{семантическая окрестность по выходящим дугам}
            \scnitem{семантическая окрестность по выходящим дугам принадлежности}
            \scnitem{семантическая окрестность по входящим дугам}
            \scnitem{семантическая окрестность по входящим дугам принадлежности}
            \scnitem{полная семантическая окрестность}
            \scnitem{базовая семантическая окрестность}
            \scnitem{специализированная семантическая окрестность}
            \scnitem{пояснение}
            \scnitem{примечание}
            \scnitem{правило идентификации экземпляров}
            \scnitem{терминологическая семантическая окрестность}
            \scnitem{теоретико-множественная семантическая окрестность}
            \scnitem{описание декомпозиции}
            \scnitem{логическая семантическая окрестность}
            \scnitem{спецификация типичного экземпляра}
            \scnitem{сравнительный анализ}
        \end{scnhaselementrolelist}
        
        \scnheader{семантическая окрестность}
        \scnidtf{sc-окрестность}
        \scnidtf{семантическая окрестность, представленная в виде sc-текста}
        \scnidtf{sc-текст, являющийся семантической окрестностью некоторого sc-элемента}
        \scnidtf{спецификация заданной сущности, знак которой указывается как ключевой элемент этой спецификации}
        \scnidtf{описание заданной сущности, знак которой указывается как ключевой элемент этой спецификации}
        \scnsubset{знание}
        \scnsuperset{семантическая окрестность по инцидентным коннекторам}
        \scnsuperset{полная семантическая окрестность}
        \scnsuperset{базовая семантическая окрестность}
        \scnsuperset{специализированная семантическая окрестность}
        \scnidtftext{пояснение}{\textit{знание}, являющееся спецификацией (описанием) некоторой \textit{сущности}, знак которой является \textit{ключевым знаком\scnrolesign} указанного \textit{знания}. Заметим, что каждая \textit{семантическая окрестность} в отличие от \textit{знаний} других видов имеет только один \textit{ключевой знак\scnrolesign} (ключевой элемент\scnrolesign, знак описываемой сущности\scnrolesign). Заметим также, что многообразие видов семантических окрестностей свидетельствует о многообразии семантических видов описаний различных сущностей.}
        \scntext{примечание}{Понятие \textit{семантической окрестности}, как и любой другой \uline{семантически} выделяемый класс \textit{знаний}, абсолютно не зависит от \textit{языка представления знаний}. Этим \textit{языком} может быть не только \textit{SC-код} или другой \textit{формальный язык представления знаний} или даже \textit{естественный язык}, тексты которых в \textit{памяти ostis-системы} представляются в виде \textit{файлов}.}
        
        \scnheader{семантическая окрестность по инцидентным коннекторам}
        \scnsuperset{семантическая окрестность по выходящим дугам}
        \scnsuperset{семантическая окрестность по входящим дугам}
        \scnidtftext{пояснение}{вид \textit{семантической окрестности}, в которую входят все коннекторы, инцидентные заданному элементу, а также все элементы, инцидентные указанным коннекторам.}
        
        \scnheader{семантическая окрестность по выходящим дугам}
        \scnsuperset{семантическая окрестность по выходящим дугам принадлежности}
        \scnidtftext{пояснение}{вид \textit{семантической окрестности}, в которую входят все дуги, выходящие из заданного sc-элемента и вторые компоненты этих дуг. Также указывается факт принадлежности этих дуг каким-либо отношениям.}
        
        \scnheader{семантическая окрестность по выходящим дугам принадлежности}
        \scnidtftext{пояснение}{вид \textit{семантической окрестности}, в которую входят все дуги принадлежности, выходящие из заданного \textit{sc-элемента}, а также их вторые компоненты. При необходимости может указывается факт \textit{принадлежности} этих дуг каким-либо \textit{ролевым отношениям}.}
        
        \scnheader{семантическая окрестность по входящим дугам}
        \scnsuperset{семантическая окрестность по входящим дугам принадлежности}
        \scnidtftext{пояснение}{вид \textit{семантической окрестности}, в которую входят все дуги, входящие в заданный sc-элемент, а также их первые компоненты. Также указывается факт принадлежности этих дуг каким-либо отношениям.}
        
        \scnheader{семантическая окрестность по входящим дугам принадлежности}
        \scnidtftext{пояснение}{вид \textit{семантической окрестности}, в которую входят все дуги принадлежности, входящие в заданный sc-элемент, а также их первые компоненты. При необходимости может указывается факт принадлежности этих дуг каким-либо ролевым отношениям.}
        
        \scnheader{полная семантическая окрестность}
        \scnidtf{полная спецификация некоторой описываемой сущности}
        \scnidtftext{пояснение}{вид \textit{семантической окрестности}, включающий описание всех связей описываемой сущности. Структура полной семантической окрестности определяется прежде всего семантической типологией описываемой сущности. Так, например, для \textit{понятия} в \textit{полную семантическую окрестность} необходимо включить следующую информацию (при наличии):
            \begin{scnitemize}
                \item варианты идентификации на различных внешних языках (sc-идентификаторы);
                \item принадлежность некоторой \textit{предметной области} с указанием роли, выполняемой в рамках этой предметной области;
                \item теоретико-множественные связи заданного \textit{понятия} с другими \textit{sc-элементами};
                \item определение или пояснение;
                \item высказывания, описывающие свойства указанного \textit{понятия};
                \item задачи и их классы, в которых данное \textit{понятие} является ключевым;
                \item описание типичного примера использования указанного \textit{понятия};
                \item экземпляры описываемого \textit{понятия}.
            \end{scnitemize}
            Для понятия, являющегося отношением дополнительно указываются:
            \begin{scnitemize}
                \item домены;
                \item область определения;
                \item схема отношения;
                \item классы отношений, которым принадлежит описываемое отношение.
            \end{scnitemize}}
        
        \scnheader{базовая семантическая окрестность}
        \scnidtf{минимально достаточная семантическая окрестность}
        \scnidtf{минимальная спецификация описываемой сущности}
        \scnidtf{сокращенная спецификация описываемой сущности}
        \scnidtf{основная семантическая окрестность}
        \scnidtftext{пояснение}{вид \textit{семантической окрестности}, содержащий минимальную (краткую) информацию об описываемой сущностиСтруктура базовой семантической окрестности определяется прежде всего семантической типологией описываемой сущности. Так, например, для \textit{понятия} в базовую семантическую окрестность необходимо включить следующую информацию (при наличии):
            \begin{scnitemize}
                \item варианты идентификации на различных внешних языках (sc-идентификаторы);
                \item принадлежность некоторой \textit{предметной области} с указанием роли, выполняемой в рамках этой предметной области;
                \item \textit{определение} или пояснение.
            \end{scnitemize}
            Для \textit{понятия}, являющегося \textit{отношением} дополнительно указываются:
            \begin{scnitemize}
                \item \textit{домены};
                \item \textit{область определения};
                \item описание типичного примера связки указанного отношения (спецификация типичного экземпляра).
            \end{scnitemize}}
        
        \scnheader{специализированная семантическая окрестность}
        \scnsuperset{пояснение}
        \scnsuperset{примечание}
        \scnsuperset{правило идентификации экземпляров}
        \scnsuperset{терминологическая семантическая окрестность}
        \scnsuperset{теоретико-множественная семантическая окрестность}
        \scnsuperset{логическая семантическая окрестность}
        \scnsuperset{описание типичного экземпляра}
        \scnsuperset{описание декомпозиции}
        \scnidtftext{пояснение}{вид \textit{семантической окрестности}, набор связей для которой уточняется отдельно для каждого типа такой окрестности.}
        
        \scnheader{пояснение}
        \scnidtf{sc-пояснение}
        \scnidtftext{пояснение}{знак sc-текста, поясняющего описываемую сущность.}
        
        \scnheader{примечание}
        \scnidtf{sc-примечание}
        \scnidtftext{пояснение}{знак sc-текста, являющегося примечанием к описываемой сущности. В примечании обычно описываются особые свойства и исключения из правил для описываемой сущности.}
        
        \scnheader{правило идентификации экземпляров}
        \scnidtf{правило идентификации экземпляров заданного класса}
        \scnidtftext{пояснение}{sc-текст являющийся описанием правил построения идентификаторов элементов заданного класса.}
        
        \scnheader{терминологическая семантическая окрестность}
        \scnidtftext{пояснение}{\textit{семантическая окрестность}, описывающая внешнюю идентификацию указанной сущности, т.е. её sc-идентификаторы}
        
        \scnheader{теоретико-множественная семантическая окрестность}
        \scnidtftext{пояснение}{описание связи описываемого множества с другими множествами с помощью теоретико-множественных отношений}
        
        \scnheader{описание декомпозиции}
        \scnidtf{\textit{семантическая окрестность}, описывающая декомпозицию некоторой сущности}
        \scnidtftext{пояснение}{\textit{семантическая окрестность}, описывающая декомпозицию некоторой сущности на её части}
        
        \scnheader{логическая семантическая окрестность}
        \scnidtftext{пояснение}{\textit{семантическая окрестность}, описывающая семейство высказываний, описывающих свойства данного \textit{понятия} или какого-либо конкретного экземпляра некоторого понятия}
        
        \scnheader{спецификация типичного экземпляра}
        \scnidtf{описание типичного экземпляра заданного класса}
        \scnidtftext{пояснение}{sc-текст являющийся описанием типичного примера рассматриваемого класса.}
        
        \scnheader{сравнительный анализ}
        \scnidtftext{пояснение}{описание сравнения некоторой сущности с другими аналогичными сущностями}
        
        \scnheader{сравнение}
        \scnidtftext{пояснение}{описание сравнения (сходств и отличий) двух сущностей, которые заданы \textit{парой} (двухмощными множествами), которому принадлежат знаки обеих сравниваемых сущностей}
        
        \scnheader{семантическая окрестность}
        \scntext{примечание}{Всему классу \textit{семантических окрестностей} и всем подклассам этого \textit{класса}, а так же всем другим классам \textit{знаний}, ставятся в соответствие \textit{бинарные ориентированые отношения}, вторыми \textit{доменами} которых являются указанные \textit{классы} и объединение которых является \textit{обратным отношением} для \textit{отношения} быть ключевым знаком\scnrolesign. Эти отношения не следует причислять к основным отношениям, т.к. они вместе с выделенными классами семантических окрестностей привносят дополнительную логическую эквивалентность в базу знаний.}
        
        \scnheader{семантическая окрестность}
        \begin{scnrelfromlist}{отношение, заданное на}
            \scnitem{семантическая окрестность*}
            \scnitem{семантическая окрестность по инцидентным коннекторам*}
            \scnitem{полная семантическая окрестность*}
            \scnitem{базовая семантическая окрестность*}
            \scnitem{специализированная синтетическая окрестность*}
            \scnitem{и т.д.}
        \end{scnrelfromlist}
        \scntext{примечание}{Понятие семантической окрестности, дополненное уточнением таких понятий, как семантическое расстояние между знаками (семантическая близость знаков), радиус семантической окрестности, является перспективной основой для исследования свойств смыслового пространства.}
        
        \bigskip
    \end{scnsubstruct}
    \scnendcurrentsectioncomment
\end{SCn}
