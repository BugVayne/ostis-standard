\begin{SCn}
    \scnsectionheader{\currentname}
    \begin{scnsubstruct}
        \begin{scnrelfromlist}{соавтор}
            \scnitem{Гордей А.Н.}
            \scnitem{Никифоров С.А.}
            \scnitem{Бобёр Е.С.}
            \scnitem{Святощик М.И.}
        \end{scnrelfromlist}
        \scnheader{Предметная область естественных языков}
        \scniselement{предметная область}
        \begin{scnhaselementrolelist}{максимальный класс объектов исследования}
            \scnitem{язык}
        \end{scnhaselementrolelist}
        \begin{scnhaselementrolelist}{класс объектов исследования}
            \scnitem{плановый язык}
            \scnitem{язык общения}
            \scnitem{лексема}
            \scnitem{номинативная единица}
            \scnitem{комбинаторный вариант лексемы}
            \scnitem{естественный язык}
            \scnitem{тайген}
            \scnitem{ёген}
        \end{scnhaselementrolelist}
        \begin{scnhaselementrolelist}{исследуемое отношение}
            \scnitem{морфологическая парадигма*}
            \scnitem{член предложения\scnrolesign}
        \end{scnhaselementrolelist}
        \scnheader{язык}
        \begin{scnsubdividing}
            \scnitem{естественный язык
                ~\\\scntext{explanation}{Естественный язык представляет собой язык, который не был создан целенаправленно}}
            \scnitem{искусственный язык
                ~\\\scntext{explanation}{Искусственный язык представляет собой язык, специально разработанный для достижения определённых целей}
                \scnhaselement{Эсперанто}
                \scnhaselement{Python}
                \scnsuperset{сконструированный язык}
                \scntext{explanation}{Сконструированный язык представляет собой искусственный язык, предназначенный для общения людей}
                \scnhaselement{Эсперанто}
            }
        \end{scnsubdividing}
        \scnsuperset{международный язык}
        \scntext{explanation}{Международный язык представляет собой естественный или искусственный язык, использующийся для общения людей разных из стран}
        \scnhaselement{Английский язык}
        \scnhaselement{Русский язык}
        \scnheader{плановый язык}
        \begin{scnreltoset}{пересечение}
            \scnitem{сконструированный язык}
            \scnitem{международный язык}
        \end{scnreltoset}
        \scnheader{язык общения}
        \begin{scnreltoset}{объединение}
            \scnitem{естественный язык}
            \scnitem{сконструированный язык}
        \end{scnreltoset}
        \scnhaselement{Английский язык}
        \scnhaselement{Русский язык}
        \scnhaselement{Эсперанто}
        \begin{scnreltoset}{объединение}
            \scnitem{корневой язык
                ~\\\scntext{explanation}{Корневой язык представляет собой язык, для которого характерно полное отсутствие словоизменения и наличие грамматической значимости порядка слов, состоящих только из корня.}
                \scnhaselement{Английский язык}
            }
            \scnitem{агглютинативный язык
                ~\\\scntext{explanation}{Агглютинативный язык характеризуется развитой системой употребления суффиксов, приставок, добавляемых к неизменяемой основе слова, которые используются для выражения категорий числа, падежа, рода и др.}
                \scnhaselement{Английский язык}
            }
            \scnitem{флективный язык
                ~\\\scntext{explanation}{Для флективного языка характерно развитое употребление окончаний для выражения категорий рода, числа, падежа, сложная система склонения глаголов, чередование гласных в корне, а также строгое различение частей речи.}
                \scnhaselement{Русский язык}
            }
            \scnitem{профлективный язык
                ~\\\scntext{explanation}{Для профлективного языка характерны агглютинация (в случае именного словоизменения), флексия и чередование гласных (аблаут)(в случае глагольного словоизменения).}}
        \end{scnreltoset}
        \scnheader{лексема}
        \scnsubset{файл}
        \scntext{explanation}{\textit{Лексема} -- тайген или ёген конкретного естественного языка.}
        \scnrelfrom{источник}{\cite{Hardzei2005}}
        \scnheader{номинативная единица}
        \scnsubset{файл}
        \scntext{explanation}{\textit{Номинативная единица} -- устойчивая последовательность комбинторных вариантов лексем, в которой один вариант лексемы (модификатор) определяет другой (актуализатор), например: записная книжка, бежать галопом.}
        \scnrelfrom{источник}{\cite{Hardzei2005}}
        \scnheader{комбинаторный вариант лексемы}
        \scnsubset{файл}
        \scntext{explanation}{\textit{Комбинторный вариант лексемы } -- вариант лексемы в упорядоченном наборе её вариантов (парадигме).}
        \scnrelfrom{источник}{\cite{Hardzei2007}}
        \scnheader{морфологическая парадигма*}
        \scniselement{квазибинарное отношение}
        \scntext{explanation}{\textit{Морфологическая парадигма*} -- квазибинарное отношение, связывающее лексему с её комбинторными вариантами.}
        \scnrelfrom{первый домен}{словоформа}
        \scnrelfrom{второй домен}{лексема}
        \scnheader{естественный язык}
        \begin{scnsubdividing}
            \scnitem{часть языка
                ~\\\begin{scnsubdividing}
                    \scnitem{тайген}
                    \scnitem{ёген}
                \end{scnsubdividing}
            }
            \scnitem{знак алфавита синтаксиса
                ~\\\scntext{explanation}{\textit{Знаки алфавита синтаксиса} -- вспомогательные средства синтаксиса (на макроуровне -- предлоги, послелоги, союзы, частицы и др., на микроуровне -- флексии, префиксы, постфиксы, инфиксы и др.), служащие для соединения составных частей языковых структур и образования морфологических парадигм.}
                \scnrelfrom{источник}{\cite{Hardzei2005}}
            }
        \end{scnsubdividing}
        \scnheader{тайген}
        \scntext{explanation}{\textit{Тайген} -- часть языка, обозначающая индивида.}
        \scnrelfrom{источник}{\cite{Hardzei2006}}
        \scnrelfrom{источник}{\cite{Hardzei2015}}
        \begin{scnsubdividing}
            \scnitem{развёрнутый тайген
                ~\\\begin{scnsubdividing}
                    \scnitem{составной тайген}
                    \scnitem{сложный тайген}
                \end{scnsubdividing}
            }
            \scnitem{свёрнутый тайген
                ~\\\begin{scnsubdividing}
                    \scnitem{сокращённый тайген}
                    \scnitem{сжатый тайген
                        ~\\\begin{scnsubdividing}
                            \scnitem{информационный тайген
                                ~\\\scntext{explanation}{\textit{Информационный тайген} -- тайген, обозначающий индивида в информационном фрагменте модели мира.}
                                \scnrelfrom{источник}{\cite{Hardzei2006}}
                                \scnrelfrom{источник}{\cite{Hardzei2015}}
                            }
                            \scnitem{физический тайген
                                ~\\\scntext{explanation}{\textit{Физический тайген} -- тайген, обозначающий индивида в физическом фрагменте модели мира. }
                                \scnrelfrom{источник}{\cite{Hardzei2006}}
                                \scnrelfrom{источник}{\cite{Hardzei2015}}
                                \begin{scnsubdividing}
                                    \scnitem{постоянный тайген
                                        ~\\\scntext{explanation}{\textit{Постоянный тайген} -- физический тайген, обозначающий постоянного индивида.}
                                        \scnrelfrom{источник}{\cite{Hardzei2006}}
                                        \scnrelfrom{источник}{\cite{Hardzei2015}}
                                    }
                                    \scnitem{переменный тайген
                                        ~\\\scntext{explanation}{\textit{Переменный тайген} -- физический тайген, обозначающий переменного индивида.}
                                        \scnrelfrom{источник}{\cite{Hardzei2006}}
                                        \scnrelfrom{источник}{\cite{Hardzei2015}}
                                    }
                                \end{scnsubdividing}
                                \begin{scnsubdividing}
                                    \scnitem{качественный тайген}
                                    \scnitem{количественный тайген}
                                \end{scnsubdividing}
                                \begin{scnsubdividing}
                                    \scnitem{одноместный тайген}
                                    \scnitem{многоместный тайген
                                        ~\\\scnsuperset{интенсивный тайген}
                                        \scnsuperset{экстенсивный тайген}
                                    }
                                \end{scnsubdividing}
                            }
                        \end{scnsubdividing}
                    }
                \end{scnsubdividing}
            }
        \end{scnsubdividing}
        \scnheader{ёген}
        \scntext{explanation}{\textit{Ёген} -- часть языка, обозначающая признак индивида.}
        \scnrelfrom{источник}{\cite{Hardzei2006}}
        \scnrelfrom{источник}{\cite{Hardzei2015}}
        \begin{scnsubdividing}
            \scnitem{развёрнутый ёген
                ~\\\begin{scnsubdividing}
                    \scnitem{составной ёген}
                    \scnitem{сложный ёген}
                \end{scnsubdividing}
            }
            \scnitem{свёрнутый ёген
                ~\\\begin{scnsubdividing}
                    \scnitem{сокращённый ёген
                        ~\\\begin{scnsubdividing}
                            \scnitem{информационный ёген
                                ~\\\scntext{explanation}{\textit{Информационный ёген} -- еген, обозначающий признак индивида в информационном фрагменте модели мира.}
                                \scnrelfrom{источник}{\cite{Hardzei2006}}
                                \scnrelfrom{источник}{\cite{Hardzei2007a}}
                            }
                            \scnitem{физический ёген
                                ~\\\scntext{explanation}{\textit{Физический ёген} -- еген, обозначающий признак индивида в физическом фрагменте модели мира.}
                                \scnrelfrom{источник}{\cite{Hardzei2006}}
                                \scnrelfrom{источник}{\cite{Hardzei2007a}}
                                \begin{scnsubdividing}
                                    \scnitem{постоянный ёген
                                        ~\\\scntext{explanation}{\textit{Постоянный ёген} - физический ёген, обозначающий постоянный признак индивида.}
                                        \scnrelfrom{источник}{\cite{Hardzei2006}}
                                        \scnrelfrom{источник}{\cite{Hardzei2007a}}
                                    }
                                    \scnitem{переменный ёген
                                        ~\\\scntext{explanation}{\textit{Переменный ёген} -- физический ёген, обозначающий переменный признак индивида.}
                                        \scnrelfrom{источник}{\cite{Hardzei2006}}
                                        \scnrelfrom{источник}{\cite{Hardzei2007a}}
                                    }
                                \end{scnsubdividing}
                                \begin{scnsubdividing}
                                    \scnitem{качественный ёген}
                                    \scnitem{количественный ёген}
                                \end{scnsubdividing}
                                \begin{scnsubdividing}
                                    \scnitem{одноместный ёген}
                                    \scnitem{многоместный ёген
                                        ~\\\begin{scnsubdividing}
                                            \scnitem{интенсивный ёген}
                                            \scnitem{экстенсивный ёген}
                                        \end{scnsubdividing}
                                    }
                                \end{scnsubdividing}
                            }
                        \end{scnsubdividing}
                    }
                    \scnitem{сжатый ёген}
                \end{scnsubdividing}
            }
        \end{scnsubdividing}
        \scnheader{член предложения\scnrolesign}
        \scniselement{ролевое отношение}
        \scntext{explanation}{\textit{Член предложения\scnrolesign} -- это отношение, связывающее декомпозицию текста с файлом, содержимое которого (часть языка) играет в декомпозируемом тексте определенную синтаксическую роль.}
        \scnrelfrom{источник}{\cite{Hardzei2005}}
        \begin{scnsubdividing}
            \scnitem{главный член предложения\scnrolesign
                ~\\\begin{scnsubdividing}
                    \scnitem{подлежащее\scnrolesign
                        ~\\\scntext{explanation}{\textit{Подлежащее\scnrolesign} -- это одно из главных ролевых отношений, связывающее декомпозицию текста с файлом, содержимое которого обозначает исходный пункт описания события, выбранный наблюдателем. }
                        \scnrelfrom{источник}{\cite{Hardzei2020}}
                    }
                    \scnitem{сказуемое\scnrolesign
                        ~\\\scntext{explanation}{\textit{Сказуемое\scnrolesign} -- это одно из главных ролевых отношений, связывающее декомпозицию текста с файлом, содержимое которого обозначает отображение наблюдателем исходного пункта описания события в конечный.}
                        \scnrelfrom{источник}{\cite{Hardzei2020}}
                    }
                    \scnitem{прямое дополнение\scnrolesign
                        ~\\\scntext{explanation}{\textit{Прямое дополнение\scnrolesign} -- это одно из главных ролевых отношений, связывающее декомпозицию текста с файлом, содержимое которого обозначает конечный пункт описания события, выбранный наблюдателем.}
                        \scnrelfrom{источник}{\cite{Hardzei2020}}
                    }
                \end{scnsubdividing}
            }
            \scnitem{второстепенный член предложения\scnrolesign
                ~\\\begin{scnsubdividing}
                    \scnitem{косвенное дополнение\scnrolesign}
                    \scnitem{определение\scnrolesign
                        ~\\\scntext{explanation}{\textit{Определение\scnrolesign} -- это одно из второстепенных ролевых отношений, связывающее декомпозицию текста с файлом, содержимое которого обозначает модификацию подлежащего, дополнения, обстоятельства места и времени.}
                        \scnrelfrom{источник}{\cite{Hardzei2007a}}
                        \scnrelfrom{источник}{\cite{Hardzei2017a}}
                        \scnrelfrom{источник}{\cite{Hardzei2007b}}
                    }
                    \scnitem{обстоятельство\scnrolesign
                        ~\\\scntext{explanation}{\textit{Обстоятельство\scnrolesign} -- это одно из второстепенных ролевых отношений, связывающее декомпозицию текста с файлом, содержимое которого обозначает либо модификацию, либо локализацию сказуемого.}
                        \scnrelfrom{источник}{\cite{Hardzei2007a}}
                        \scnrelfrom{источник}{\cite{Hardzei2017a}}
                        \scnrelfrom{источник}{\cite{Hardzei2007b}}
                        \begin{scnsubdividing}
                            \scnitem{обстоятельство степени\scnrolesign
                                ~\\\scntext{explanation}{Обстоятельство степени -- обстоятельство, обозначающее модификацию сказуемого.}}
                            \scnitem{обстоятельство образа действия\scnrolesign
                                ~\\\scntext{explanation}{Обстоятельство образа действия -- обстоятельство, обозначающее модификацию сказуемого.}}
                            \scnitem{обстоятельство места\scnrolesign
                                ~\\\scntext{explanation}{Обстоятельства места -- обстоятельство, обозначающее пространственную локализацию сказуемого.}
                                \begin{scnsubdividing}
                                    \scnitem{динамическое обстоятельство места\scnrolesign}
                                    \scnitem{статическое обстоятельство места\scnrolesign}
                                \end{scnsubdividing}
                            }
                            \scnitem{обстоятельство времени\scnrolesign
                                ~\\\scntext{explanation}{Обстоятельство времени -- обстоятельство, обозначающее временную локализацию сказуемого.}
                                \begin{scnsubdividing}
                                    \scnitem{динамическое обстоятельство времени\scnrolesign}
                                    \scnitem{статическое обстоятельство времени\scnrolesign}
                                \end{scnsubdividing}
                            }
                        \end{scnsubdividing}
                    }
                \end{scnsubdividing}
            }
        \end{scnsubdividing}
        \bigskip
        \bigskip
        \scnheader{Пример sc.g-текста, описывающего лексему}
        \scniselement{sc.g-текст}
        \scntext{explanation}{Здесь представлено описание лексемы с указанием ее принадлежности определённой части речи. Также описание содержит морфологическую парадигму данной лексемы, связывающую ее с ее словоформами.}
        \scneq{\scnfileimage[20em]{figures/sd_natural_languages/lexeme_example.png}}
        \scnheader{Пример этапов разбора текста естественного языка}
        \begin{scnsubstruct}
            \scnfileimage[20em]{figures/sd_natural_languages/nl_text.png}
            \scntext{explanation}{с точки зрения ostis-системы, любой естественно-языкой текст является \textit{файлом.}}
            \scnrelfrom{лексическая структура}{\scnfileimage[20em]{figures/sd_natural_languages/nl_lexical.png}
            }
            \scntext{explanation}{Данная конструкция описывает декомпозицию исходного текста на фрагменты с указанием их принадлежности определённой \textit{номинативной единице} или \textit{знаку алфавита синтаксиса}.}
            \scnrelfrom{синтаксическая структура}{\scnfileimage[20em]{figures/sd_natural_languages/nl_synactical.png}
            }
            \bigskip
            \bigskip
            \scntext{explanation}{Здесь приведена только частью синтаксической структуры. Оставшаяся часть записывается аналогично.}
        \end{scnsubstruct}
    \end{scnsubstruct}
    \scnendcurrentsectioncomment
\end{SCn}
