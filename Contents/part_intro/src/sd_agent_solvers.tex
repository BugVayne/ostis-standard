\begin{SCn}
	\scnsectionheader{Предметная область и онтология многоагентных моделей решателей задач, основанных на смысловом представлении информации}

	\begin{scnsubstruct}

		\scnheader{Предметная область многоагентных онтологических моделей решателей задач, основанных на смысловом представлении информации}
		\scniselement{предметная область}
		\begin{scnhaselementrolelist}{класс объектов исследования}
			\scnitem{многоагентный подход к обработке информации}
		\end{scnhaselementrolelist}

		\begin{scnhaselementrolelist}{класс объектов исследования}
			\scnitem{интеграция решателей задач}
		\end{scnhaselementrolelist}

		\begin{scnhaselementrolelist}{исследуемое отношение}
			\scnitem{совместимость моделей решения задач*}
		\end{scnhaselementrolelist}

		\scnheader{агентно-ориентированный подход к обработке информации}
		\scntext{примечание}{В качестве основы унификации принципов обработки
			информации в компьютерных системах предлагается использовать
			\textit{агентно-ориентированный подход к обработке информации}, обладающий
			рядом важных достоинств.}
		\begin{scnrelfromset}{достоинства}
			\scnfileitem{Автономность (независимость) агентов, что позволяет
				локализовать изменения, вносимые в систему при ее эволюции, и снизить
				соответствующие трудозатраты.}
			\scnfileitem{Децентрализация обработки, т.е. отсутствие единого
				контролирующего центра, что также позволяет локализовать вносимые в систему
				изменения.}
			\scnfileitem{Возможность параллельной работы разных информационных
				процессов, соответствующих как одному агенту, так и разным агентам, как
				следствие, --- возможность распределенного решения задач. Однако возможность
				параллельного выполнения информационных процессов подразумевает наличие средств
				синхронизации такого выполнения, разработка которых является отдельной
				задачей.}
			\scnfileitem{Активность агентов и многоагентной системы в целом, дающая
				возможность при общении с такой системой не указывать явно способ решения
				поставленной задачи, а формулировать задачу в \uline{декларативном ключе}.}
		\end{scnrelfromset}

		\begin{scnindent}
			\scnrelfrom{источник}{\cite{Wooldridge2009}}
		\end{scnindent}

		\begin{scnrelfromset}{недостатки современного состояния}
			\scnfileitem{Знания агента представляются при помощи
				узкоспециализированных языков, зачастую не предназначенных для представления
				знаний в широком смысле и онтологий в частности.}
			\scnfileitem{Большинство современных многоагентных систем предполагает,
				что взаимодействие агентов осуществляется путем обмена сообщениями
				непосредственно от агента к агенту.}
			\scnfileitem{Логический уровень взаимодействия агентов жестко привязан
				к физическому уровню реализации многоагентной системы.}
			\scnfileitem{Среда, с которой взаимодействуют агенты, уточняется
				отдельно разработчиком для каждой многоагентной системы, что приводит к
				существенным накладным расходам и несовместимости таких многоагентных систем.}
		\end{scnrelfromset}
		\begin{scnindent}
			\begin{scnrelfromset}{принципы устранения}
				\scnfileitem{Коммуникацию агентов предлагается осуществлять путем
					спецификации (в общей памяти компьютерной системы) действий (процессов),
					выполняемых агентами и направленных на решение задач.}
					\begin{scnindent}
						\scntext{детализация}{Коммуникацию агентов предлагается
							осуществлять по принципу \scnqqi{доски объявлений}, однако в отличие от классического
							подхода в роли сообщений выступают спецификации в общей семантической памяти
							выполняемых агентами действий (процессов), направленных на решение каких-либо
							задач, а в роли среды коммуникации выступает сама эта семантическая память.
							Такой подход позволяет:
							\begin{scnitemize}
								\item исключить необходимость разработки специализированного
											языка для обмена сообщениями
								\item обеспечить \scnqq{обезличенность} общения, т. е. каждый из
								агентов в общем случае не знает, какие еще агенты есть в системе, кем
								сформулирован и кому адресован тот или иной запрос. Таким образом, добавление
								или удаление агентов в систему не приводит к изменениям в других агентах, что
								обеспечивает модифицируемость всей системы
								\item агентам, в том числе конечному пользователю,
											формулировать задачи в \uline{декларативном ключе}, т. е. не указывать для
											каждой задачи способ ее решения. Таким образом, агенту заранее не нужно знать,
											каким образом система решит ту или иную задачу, достаточно лишь специфицировать
											конечный результат
								\item сделать средства коммуникации агентов и синхронизации их
											деятельности более понятными разработчику и пользователю системы, не требующими
											изучения специальных низкоуровневых типов данных и форматов сообщений. Таким
											образом повышается доступность предлагаемых решений широкому кругу
											разработчиков.
							\end{scnitemize}
							Следует отметить, что такой подход позволяет при необходимости
							организовать обмен сообщениями между агентами напрямую и, таким образом, может
							являться основой для моделирования многоагентных систем, предполагающих другие
							способы взаимодействия между агентами.}
						\end{scnindent}	
				\scnfileitem{В роли внешней среды для агентов выступает та же общая
					память, в которой формулируются задачи и посредством которой осуществляется
					взаимодействие агентов. Такой подход обеспечивает унификацию среды для
					различных систем агентов, что, в свою очередь, обеспечивает их совместимость.}
				\scnfileitem{Спецификация каждого агента описывается средствами языка
					представления знаний в той же памяти, что позволяет:
					\begin{scnitemize}
						\item минимизировать число специализированных средств, необходимых для
						спецификации агентов, как языковых, так и инструментальных
						\item с одной стороны --- минимизировать необходимую в общем случае
						спецификацию агента, которая включает условие его инициирования и программу,
						описывающую алгоритм работы агента, с другой стороны --- обеспечить возможность
						произвольного расширения спецификации для каждого конкретного случая, в том
						числе возможность реализации различных современных моделей спецификации агента.
					\end{scnitemize}}
				\scnfileitem{Синхронизацию деятельности агентов предполагается
					осуществлять на уровне выполняемых ими процессов, направленных на решение тех
					или иных задач в общей семантической памяти. Таким образом, каждый агент
					трактуется как некий абстрактный процессор, способный решать задачи
					определенного класса. При таком подходе необходимо решить задачу обеспечения
					взаимодействия параллельных асинхронных процессов в общей семантической памяти,
					для решения которой можно заимствовать и адаптировать решения, применяемые в
					традиционной линейной памяти. При этом вводится дополнительный класс агентов --
					метаагенты, задачей которых является решение возникающих проблемных ситуаций,
					таких как взаимоблокировки}
				\scnfileitem{Каждый информационный процесс в любой момент времени имеет
					ассоциативный доступ к необходимым фрагментам базы знаний, хранящейся в
					семантической памяти, за исключением фрагментов, заблокированных другими
					процессами в соответствии с соответствующим механизмом синхронизации. Таким
					образом, с одной стороны, исключается необходимость хранения каждым агентом
					информации о внешней среде, с другой стороны, каждый агент, как и в
					классических многоагентных системах, обладает только частью всей информации,
					необходимой для решения задачи.\\Важно отметить, что в общем случае невозможно
					априори предсказать, какие именно знания, модели и способы решения задач
					понадобятся системе для решения конкретной задачи. В связи с этим необходимо
					обеспечить, с одной стороны, возможность доступа ко всем необходимым фрагментам
					базы знаний (в пределе --- ко всей базе знаний), с другой стороны --- иметь
					возможность локализовать область поиска пути решения задачи, например, рамками
					одной \textit{предметной области}.\\Каждый из агентов обладает набором ключевых
					элементов (как правило, понятий), которые он использует в качестве отправных
					точек при ассоциативном поиске в рамках базы знаний. Набор таких элементов для
					каждого агента уточняется на этапах проектирования многоагентной системы в
					соответствии с рассматриваемой ниже методикой. Уменьшение числа ключевых
					элементов агента делает его более универсальным, однако снижает эффективность
					его работы за счет необходимости выполнения дополнительных  операций.}
			\end{scnrelfromset}	
				\begin{scnindent}
					\scntext{примечание}{Предлагаемый подход позволяет рассматривать решатель
						задач как иерархическую систему. Некий целостный коллектив агентов, реализующий
						какую-либо подсистему решателя (например, машину дедуктивного вывода,
						подсистему верификации базы знаний и т. д.), может рассматриваться как единый
						неатомарный агент, поскольку коллективы агентов и отдельные агенты работают в
						соответствии с одними и теми же принципами.}
				\end{scnindent}
		\end{scnindent}

		\scnheader{совместимость моделей решения задач*}
		\scntext{примечание}{\textbf{\textit{совместимость моделей решения задач*}}
			-- это возможность одновременного использования разными моделями решения задач
			одних и тех же информационных ресурсов.}

		\begin{scnrelfromset}{принципы реализации}
			\scnfileitem{Вся информация, хранимая в памяти каждой ostis-системы и
				используемая \textit{\textbf{решателем задач}} (как собственно обрабатываемая
				информация, так и хранимые в памяти интерпретируемые методы, например,
				различного вида программы), записывается в форме смыслового представления этой
				информации}
			\scnfileitem{Собственно решение каждой задачи осуществляется
				коллективом агентов, работающих над общей для них смысловой (семантической)
				памятью и выполняющих интерпретацию хранимых в этой же памяти
				\textit{методов}.}
		\end{scnrelfromset}

		\scnheader{интеграция решателей задач}
		\scnsubset{процесс}
		\begin{scnrelfromvector}{алгоритм реализации}
			\scnfileitem{Объединение множества методов первого решателя и множества
				методов второго решателя}
			\scnfileitem{Интеграция множества методов первого решателя и множества
				методов второго решателя путем взаимного погружения соответствующих
				информационных конструкций друг в друга, т.е. путем склеивания синонимов, а
				также путем выравнивания используемых ими понятий.}
			\scnfileitem{Объединение множества агентов, входящих в состав первого
				решателя, со множеством агентов, входящих во второй решатель задач.}
		\end{scnrelfromvector}

		\scntext{пояснение}{Таким образом, унификация моделей решения задач
			путем приведения этих моделей к виду семантических моделей (т. е. моделей
			обработки информации, представленной в смысловой форме) повышает уровень
			совместимости этих моделей благодаря наличию прозрачной процедуры интеграции
			информации, представленной в смысловой форме, и тривиальной процедуры
			объединения множеств \textit{агентов}. Простота процедуры объединения множеств
			\textit{агентов}, соответствующих разным решателя задач, обусловлена тем, что
			непосредственного взаимодействия между этими агентами нет, а инициирование
			каждого из них определяется им самим, а также \uline{текущим состоянием}
			хранимой в памяти информации.}
		\bigskip
	\end{scnsubstruct}
\end{SCn}
%\scnsourcecomment{Завершили Раздел \scnqq{Предметная область и онтология многоагентных моделей решения задач, основанных на смысловом представлении информации}}
