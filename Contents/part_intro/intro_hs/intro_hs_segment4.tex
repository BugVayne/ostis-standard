\scnsegmentheader{Комплекс свойств, определяющих уровень интеллекта
    кибернетической системы}
\begin{scnsubstruct}
    \scnheader{интеллект}
    \scniselement{свойство}
    \scniselement{упорядоченное свойство}
    \scnidtf{уровень (степень, величина) интеллекта кибернетической системы}
    \scnidtf{Семейство классов \textit{кибернетических систем}, обладающих
        эквивалентным (одинаковым) уровнем интеллекта -- от низкого до высокого уровня
        интеллекта}
    \scnidtf{свойство кибернетических систем, характеризующее эффективность их
        взаимодействия со своей средой (средой их жизнедеятельности)}
    \scnrelfrom{область определения}{кибернетическая система}
    \newpage\scntext{explanation}{С формальной точки зрения интеллектуальность --
        это семейство классов кибернетических систем, в каждый из которых входят
        кибернетические системы, эквивалентные по уровню и характеру проявления
        интеллектуальных свойств (в том числе способностей).
        ~\\Таким образом, характер (вид) интеллектуальных свойств кибернетических
        систем и уровень их развития для разных кибернетических систем может быть
        разным. В соответствии с этим кибернетические системы можно сравнивать между
        собой.}\scntext{note}{Основным свойством (характеристикой, качеством,
        параметром) кибернетической системы является уровень (степень) ее интеллекта,
        который является \uline{интегральной} характеристикой, определяющей уровень
        эффективности взаимодействия кибернетической системы со средой своего
        существования.}\scnidtf{комплексное свойство (качество) кибернетической
        системы, определяющее уровень ее выживаемости во внешней среде и предполагающее
        возможность воздействия на эту среду и даже возможность ее преобразования}
    \scnidtf{интеллектуальный потенциал кибернетической системы}
    \scnidtf{спектр знаний, навыков и способностей к обучению кибернетической
        системы}
    \scnidtf{интеллектуальность кибернетической системы}
    \scntext{note}{Процесс эволюции \textit{кибернетических систем} следует
        рассматривать как процесс повышения уровня их качества по целому ряду свойств
        (характеристик) и, в первую очередь, как процесс повышения уровня их
        \textit{интеллекта}. При этом можно говорить об эволюции каждой конкретной
        \textit{кибернетической системы} в процессе своей жизнедеятельности, а также об
        эволюции целого класса \textit{кибернетических систем}, когда новые экземпляры
        этого класса являются более интеллектуальными, чем их предшественники. В таком
        аспекте, в частности, можно рассматривать эволюцию \textit{компьютерных систем}
        (искусственных кибернетических систем).}\scntext{note}{Очень важно уточнить,
        какими иными свойствами \textit{кибернетических систем} определяется уровень и
        характер их интеллектуальности. Подчеркнем, что \uline{любая}
        \textit{кибернетическая система} обладает соответствующим уровнем
        интеллектуальности. Пусть даже и достаточно низким. Существенным является
        уточнение того, за счет чего уровень интеллектуальности \textit{кибернетической
            системы} может быть повышен. Нет смысла проводить четкую границу между
        \textit{интеллектуальными кибернетическими системами} и неинтеллектуальными. Но
        есть смысл уточнять направления повышения уровня интеллектуальности
        \textit{кибернетических систем.}}\scntext{эпиграф}{Никто не может провести
        линию, отделяющую атмосферу от космоса, или черту, за которой начинается жизнь,
        или границу электронного облака. Все дело в степени проявления свойства.}
    \scnrelfrom{автор}{Барт Коско}
    \scntext{note}{Прежде, чем говорить о требованиях, предъявляемых к
        \textit{технологии проектирования и производства интеллектуальных компьютерных
            систем (искусственных кибернетических систем}, обладающих высоким уровнем
        \textit{интеллекта)}, необходимо уточнить (детализировать) \textit{свойства},
        присущие указанным системам и являющиеся предпосылками, обеспечивающими высокий
        уровень \textit{интеллекта}. Подчеркнем, что указанные \textit{свойства},
        уточняющие (детализирующие, обеспечивающие, определяющие) \textit{свойства}
        %\bigspace
        \textit{интеллектуальных систем}
        %\bigspace
        (\textit{свойства}, определяющие уровень \textit{интеллекта} этих систем)
        должны быть общими как для искусственных кибернетических систем
        (\textit{компьютерных систем}), так и для \textit{естественных кибернетических
            систем.}}\scnidtf{интегральное качество информационного обеспечения и
        информационных процессов в кибернетической системе}
    \scnidtf{интегральное качество кибернетической системы,
        определяемое:\begin{scnitemize}
            \item уровнем ее образованности -- качеством накопленных к заданному моменту
            знаний и умений (навыков);
            \item уровнем ее обучаемости -- способностью \uline{самостоятельно} повышать
            уровень своей образованности.\end{scnitemize}
    }
    \begin{scnrelfromlist}{свойство-предпосылка}
        \scnitem{образованность кибернетической системы}
        \scnitem{обучаемость кибернетической системы}
        \scnitem{социализация кибернетической системы
            ~\\\scntext{note}{Интеллект \textit{кибернетической системы}, как и лежащий в
                его основе познавательный процесс, выполняемый кибернетической системой, имеет
                социальный характер, поскольку наиболее эффективно формируется и развивается в
                форме взаимодействия \textit{кибернетической} системы с другими
                \textit{кибернетическими системами}.}}

        \scnheader{образованность кибернетической системы}
        \scnidtf{уровень навыков (умений), а также иных знаний, приобретенных
            \textit{кибернетической системой} к заданному моменту}
        \scnitem{\textbf{качество навыков, приобретенных кибернетической
                системой}\scnidtf{качество умений, которыми владеет кибернетическая система в
                текущий момент}
            \scnrelfrom{свойство-предпосылка}{\textbf{качество информации, хранимой в
                    памяти кибернетической системы}}
            \scnidtf{качество знаний, приобретенных кибернетической системой к заданному
                моменту}
        }
        \scnitem{\textbf{качество информации, хранимой в памяти кибернетической
                системы}
            ~\\\scntext{note}{Следует обратить внимание на то, что \textit{качество
                    информации, хранимой в памяти кибернетической системы}, является фактором,
                обеспечивающим не только \textit{качество навыков, приобретенных
                    кибернетической системой}, но и общий \textit{уровень качества кибернетической
                    системы}.}}
    \end{scnrelfromlist}
    \scnheader{кибернетическая система}
    \scnrelto{объединение}{Признак интеллектуальности кибернетических систем}
    \begin{scneqtoset}
        \scnitem{неинтеллектуальная кибернетическая система}
        \scnitem{интеллектуальная система
            ~\\\scnidtf{интеллектуальная кибернетическая система}
            \begin{scnreltoset}{объединение}
                \scnitem{слабоинтеллектуальная система
                    ~\\\scnidtf{кибернетическая система со слабым интеллектом}
                    \scnidtf{кибернетическая система с низким уровнем интеллекта}
                    \scnidtf{кибернетическая система с элементами интеллекта}
                }
                \scnitem{высокоинтеллектуальная система
                    ~\\\scnidtf{идеальная интеллектуальная система}
                    \scnidtf{кибернетическая система с сильным интеллектом}
                    \scnidtf{кибернетическая система с высоким уровнем интеллекта}
                    \scnidtf{действительно интеллектуальная система}
                }
            \end{scnreltoset}
        }
    \end{scneqtoset}
    \scnheader{Признак интеллектуальности кибернетических систем}
    \scntext{note}{Данный признак классификации кибернетических систем формально
        является не разбиением, а покрытием множества \textit{кибернетических систем},
        так как отсутствует четкая грань между неинтеллектуальными и интеллектуальными
        кибернетическими системами, а также между слабоинтеллектуальными и
        высокоинтеллектуальными кибернетическими системами.}\scnheader{интеллектуальная
        система}
    \scnidtf{интеллектуальная кибернетическая система}
    \scntext{note}{В этом термине слово кибернетическая можно опустить, так как
        интеллектуальными могут быть только \textit{кибернетические
            системы}}\scntext{note}{Интеллектуальные кибернетические системы могут быть
        \textit{естественными интеллектуальными системами}, искусственными
        интеллектуальными системами (которые будем называть \textit{интеллектуальными
            компьютерными системами}), а также естественно-искусственными интеллектуальными
        системами, состоящими из компонентов как естественного, так и искусственного
        происхождения. Важнейшим примером естественно-искусственных интеллектуальных
        систем являются человеко-машинные системы, представляющие собой коллективы
        (многоагентные системы), состоящие из \textit{интеллектуальных компьютерных
            систем} и людей (конечных пользователей и разработчиков этих компьютерных
        систем).}\newpage\scntext{note}{Вводя понятие \textit{интеллектуальной
            системы}, важно, во-первых, уточнить понятие \textit{кибернетической системы} и
        определить те свойства, которые присущи \uline{всем} кибернетическим системам,
        и, во-вторых, локализовать ту условную \uline{грань} перехода от
        неинтеллектуальных \textit{кибернетических систем} к интеллектуальным, а также
        \uline{грань} перехода от слабоинтеллектуальных к высокоинтеллектуальным
        кибернетическим системам. В этом и заключается уточнение феномена
        \textit{интеллекта} (интеллектуальности) кибернетических
        систем.}\scntext{note}{Все \textit{свойства} (в том числе способности и
        активности), присущие \textit{кибернетическим системам}, в различных
        \textit{кибернетических системах} могут иметь самый различный уровень (уровень
        развития). Более того, в некоторых \textit{кибернетических системах} некоторые
        из этих свойств могут вообще отсутствовать. При этом в кибернетических
        системах, которые условно будем называть \textit{\textbf{интеллектуальными
                системами}}, \uline{все} указанные выше свойства должны быть представлены в
        достаточно развитом виде. Заметим также, что мы называем
        \textit{интеллектуальными системами}, иногда называют кибернетическими
        системами с сильным интеллектом (с высоким уровнем интеллекта),
        противопоставляя их кибернетическим системам со слабым интеллектом (с низким
        уровнем интеллекта).}\scnsubset{образованная кибернетическая система}
    \scnidtf{кибернетическая система, имеющая высокий уровень образованности}
    \scnidtf{кибернетическая система, обладающая высоким уровнем знаний и навыков}
    \scnsubset{кибернетическая система, основанная на знаниях}
    \scnsubset{кибернетическая система, управляемая знаниями}
    \scnsubset{целенаправленная кибернетическая система}
    \scnsubset{гибридная кибернетическая система}
    \scnsubset{потенциально универсальная кибернетическая система}
    \scnsubset{обучаемая кибернетическая система}
    \scnidtf{когнитивная кибернетическая система}
    \scnidtf{кибернетическая система, имеющая высокий уровень обучаемости}
    \scnsubset{кибернетическая система с высоким уровнем стратифицированности своих
        знаний и навыков}
    \scnsubset{рефлексивная кибернетическая система}
    \scnsubset{самообучаемая кибернетическая система}
    \scnsubset{кибернетическая система с высоким уровнем познавательной активности}
    \scnsubset{социально ориентированная кибернетическая система}
    \scnidtf{кибернетическая система, имеющая высокий уровень социализации}
    \scnsubset{кибернетическая система, способная устанавливать и поддерживать
        высокий уровень семантической совместимости и взаимопонимания с другими
        системами}
    \scnsubset{договороспособная кибернетическая система}
    \scnidtf{кибернетическая система, способная координировать (согласовывать) свою
        деятельность с другими системами}
    \scnheader{кибернетическая система, основанная на знаниях}
    \scnidtf{кибернетическая система, в основе которой лежит формируемая в ее
        памяти, постоянно совершенствуемая и структурированная информационная модель
        той среды, в рамках которой она существует и решает соответствующие задачи}
    \scnidtf{кибернетическая система, в основе которой лежит ее база знаний --
        систематизированная совокупность всех используемых ею знаний}
    \scnidtf{кибернетическая система, формирующая в своей памяти
        систематизированную информационную модель среды своего обитания и использующая
        эту модель для организации своего целенаправленного поведения}
    \scnheader{кибернетическая система, управляемая знаниями}
    \scnidtf{кибернетическая система, в которой выполняемые ею действия
        инициируются соответствующими ситуациями и/или событиями, возникающими в ее
        базе знаний}
    \scnheader{целенаправленная кибернетическая система}
    \scnidtf{субъект, осознанно и целенаправленно осуществляющий свою деятельность,
        ведающий то, что он творит}
    \scnheader{обучаемая кибернетическая система}
    \scnidtf{когнитивная система}
    \scnidtf{кибернетическая система, способная познавать (изучать) среду своего
        обитания, то есть строить и постоянно уточнять в своей памяти информационную
        модель (описание) этой среды, а также использовать эту модель для решения
        различных задач (для организации своей деятельности (поведения)) в указанной
        среде}
    \scnidtf{кибернетическая система, способная к самосовершенствованию}
    \scnheader{социально ориентированная кибернетическая система}
    \scnidtf{кибернетическая система, имеющая достаточно высокий уровень
        интеллекта, чтобы быть полезным членом различных, в том числе, и
        человеко-машинных сообществ}
    \scntext{note}{Определенный уровень социально значимых качеств является
        необходимым условием интеллектуальности кибернетической системы. Это, своего
        рода, модификация теста Тьюринга -- важна не имитация, не иллюзия
        человекоподобия, а \uline{реальная} польза в процессе коллективного решения
        сложных задач.}\scnheader{интеллектуальная компьютерная система}
    \scnidtf{искусственная интеллектуальная система}
    \scnidtf{искусственная кибернетическая система, обладающая высоким уровнем
        интеллекта (высоким уровнем знаний и умений), а также высоким уровнем
        обучаемости}
    \scnsubset{компьютерная система}
    \scnsubset{кибернетическая система}
    \scnidtftext{основной sc-идентификатор}{интеллектуальная компьютерная система}
    \scntext{сокращение}{и.к.с.}
    \scnidtf{система искусственного интеллекта}
    \scnidtf{искусственная интеллектуальная система}
    \scnsubset{интеллектуальная система}
    \scnsubset{кибернетическая система}
    \bigskip\end{scnsubstruct}