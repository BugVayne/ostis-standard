\begin{SCn}
    \scnsectionheader{Предметная область и онтология базовых интерпретаторов логико-семантических моделей ostis-систем}
    \begin{scnsubstruct}
        \scnidtf{Предметная область и онтология платформ интерпретации sc-моделей компьютерных систем}
        \scniselement{раздел базы знаний}
        \begin{scnrelfromlist}{дочерний раздел}
            \scnitem{\nameref{sd_program_interp}}
            \scnitem{\nameref{sd_sem_comp}}
        \end{scnrelfromlist}
        \scnheader{Предметная область базовых интерпретаторов логико-семантических моделей ostis-систем}
        \scniselement{предметная область}
        \begin{scnhaselementrolelist}{класс объектов исследования}
            \scnitem{платформа интерпретации sc-моделей компьютерный систем}
        \end{scnhaselementrolelist}
        \scnheader{логико-семантическая модель кибернетической системы}
        \scnidtf{формальная модель (формальное описание) функционирования кибернетической системы, состоящая из:
            \begin{scnitemize}
                \item формальной модели информации, хранимой в памяти кибернетической системы
                \item формальной модели коллектива агентов, осуществляющих обработку указанной информации.
            \end{scnitemize}
        }
        \scnsuperset{sc-модель кибернетической системы}
        \begin{scnindent}
            \scnidtf{логико-семантическая модель кибернетической системы, представленная в SC-коде}
            \scnidtf{логико-семантическая модель ostis-системы, которая, в частности, может быть функционально эквивалентной моделью какой-либо кибернетической системы, не являющейся ostis-системой}
        \end{scnindent}
        \scnheader{кибернетическая система}
        \scnsuperset{компьютерная система}
        \begin{scnindent}
            \scnidtf{искусственная кибернетическая система}
            \scnsuperset{ostis-система}
            \begin{scnindent}
                \scnidtf{компьютерная система, построенная по технологии OSTIS на основе интерпретации спроектированной логико-семантическая sc-модель этой системы}
            \end{scnindent}
        \end{scnindent}
        \scnheader{платформа интерпретации sc-моделей компьютерных систем}
        \scnidtf{базовый интерпретатор логико-семантических моделей ostis-систем}
        \scnidtf{Семейство платформ интерпретации sc-моделей компьютерных систем}
        \scnidtf{платформа реализации sc-моделей компьютерных систем}
        \scnidtftext{часто используемый sc-идентификатор}{универсальный интерпретатор sc-моделей компьютерных систем}
        \scnidtf{универсальный интерпретатор унифицированных логико-семантических моделей компьютерных систем}
        \scnsubset{встроенная ostis-система}
        \scnidtf{встроенная пустая ostis-система}
        \scnidtf{универсальный интерпретатор sc-моделей ostis-систем}
        \scnidtf{универсальная базовая ostis-система, обеспечивающая имитацию любой ostis-системы путем интерпретации sc-модели имитируемой ostis-системы}
        \scntext{примечание}{соотношение между имитируемой и универсальной ostis-системой в известной мере аналогично соотношению между машиной Тьюринга и универсальной машиной Тьюринга}
        \scntext{пояснение}{Под \textbf{\textit{платформой интерпретации sc-моделей компьютерных систем}} понимается реализация платформы интерпретации sc-моделей, которая в общем случае включает в себя:Реализация \textit{платформы интерпретации sc-моделей компьютерных систем} (\textit{универсального интерпретатора sc-моделей компьютерных систем}) может иметь большое число вариантов --- как программно, так и аппаратно реализованных. Логическая архитектура \textit{платформы интерпретации sc-моделей компьютерных систем} обеспечивает независимость проектируемых компьютерных систем от многообразия вариантов реализации интерпретатора их моделей и в общем случае включает в себя:
            \begin{scnitemize}
                \item хранилище \textit{sc-текстов} (\textit{sc-хранилище}, хранилище знаковых конструкций, представленных SC-коде);
                \item файловую память \textit{sc-машины};
                \item средства, обеспечивающие взаимодействие \textit{sc-агентов} над общей памятью (sc-памятью);
                \item базовые средства интерфейса для взаимодействия системы с внешним миром (пользователем или другими системами). Указанные средства включают в себя, как минимум, редактор, транслятор (в sc-память и из нее) и визуализатор для одного из базовых универсальных вариантов представления \textit{SC-кода} (\textit{SCg-код}, \textit{SCs-код}, \textit{SCn-код}), средства, позволяющие задавать системе вопросы из некоторого универсального класса (например, запрос семантической окрестности некоторого объекта);
                \item реализацию \textit{Абстрактной scp-машины}, то есть интерпретатор \textit{scp-программ} (программ Языка SCP).
            \end{scnitemize}
            При необходимости, в \textbf{\textit{платформу интерпретации sc-моделей компьютерных систем}} могут быть заранее на платформенно-зависимом уровне включены какие-либо компоненты машин обработки знаний или баз знаний, например, с целью упрощения создания первой версии \textit{прикладной ostis-системы}.Реализация платформы может осуществляться на основе произвольного набора существующих технологий, включая аппаратную реализацию каких-либо ее частей. С точки зрения компонентного подхода любая \textbf{\textit{платформа интерпретации sc-моделей компьютерных систем}} является \textbf{\textit{платформенно-зависимым многократно используемым компонентом}}.}
        \scnsuperset{программный вариант реализации платформы интерпретации sc-моделей компьютерных систем}
        \scnsuperset{семантический ассоциативный компьютер}
        \begin{scnsubdividing}
            \scnitem{однопользовательский вариант реализации платформы интерпретации sc-моделей компьютерных систем}
            \begin{scnindent}
                \scnidtf{вариант реализации платформы интерпретации sc-моделей компьютерных систем, рассчитанный на то, что с конкретной ostis-системой взаимодействует только один пользователь (владелец)}
                \scntext{примечание}{При таком варианте реализации платформы оказывается невозможным реализовать некоторые важные принципы \textit{Технологии OSTIS}, например, коллективную согласованную разработку базы знаний системы в процессе ее эксплуатации. При этом могут использоваться различные сторонние средства, например для разработки базы знаний на уровне исходных текстов.}
            \end{scnindent}
            \scnitem{многопользовательский вариант реализации платформы интерпретации sc-моделей компьютерных систем}
            \begin{scnindent}
                \scnidtf{вариант реализации платформы интерпретации sc-моделей компьютерных систем, рассчитанный на то, что с конкретной ostis-системой одновременно или в разное время могут взаимодействовать разные пользователи, в общем случае обладающие разными правами, сферами ответственности, уровнем опыта, и имеющие свою конфиденциальную часть хранимой в базе знаний информации}
            \end{scnindent}
        \end{scnsubdividing}
        \scnheader{платформа интерпретации sc-моделей компьютерных систем}
        \begin{scnsubdividing}
            \scnitem{программный вариант реализации платформы интерпретации sc-моделей компьютерных систем}
            \begin{scnindent}
                \scnidtf{программная платформа интерпретации sc-моделей ostis-систем}
                \scnidtf{программный базовый интерпретатор sc-моделей ostis-систем}
            \end{scnindent}
            \scnitem{семантический ассоциативный компьютер}
            \begin{scnindent}
                \scnidtf{аппаратная платформа интерпретации sc-моделей ostis-систем}
                \scnidtf{аппаратно реализованный базовый интерпретатор sc-моделей ostis-систем}
            \end{scnindent}
        \end{scnsubdividing}
        \bigskip
    \end{scnsubstruct}
\end{SCn}
